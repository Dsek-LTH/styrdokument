\documentclass{dsekprotokoll}

\usepackage[T1]{fontenc}
\usepackage[utf8]{inputenc}
\usepackage[swedish]{babel}
\usepackage{multicol}

\setheader{Jämlikhetspolicy}{Policydokument}{}

\title{Policy för Jämlikhet}
\author{}

\begin{document}
\section*{Jämlikhetspolicy}
\section{Formalia}
\subsection{Sammanfattning}
Policyn beskriver vilka värderingar D-sektionen har och verkar för.
\subsection{Syfte}
D-sektionen vid Lunds Tekniska Högskola har en omfattande verksamhet, exempelvis studiebevakning, nollningsaktiviteter, sittningar och resor. Det är viktigt att all verksamhet
genomsyras av en öppen och välkomnande inställning, där alla medlemmar behandlas med
respekt och med ett jämlikt förhållningssätt.
\subsection{Omfattning}
Omfattar sektionen i sinhelhet men där styrelsen är ansvariga för att policyn efterföljs.
\subsection{Historik}
Antagen på S09 2013.
Uppdaterad enl. Policy för Policyer på HTM2 2021. Uppdaterad enl. Policy för styrdokument på VTM-extra 2023.


\section{D-sektionen skall verka för}

\begin{attlista}
	\item verksamheten präglas av jämlikhet.
	\item verksamheten i möjligaste mån planeras på ett sådant sätt att alla får samma möjlighet att delta.
	\item en diskussion om jämlikhet fortlöpande hålls inom organisationen.
	\item dess funktionärer är medvetna om vilka attityder och värderingar man förmedlar till
	medlemmarna.
	\item verka för att kränkande händelser rapporteras och följs upp, samt att i möjligaste mån
	förhindra att liknande situationer uppstår.
	\item ingen negativ särbehandling sker med avseende på kön, ålder, könsöverskridande identitet eller uttryck, etnisk tillhörighet, funktionshinder, trosuppfattning, sexuellt läggning,
	politisk uppfattning eller social särbehandling på grund av studieresultat inom sektionen.
	\item påtvingad alkoholkonsumtion ej accepteras inom sektionen.
	\item sektionen ska tillåta positiv särbehandling, så länge detta främjar sektionen i sin helhet.
	\item trakasserier eller andra kränkande moment ej accepteras inom sektionen. En händelse
	eller miljö är att betrakta som kränkande ifall någon uppfattar den som sådan.
\end{attlista}



\end{document}