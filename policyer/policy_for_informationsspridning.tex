\documentclass{dsekprotokoll}

\usepackage[T1]{fontenc}
\usepackage[utf8]{inputenc}
\usepackage[swedish]{babel}
\usepackage{multicol}

\setheader{Policy för informationsspridning}{Policydokument}{Lund -- 27 januari 2013}

\title{Policy för Informationsspridning genom sektionens kanaler}
\author{Jens Elofsson}

\begin{document}

\section*{Policy för informationsspridning genom sektionens kanaler}
\section{Formalia}
\subsection{Sammanfattning}
Policyn ger en beskrivning hur informationsspridning skall ske utifrån olika situationer.
\subsection{Syfte}
Denna policy togs fram under diskussion mellan berörda parter från sektioner och Informationsansvarige på TLTH. Syftet med policyn är att göra det lättare för annonsörer att få klarhet i vad som gäller, då det tidigare skiljt sig från sektion till sektion. Infokollegiet 2012 tog fram följande policy för informationsspridning genom de kanaler som finns under TLTH. Texten är omformulerad för att passa för D-sektionen.

\subsection{Omfattning}
Eftersom policyn inte har antagits på något möte så omfattar policyn inget men agerar behjälplig i omfattade situationer beskrivna nedan.
\subsection{Historik}
Uppskattad historik:\\
Skriven 2012, uppdaterad och genomarbetad 2013. \\
Upplagd på hemsidan den 15 februari 2015. \\
Uppdaterad enl. Policy för Policyer på HTM2 2021. Uppdaterad enl. Policy för styrdokument på VTM-extra 2023.


\textbf{D-sektionen}\\
Varje mästare inom sektionen får sätta upp affischer på anslagstavlor, information på
hemsidan eller information på skärmen som denne behagar.


\textbf{Nationer}\\
Nationer bedöms kunna av egen kraft både engagera medlemmar och locka folk till sina
evenemang. Därför kan nationerna få sätta upp sina affischer på allmänna anslagstavlor i
E-huset, men de ska inte erbjudas plats på de av sektionen administrerade informationskanalerna. Akademiska Föreningen (AF), och dess underföreningar, bör ha behandlas på
samma sätt som nationerna.

\textbf{Idrottsföreningar}\\
Idrottsföreningar bör inte tillåtas annonsera genom sektionens kanaler, utan endast på
de allmänna anslagstavlorna. Detta då marknadsförandet av dessa kan konkurrera med
TLTH:s egna verksamhet (t.ex. Teknologkårens IF).

\textbf{Företag}\\
All företagsannonsering (jobberbjudanden, lunchföredrag mm.) ska gå genom sektionens
Industrimästeri eller Teknologkårens näringslivsutskott.

\textbf{Undersökningar/Testpersoner}\\
Infokollegiet fann ingen nytta med att marknadsföra varken enkätundersökningar eller
förfrågningar om testpersoner.

\textbf{Studierelaterat}\\
Med detta menas arrangemang som rör utbytesstudier, kurser eller annat studierelaterat
som inte kommer från LU/LTH. Dessa bör ses som vinstdrivande företag, och ska gå
igenom sektionens Industrimästeri eller Teknologkårens näringslivsutskott.

\textbf{LU/LTH}\\
Material som publiceras från LU/LTH ska i stor mån marknadsföras så länge det gäller
större evenemang, såsom information om utbytesstudier eller liknande. Marknadsföring
av enskilda kurser anses inte vara av något stort intresse för sektion, och den bör ske i
diskussion med Studierådsordförande.

\textbf{Kultur}\\
Ansvaret för att marknadsföra dessa sällskap och grupper anses inte ligga på sektionen,
och bör hänvisas till de allmänna tavlorna.

\newpage

Lund, dag som ovan\\

Jens Elofsson \\
Sekreterare, tillika Propagandamästare

\end{document}