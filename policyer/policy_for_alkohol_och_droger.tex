\documentclass[]{dsekprotokoll}
\usepackage[T1]{fontenc}
\usepackage[utf8]{inputenc}
\usepackage[swedish]{babel}
\usepackage{float}
\usepackage{tabularx}
\usepackage{array}
\usepackage{hyperref}

\setheader{Policy för alkohol och droger}{Policydokument}{}

\title{Policy för alkohol och droger}
\author{Linus Åbrink}

\begin{document}
\maketitle

\section{Formalia}

\subsection{Sammanfattning}
Denna policy beskriver hur D-sektionen inom TLTH arbetar med frågor som rör alkohol och droger, ställningstaganden samt visioner med arbetet.

\subsection{Syfte}
D-sektionen inom TLTH är en ideell förening som arrangerar evenemang
med syftet att främja sektionens medlemmars studiesociala miljö. Dessa evenemang tar ibland formen av eftersläpp, pubar och sittningar. Vid dessa tillställningar kan det förekomma försäljning av alkoholhaltiga drycker. Syftet med denna policy är att främja en ansvarsfull alkoholservering vid tillställningar anordnade av D-sektionen samt att göra ett aktivt ställningstagande mot användande av narkotika. D-sektionen ställer sig även bakom TLTH:s alkohol- och drogpolicy. Denna policy bör ses som ett komplement till TLTH:s policy

\subsection{Historik}
Utkast färdigställt av: Linus Åbrink

Ursprungligen antagen enligt beslut: VTM Extra 2021.

Uppdaterad enl. Policy för styrdokument på VTM-extra 2023.

Uppdaterad enligt beslut på VTM-extra 2023.

Omarbetning fastställd enligt beslut (samt motionär):
\begin{itemize}
    \item VTM 2024 (Måns Bard Nilsson, Sexmästare)
\end{itemize}

\section{Alkoholservering}
\subsection{Allmänt}
Denna policy utgår i från alkohollagen (SFS 2010:1622) som ska följas vid alla arrangemang anordnade av D-sektionen. Med anledning av detta ska den som är serveringsansvarig under ett arrangemang där alkoholförsäljning förekommer ha god kunskap om innehållet i alkohollagen, i synnerhet de kapitel som handlar om servering av alkohol samt tillsyn av serveringsställen. Det åligger den serveringsansvarige att tillse att de som jobbar under ett arrangemang har tillfredställande kunskap om alkohollagens innehåll, i synnerhet de kapitel som handlar om servering av alkoholhaltiga drycker. För att säkerställa att den serveringsansvarige har en tillfredsställande kunskap om alkohollagens innehåll bör alla som är anmälda som serveringsansvariga hos tillståndsmyndigheten ha genomgått en av kommunens kurser om ansvarsfull alkoholservering.

\subsection{Under arrangemangets gång}
Den serveringsansvarige bör kontinuerligt under arrangemangets gång kontrollera att nödutgångar och brandgångar ej är blockerade samt tillse att ordning upprätthålls och att en hög berusningsgrad hos gästerna undviks. I det fallet att en gäst skulle dricka sig för berusad ska
denne avvisas från tillställningen. 

På ett serveringsställe där serveringstillstånd gäller får inte någon inta eller tillåtas inta andra alkoholdrycker eller alkoholdrycksliknande preparat än sådana som har serverats i enlighet med tillståndet.

\subsection{Alkoholfria alternativ}
Vid varje tillställning anordnad av D-sektionen där alkoholservering förekommer ska det finnas vatten att tillgå för gästerna. Alkoholfria alternativ, exempelvis alkoholfri öl och cider ska finnas i den omfattning som alkohollagen föreskriver. Vinstmarginalen för alkoholfria drycker får aldrig vara större än den för drycker som innehåller alkohol. Detta innebär exempelvis att priset på sittningsbiljetter bör ändras om inköpspriset för det alkoholfria alternativet är annat än alternativet som innehåller alkohol.

\section{Ställningstagande mot narkotika}
D-sektionen tar avstånd från all användning av narkotikaklassade medel annat än för medicinskt bruk som sker enligt läkares föreskrift. Om någon gäst skulle ertappas med att missbruka narkotika ska denne avvisas från tillställningen. Dessutom ska en polisanmälan göras av den serveringsansvarige.

\end{document}