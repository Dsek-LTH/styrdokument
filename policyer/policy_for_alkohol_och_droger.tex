\documentclass[]{dsekprotokoll}
\usepackage[T1]{fontenc}
\usepackage[utf8]{inputenc}
\usepackage[swedish]{babel}
\usepackage{float}
\usepackage{tabularx}
\usepackage{array}
\usepackage{hyperref}

\setheader{Policy för alkohol och droger}{Policydokument}{}

\title{Policy för alkohol och droger}
\author{Linus Åbrink}

\begin{document}
\maketitle

\section{Formalia}

\subsection{Sammanfattning}
Denna policy beskriver hur D-sektionen inom TLTH arbetar med frågor som rör alkohol och droger, ställningstaganden samt visioner med arbetet.

\subsection{Syfte}
Syftet med denna policy är att ge stöd till alla funktionärer som på något sätt hanterar alkohol
i sina uppdrag samt att klargöra D-sektionens visioner kring arbetet med en sund alkoholkultur.

\subsection{Omfattning}
Sektionen i sin helhet.

\subsection{Historik}
Utkast färdigställt av: Linus Åbrink

Ursprungligen antagen enligt beslut: VTM Extra 2021.

Uppdaterad enl. Policy för styrdokument på VTM-extra 2023.

Uppdaterad enligt beslut på VTM-extra 2023.

\section{Bakgrund}
D-sektionen inom TLTH är en ideell förening som samlar cirka 800 medlemmar. D-sektionens huvudsyfte är som sektion att främja medlemmarnas studier och vad därmed äger sammanhang. Utöver detta arrangeras emellertid omfattande sociala aktiviteter inbegripande mässor, utflykter, fester och nollning för nya studenter där en stor del av aktiviteterna på något sätt innehåller mer eller mindre alkohol.

\section{Vision}
D-sektionen ska främja en sund inställning till alkohol som minskar risken för sociala och
medicinska skadeverkningar.\footnote{Målet för den svenska alkoholpolitiken och det som ligger till grund för nuvarande lagstiftning. Se till exempel Folkhälsomyndighetens utbildningsmaterial Ansvarsfull alkoholhantering:\url{https://www.folkhalsomyndigheten.se/livsvillkor-levnadsvanor/andts/tillsynsvagledning/alkoholdrycker/ansvarsfull-alkoholservering/}
}

Att välja alkoholfritt ska alltid vara ett naturligt alternativ och ingen skall känna sig uppmanad eller tvingad att inmundiga alkohol för att passa in på D-sektionens arrangemang. Deltagare ska därför alltid göra ett aktivt val om de vill ha alkohol.

Berusningsgraden ska hållas på en trivsam nivå.\footnote{Alkohollagen (2010:1622) Kap 3 §5} D-sektionens festverksamhet ska vara
roliga och välkomnande för alla.

D-sektionens fester ska ses som ett föredöme inom studentvärlden.

D-sektionen ska ha nolltolerans mot alla typer av narkotika och stämningshöjande substanser utöver tobak och alkohol för annat än medicinskt bruk, och allt bruk av dessa räknas som missbruk.\footnote{Information om vilka preparat som räknas som narkotika finns på Läkemedelsverkets hemsida.}

D-sektionen ska verka för att öka medvetenheten om riskerna med överkonsumtion av
berusningsmedel och arbeta för att minimera olägenheter orsakade av alkohol och narkotika.

Denna policy har som underlag Teknologkårens policy för alkohol och droger och skall uppdateras i samband med att nya policy antas av Teknologkåren. Policyn skall i största mån följa policy och riktlinjer från Teknologkåren med förbehåll för kulturella och strukturella skillnader.\footnote{\url{https://drive.google.com/file/d/1ExHXDDoZDZzZ_eD1ZnrtD9qzjfv_cj2Z/view?usp=sharing}}

Denna policy har sin utgångspunkt i alkohollagen (SFS 2010:1622), narkotikastrafflagen
(SFS 1968:64) samt Lunds kommuns riktlinjer för serveringsställen vilka självfallet alltid ska
följas. Myndighetsrekommendationer bör också alltid beaktas.

\section{Handlingsplan}
\subsection{Alkoholfria alternativ}
Alkoholfria drycker och lättdrycker\footnote{Dryck med en volymprocent under 2.25 benämns lättdrycker, Alkohollagen (2010:1622) 1 kap 5 §} skall finnas att tillgå i tillfredsställande urval\footnote{Alkohollagen (2010:1622) 8 kap 22 §}. Ett varierat utbud av lättdrycker innebär att det ska finnas minst ett lättdrycksalternativ till respektive slag av alkoholdryck som erbjuds i verksamheten\footnote{Riktlinjer för serveringstillstånd i Lunds kommun. Kap 4. Sida 6.}. Detta innebär att om exempelvis öl, cider, rött
vin och vitt vin serveras krävs det att det finns minst ett lättdrycksalternativ till alla dessa.

Alkoholfritt alternativ skall aldrig bekosta alkohol och skall vara billigare där sådan prisskillnad motiveras av inköpspriset, detta gäller även vid försäljning av sittningsbiljetter. Prissättningen skall alltid vara sådan att starkare drycker inte främjas. Vatten skall alltid finnas tillgängligt utan kostnad.

\subsection{Servering}

Alkohol skall serveras med återhållsamhet och serveringen ska avbrytas innan gästen blir så
onykter att denna måste avvisas. Strävan skall vara att hålla berusningsnivån vid ''salongsberusning'' eller lägre. Det är dags att sluta servera innan dess att gästen är ''märkbart påverkad''\footnote{Alkohollagen (2010:1622) 3 kap 8 §
}. Tecken på sådan berusning är att gästen talar otydligt eller högljutt, går ostadigt, blir fumlig eller slumrar till. När en gäst är märkbart påverkad skall gästen enligt lag avvisas\footnote{Alkohollagen (2010:1622) 3 kap 8 §}. Se råd och tips nedan för mer information om berusningsgrader.

En person som trots allt fått för mycket att dricka skall avvisas från lokalen. Ombesörj gärna att personen kommer hem genom sällskap, skjutsa hem eller hjälp till att beställa en taxi. Serveringspersonalen kan stå ansvariga för att en gäst skadat sig på väg hem från tillställningen om personen i fråga har blivit överserverad.

För att säkerställa en säker och trevlig miljö är personal som handskas med alkohol givetvis nykter på arbetstid och förtär heller ingen alkohol under arbetstid\footnote{Alkohollagen (2010:1622) 8 kap 18 §}.

Vid serveringstillstånd till slutet sällskap ska det finnas minst en tillredd maträtt samt ett par
enklare alternativ som erbjuds gästerna under hela serveringstiden\footnote{Riktlinjer för serveringstillstånd i Lunds kommun. Kap 4. Sida 6.}.

Animering – aktiv påverkan för ökad försäljning – skall inte förekomma under några former. Det innebär till exempel att försäljning av s.k. spritpaket till ett billigare pris än varorna var för sig ej är tillåten. Gästerna får inte uppmanas eller förmås att köpa alkoholdrycker\footnote{Alkohollagen (2010:1622) 8 kap 21 §}. Det
är dock tillåtet att upplysa om sista beställning i samband med att baren stänger.

D-sektionens gäster ska aldrig uppmanas att köpa alkohol och någon typ av alkoholhets
ska aldrig förekomma på D-sektionens event. Förköpt alkoholhaltig
dryck ska, om möjligt, kunna återköpas. Annars ska de gå
att lösa in mot likvärdigt alkoholfritt alternativ om man väljer att inte utnyttja dem

Uppmuntra gärna till varannan vatten eller dylika kampanjer.

\subsection{Serveringsställen och andra utrymmen}
På ett serveringsställe där serveringstillstånd gäller får inte någon inta eller tillåtas inta andra
alkoholdrycker eller alkoholdrycksliknande preparat än sådana som har serverats i enlighet
med tillståndet.

Alkoholdrycker eller alkoholdrycksliknande preparat som inte får serveras i en lokal som
avses i denna paragraf får inte heller förvaras i lokalen eller tillhörande utrymmen\footnote{Alkohollagen (2010:1622) 8 kap 24 §}.

Detta innebär ett strikt förbud mot förtäring av medhavd alkohol i D-sektionens lokaler
kopplat till det stadigvarande tillståndet i Pub iDét.

\subsection{Säkerhet}
Nödutgångar måste vara tydligt utmärkta och ej blockerade.

God ordning och nykterhet skall råda i lokalen. Det är förbjudet att ta med alkohol ut ur eller in i våra serveringslokaler\footnote{Alkohollagen (2010:1622) 8 kap 23 §}.

Vid framkomst av sexuella trakasserier, annan diskriminering eller beteende som påverkar
andras trygghet i lokalen skall den anklagade avvisas från platsen på ett betryggande sätt.
D-sektionens jämlikhetspolicy och Teknologkårens policy för likabehandling ska genomsyra all verksamhet då alkohol förekommer. Det är nolltolerans mot alla former av diskriminering och kränkande beteende på D-sektionen och denna typen av beteende kan leda till avstängning från
D-sektionens och Teknologkårens evenemang.

Om situationer av ovanstående art skulle uppkomma har Teknologkåren en krishanteringsplan som kan användas för hanteringen av situationen.

\subsection{Utbildning}
Serveringsansvariga för D-sektionens alkoholtillstånd ska ha en enligt Aktivitetssamordnare vid Teknologkåren tillfredsställande kunskap i alkoholhantering och alkohollagen, förslagsvis genom att de gått någon av AF, KK och LUS alkoholutbildningar, som kommunen medverkar i. I regel anses A- och B-cert som tillfredsställande kunskap i alkoholhantering och alkohollagen för att vara serveringsansvarig.

\subsection{Narkotika}
Missbruk eller innehav av droger skall omedelbart anmälas till festarrangör och vakter eller
direkt till polis om situationen kräver detta.

Alla festarrangörer ska ha kunskap om hur de kontaktar polisen om en narkotikapåverkad person påträffas i serveringslokalen.

\subsection{Droger och skadligt bruk}
Med droger menas här all icke-medicinsk användning av läkemedel samt all användning av
narkotika och anabola androgena steroider. All användning av dessa skall betraktas som
missbruk och även användning av andra stämningshöjande medel som kan medföra fara för
liv eller hälsa och som används eller kan antas användas för att uppnå en stämningshöjande
effekt eller annan påverkan är otillåten.

Med skadligt bruk avses en konsumtion som är skadlig för hälsan, ekonomin och som ger negativa konsekvenser i sociala sammanhang, både på medarbetar- och organisationsnivå.

D-sektionen tar avstånd från all form av droger och att inneha eller använda droger på event
som arrangeras av D-sektionen, i D-sektionens lokaler eller på annan plats som kan associeras med D-sektionen är otillåten.

\subsection{Tecken på alkohol- och drogproblem}
Exempel på tidiga tecken vid alkohol- och drogproblem kan vara:\footnote{Tecken på drogproblem, Prevent, \url{https://www.prevent.se/amnesomrade/alkohol-och-andra-droger/tecken-pa-drogproblem/}
}
\begin{itemize}
    \item \emph{Förändrat beteende} - rastlöshet, nervositet, håglöshet, nedstämdhet, retlighet eller tendens att isolera sig.
    \item \emph{Sämre arbetsprestation} - sämre koncentration, fler misstag, glömmer saker.
    \item \emph{Frånvaro} - ökad frånvaro, tillfällig frånvaro, sen ankomst, går för tidigt, söker semester och kompensationsledighet samma dag eller i efterskott, är sjukfrånvarande i samband med helger och/eller löneutbetalning.
    \item \emph{Försvarar droganvändning} - olika sätt att försvara sitt skadliga bruk eller beroende.
    \item \emph{Problem i arbetsgruppen} - bortförklaringar, frånvaro som påverkar arbetet, svårt att samarbeta.
\end{itemize}

\subsection{Vid misstanke om eller påkommande av skadligt bruk hos funktionärer och medlemmar}

Vid misstanke om att en funktionär/ medlem har problem med skadligt bruk är det alltid rätt att handla. Tala först med individen själv om din oro, om ingen förbättring sker, tala med D-sektionens styrelse, likabehandlingsombud eller Aktivitetssamordnaren vid Teknologkåren direkt eller med hjälporgan Studenthälsan, Studentprästerna eller LTHs kuratorer för vidare stöttning.

Vid misstanke om innehav eller bruk av droger skall ansvarig person på plats omedelbart informeras och konfrontation ska ske när tillfälle ges. Vid stark misstanke bör personen avvisas från platsen på ett betryggande sätt. En dialog ska hållas med den aktuella funktionären/ medlemmen/gästen där situationens allvar tydligt framgår. Aktivitetssamordnaren vid Teknologkåren finns till förfogande vid hanteringen av dessa samtal.

Vid påkommande av bruk av droger skall festarrangör och vakter omedelbart informeras och brukaren avvisas från eventet på ett betryggande sätt. Vid användandet av olagliga substanser ska polis kontaktas. D-sektionen har ingen förpliktelse att rapportera om missbruk till Lunds Universitet men ska informera studenten om var denne kan vända sig för att få hjälp och stöttning\footnote{Alkohol och andra droger – Gemensamma riktlinjer och rutinbeskrivning vid Lunds universitet}.

En funktionär som ertappats med att inneha och/eller vara påverkad av droger i egenskap av funktionär eller i anknytning till sektionsevenemang eller i D-sektionens lokaler kan komma att bli avstängd eller entledigad från sin(a) funktionärspost(er). Övriga medlemmar kan komma att nekas tillträde till event arrangerade av D-sektionen och Teknologkåren i upp till ett år\footnote{Beslutet om avstängning från Teknologkårens event  tas i samråd med Aktivitetssamordnaren vid Teknologkåren}.

\subsection{Kontaktuppgifter}
\textbf{Kontaktlista inom Lunds universitet:}\newline
Studenthälsan tel. 046-222 43 77 \newline
Studentprästerna tel. 046-718735\newline
Kuratorer LTH, tel. 046-222 37 14, 046-222 72 47

\textbf{Kontaktlista utanför Lunds universitet:}\newline
AA Anonyma alkoholister tel. 08-720 38 42\newline
Beroendecentrum, SUS, Malmö, tel. 040- 33 16 00\newline
Rådgivningsbyrån SUS Lund, tel. 046- 17 89 30\newline
Sjukvårdsupplysningen tel. 1177

\subsection{Samarbete med artister}
Återkommande har D-sektionen samarbeten med artister som spelar på diverse event och eftersläpp. Det åligger D-sektionen att inte ge artisten alkohol som betalning för ett uppträdande. D-sektionen ska inte heller bevilja att artistens privata gäster särbehandlas
från övriga gäster i lokalen. Artisten ska framföra sitt uppträdande utan att vara märkbart
påverkad och vid misstanke om drogpåverkan kan artisten komma att nekas uppträda.

\subsection{Tackfester och arbetsglädje}
Alkohol skall undvikas som belöning till funktionärer. Det är dock tillåtet att bjuda på alkohol i samband med tackfester, men det skall vara i rimliga mängder och vara inom ramarna för intern representation. Gratis alkohol skall aldrig vara den stora begivenheten för en fest.

\subsection{Övrigt}
Studenter är överrepresenterade i riskgruppen, det vill säga gruppen av människor som riskerar skador eller beroende på grund av överkonsumtion. Personer i riskzonen ska uppmärksammas på problemet, att lägga sig i är att bry sig om. Studenthälsan kan bistå med råd och stöd åt studenter som har eller misstänker sig ha alkoholproblem.

\section{Alkoholinköp}
Vid alkoholinköp gäller samma lagar som reglerar företags alkoholinköp. Med alkoholinköp avser
både det som är till försäljning och det som inte är det. Företag får endast köpa alkohol av godkända
svenska återförsäljare. Alkoholinköp från annat land än Sverige får endast ske av behöriga företag,
vilket D-sek inte är, och om alkoholskatt betalas. Enligt alkohollagen (4 kap, §4) och smugglingslagen (§3), riskerar enskilda medlemmar att dömas för smugglingsbrott bara genom att följa med
om en förening skulle importera alkohol. Sektionen riskerar även att dömas för bokföringsbrott
och/eller skattebrott genom att köpa alkohol för föreningens pengar i annat land än Sverige. Sektionen skulle även förlora sitt serveringstillstånd, vilket skulle vara väldigt tråkigt, därför gäller
följande:

\begin{attlista}
    \item styrelsen inte kommer godkänna utläggsräkningar på inköp av alkohol på några andra ställen än godkända svenska återförsäljare.
    \item försäljning av alkohol på evenemang i sektionens namn endast får ske då ett serveringstillstånd finns, detta gäller även om alkoholen inte skulle vara inköpt av sektionen.
\end{attlista}

\section*{APPENDIX}
\section{Råd och tips}
Detta kapitel innehåller tips och råd om gällande bestämmelser och hur man lever upp till
handlingsplanen.

\subsection{Gällande rätt}
Alkoholhantering regleras av Alkohollagen (2010:1622) som finns tillgänglig på internet.
Kapitel 1, 3, 6, 7 och 8 är de som är mest relevanta för vår verksamhet. Folkhälsomyndigheten (tidigare Statens Folkhälsoinstitut) är tillsynsmyndighet och har därmed tolkningsföreträde och rätt att utge föreskrifter. Tillståndsenheten i Lund är den som i praktiken beviljar eller inte beviljar alkoholtillstånd.

\subsection{Tillstånd}
Finns serveringstillstånd för en given lokal får aldrig alkohol som ej är inköpt av tillståndshavaren finnas i lokalen, oavsett om det för tillfället är en pub eller ej\footnote{Alkohollagen (2010:1622) 8 kap 24 §}.

Serveringstillstånd ansöker man om hos tillståndsenheten vid Lunds kommun. Handläggningstiden på alkoholtillstånd är två veckor så se till att tillståndsansökan skickas i tid.

Observera att tillstånden D-sektionen beviljas oftast gäller enbart för Teknologkårens medlemmar. Finns en gästlista är den ofta beviljad för en specifik grupp, till exempel företagsrepresentanter, och det är då viktigt att det bara är personer ur den givna gruppen som står på gästlistan.

För att uppfylla bokföringskraven i alkohollagen måste kassaregister användas vid all
serveringsverksamhet på D-sektionen.

Alkoholtillståndet skall finnas lätttillgängligt i lokalen.

Lunds kommunfullmäktige har i september 2019 antagit ett dokument som kallas ''Riktlinjer för serveringstillstånd i Lunds kommun''. Detta dokument finns tillgängligt på internet och förklarar hur kommunen arbetar med och tolkar alkohollagen\footnote{\url{https://www.lund.se/globalassets/lund.se/foretagare/tillstand-regler-och-
        tillsyn/serveringstillstand/riktlinjer-serveringstillstand-2019.pdf.}
}.

\subsection{Alkoholfria alternativ}
Det finns en uppsjö av alkoholfria alternativ som komplement till läsk. Systembolaget har en speciell avdelning med alkoholfria viner, både röda, vita och rosé, cider, öl och till och med snaps. Lättöl och juice är också bra alternativ. Det är önskvärt att sträva efter att ha samma standard på det alkoholfria utbudet som på det alkoholhaltiga. Serveras ett vitt vin till förrätten, ett rött till huvudrätten och ett dessertvin till efterrätten kan man servera alkoholfritt vitt respektive rött vin till förrätt och huvudrätt, och alkoholfri cider till efterrätten. Finns dessutom öl till servering skall även alkoholfri öl vara en del av utbudet
Serveras drinkar skall det även finnas alkoholfria drinkar. Förslag på recept finns bland annat
på Systembolagets hemsida.

\end{document}