\documentclass[]{dsekkallelse}
\usepackage[T1]{fontenc}
\usepackage[utf8]{inputenc}
\usepackage[swedish]{babel}
\usepackage{graphicx}
\usepackage{longtable}

\setheader{Policy för överlämning}{Policy}{Lund, \today}

\title{Policy för överlämning}
\author{Joel Bäcker \& Josefin Wetterstrand}
\begin{document}
\section{Formalia}
\subsection{Sammanfattning}
Denna policy beskriver hur överlämningar på D-sektionen ska gå till.

\subsection{Syfte}
Syftet med denna policy är att ge funktionärer rättigheter och säkerhet kring att överlämning sker samt vad den i grund och botten ska innehålla.

\subsection{Omfattning} % KANSKE SKA ÄNDRAS
Gäller för funktionärer som har blivit invalda på ett sektionsmöte, funktionärer som är kärnposter samt styrelsen som grupp. Poster som är periodiserade, så som Jubileumsgeneral och Karnevalsanvarig, får man ha överseende med gällande dessa överlämningar.

\subsection{Ägande}
Styrelsen äger policyn i sin helhet och har rätt att behandla och ändra denna på styrelsemöten.

\subsection{Historik}
Utkast färdigställt av Joel Bäcker, Skattmästare 2021, Josefin Wetterstrand, Vice Ordförande 2021, David Jobrant, Informationsansvarig 2021 och Victor Winkelmann, Näringslivsansvarig 2021.

\section{Bakgrund}
Överlämning bland funktionärer har varierat i kvalitet under de gångna åren. Denna policy ämnar därför garantera att varje överlämning innehåller särskilda moment som säkerställer att den nyinvalda har tillräcklig kunskap och är införstådd om vad sektionsarbetet innebär.

\section{Överlämning}

\subsection{Moment i överlämning}
Följande moment ska gås igenom under överlämningen med den tillträdande funktionären:
\begin{itemize}
    \item Vad som åligger posten enligt styrdokumenten.
    \item Postens testamente.
    \item Ekonomiska rutiner.
    \item Rundvandring av de lokaler som posten använder sig av, samt vilken access posten ger tillgång till. Eventuella regler för access och lokalerna som berörs bör också genomgås.
    \item Vilka förmåner som ges med posten. Detta kan exempelvis vara tack, tröjor och medaljer.
    \item Kunskapsbank för posten. Detta kan exempelvis vara inventeringslistor och andra dokument från det gångna året.
\end{itemize}
\subsection{Tid}
Momenten i överlämningen ska ha genomförts innan personen tillträder sin funktionärspost. I händelse av att personen har fyllnadsvalts kan man frångå detta, men överlämningen ska ske snarast möjligen.

\subsection{Ansvarig}
Ansvarig för att överlämningen genomförs är i första hand nuvarande innehavaren av posten. Då denna post är vakant åläggs det styrelsen att detta sker. Det är upp till styrelsen att från fall till fall bedöma om de själva kan arrangera överlämningen eller om man kan be någon annan om hjälp, exempelvis äldre företrädare.

\section{Testamente}
Testamente är ett dokument som ska fungera som stöd för en funktionär som är ny på en post. Det är en sammanfattning av postens arbetsuppgifter, tips på hur dessa kan genomföras samt erfarenheter från tidigare år. Testamentet bör uppdateras vid slutet av varje mandatperiod med syftet att föras vidare till efterträdaren.

Inom vissa utskott kan det finnas ett övergripande testamente som täcker hela mästeriets verksamhet. I dessa fall kan de två komplettera varandra så alla punkter nedan täcks.

\subsection{Innehåll}
Följande delar bör ingå i ett testamente:
\begin{itemize}
    \item Vad som ska göras innan mandatperiodens början
    \item Postens ansvarsområden och återkommande uppgifter
    \item Oavslutade arbetsuppgifter och projekt som lämnas över
    \item Tidslinje, i den mån det är relevant
    \item Hemsidor och dylikt som man ska ha tillgång till
    \item Kontaktuppgifter till externa parter
\end{itemize}

\subsection{Arkivering}
Testamentet bör förvaras åtkomligt så framtida efterträdare och mästerier kan ta del av det.

\end{document}