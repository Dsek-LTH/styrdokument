\documentclass{dsekprotokoll}

\usepackage[T1]{fontenc}
\usepackage[utf8]{inputenc}
\usepackage[swedish]{babel}
\usepackage{multicol}

\setheader{Policy för val}{Policydokument}{}

\title{Policy för val}
\author{Anna Qvil}

\begin{document}

\maketitle
\section{Formalia}
\subsection{Sammanfattning}
Denna policyn beskriver hur valprocessen skall gå till både för val gjorda av sektionsmötet och styrelsen. Den täcker även hur processen inför valen skall gå till.

\subsection{Syfte}
Syftet med denna policy är att förtydliga stadgan och reglementet samt att vägleda valberedningen samt andra tillfälliga valberedningar i sitt arbete

\subsection{Omfattning}
Alla val som behandlas på sektionen.

\subsection{Historik}
Utkast färdigställt av: Anna Qvil \\
Ursprungligen antagen enligt beslut: VTM extra 2019.
Uppdaterad: HTM1 2021, S13 2021, HTM2 2021 (enl. Policy för Policyer), VTM 2023, VTM-extra 2023. Uppdaterad enl. Policy för styrdokument på VTM-extra 2023. HTM1 2023.


\section{Val på sektionsmöten}

\subsection{Utlysning}
Utlysning av de poster som går att söka på sektionsmötet skall anslås senast fyra veckor innan mötet. Valberedningens ordförande ansvarar för att utlysningen sker.

Utlysningen ska innehålla en beskrivning av posten, mandatperiod, hur man ansöker, sista  ansökningsdag, datum för valet samt vem som kan kontaktas för ytterligare frågor. Ansökan ska vara öppen minst 5 dagar.

Senast i samband med utlysning av en kärnpost eller en post som tillsätts av ett sektionsmöte ska kravprofil finnas tillgänglig. Vid val där gruppsammansättning vägs in ska en kravprofil för gruppen också finnas
tillgänglig i samband med utlysningen.

\subsection{Beredning}
Val som uträttas av sektionsmötet bereds av valberedningen med undantag för Likabehandlingsombud, Studerandeskyddsombud och Världsmästare samt att kandidater till valberedningen ej valbereds.

Trivselrådets kärnposter valbereds genom att Trivselmästare electus sätter ihop en valberedning på minst tre medlemmar i sektionen bestående av följande: Trivselmästare electus själv, minst en gammal utskottsmedlem och minst en medlem i Valberedningen. Denna valberedning skall stadfästas på HTM-val. Trivselmästare electus ansvarar för att skicka in de funktionärer valberedningen önskar välja in som en handling till Höstterminsmöte två. På sektionsmötet skall de redogöra för hur processen inför valet gått till.

\subsection{Valförfarande}
Motkandidatur mot valberedningens förslag måste anmälas till Talman senast klockan 23.59 två dagar efter valberedningens förslag har offentliggjorts på sektionens informationskanaler. Ämnar man motkandidera mot fler poster ska alla dessa anmälas.

Motkandidatur skall anslås omgående samt den föreslagna kandidaten skall meddelas. Ansvarig för att meddela kandidaten samt anslå motkandidaten är valberedningens ordförande. Endast personer som blivit valberedda för det berörda valet får motkandidera.
Fri nominering tillåts endast till de poster som valberedningen inte har något förslag till.

På mötet skall val av styrelsemedlem ha 7 minuter för presentation och 8 minuter för frågor. Samt att val av annan förtroendepost skall ha 4 minuter för presentation och 5 minuter för frågor.

Vid beredda val då nomineringar inte är offentliga ska statistik över antalet som genomgått en valprocess föras. Statistiken ska ingå i den möteshandling där valberedningen presenterar sitt förslag till mötet. Statistiken ska presenteras i fasta intervall om 5, där 0-4
är det lägsta intervallet, nästföljande är 5-9, sen 10-14 och så
vidare.

\section{Val på styrelsemöten}

\subsection{Utlysning}
Vid första rekryteringstillfället för en post eller de poster i appendix A samt de av sektionsmötet delegerade valen skall utlysningen av valet ske senast 2 veckor innan mötet då valet skall hållas.

När på året de olika posterna bör utlysas samt väljas finns specificerat i appendix B. Poster som inte specificeras i appendix B utlyses när utskottsordförande anser det lämpligt.

Utskottsordförande och i vissa fall utskottsordförande i samråd med utskottsordförande electus ansvarar för att utlysningen sker. Utlysningen skall vara på sådant sätt som ger möjlighet för alla sektionens medlemmar att söka posten.

Utlysningen ska innehålla en beskrivning av posten, mandatperiod, hur man söker, sista ansökningsdag, datum för valet samt vem som kan kontaktas för mer information.

Senast i samband med utlysning av en kärnpost eller en post som valbereds av en
valberedning stadfäst av styrelsen ska kravprofil finnas tillgänglig. Vid val där gruppsammansättning vägs in ska en kravprofil för gruppen också finnas
tillgänglig i samband med utlysningen.

För övriga funktionärer kan de väljas in utan särskild utlysning om så utskottsordförande anser det lämpligt. Dock skall processen för dessa även vara på ett sådant sätt att alla har möjlighet att söka posten.

Vid fyllnadsval av de poster som finns i appendix A skall alltid en utlysning ske. Dock kan styrelsen välja att tillförordna en funktionär till posten under tiden för utlysningen. Övriga poster fyllnadsväljs på så sätt utskottsmästare anser det lämpligt.

\subsection{Beredning}

Alla de poster som finns i appendix A samt av sektionsmötet delegerade val, med undantag för phaddrar, skall beredas enligt följande: 

Utskottsordförande/utskottsordförande electus sätter ihop en valberedning på minst tre medlemmar i sektionen bestående av följande: utskottsordförande/utskottsordförande electus själv, minst en gammal utskottsmedlem och minst en medlem i Valberedningen. Valberedningsrepresentanten ska inte inneha en post i utskottet under samma mandatperiod vars poster den är med och valbereder. Valberedningsrepresentanten bör väljas så att den inte är gift, sambo eller närstående till någon sökande. Denna valberedning skall stadfästas av styrelsen.

Valberedningen för Jubileumsansvarig behöver inte bestå utav en gammal utskottsmedlem.

Utskottsordförande/ utskottsordförande electus agerar valberedningens ordförande i den ovan nämnda valberedningen.

%Den ovan nämnda valberedningen faller under de policys och bestämmelser som finns angående valberedningsarbete.

Phaddrar bereds av en valberedning bestående av Øverphøs och Stabsmedlemmar. Om så
finnes lämpligt kan denna valberedning även bestå av en eller flera Øverpeppare, Peppare
och Valberedningsrepresentanter.

De ovannämnda valberedningarna faller under de bestämmelser som finns i de policyer
vilka berör valberedningsarbete.

Övriga funktionärer bereds på så sätt som utskottsordförande finner lämpligt.

\subsection{Valförfarande}
Utskottsordförande ansvarar för att skicka in de funktionärer hen önskar välja in som en handling till ett styrelsemöte. På styrelsemötet skall utskottsordföranden redogöra hur processen inför valet gått till. På styrelsemötet kan inte motkandidatur ske utan nyutlysning av valet sker ifall processen anses felaktig eller kandidaterna olämpliga.

Valberedningsordförande ansvarar för att sektionsmedlemmar anonymt ska kunna ge
feedback på valprocesserna som sker utanför sektionsmöten. Feedbacken ska i samband
med det att valprocessen presenteras för styrelsen framföras och ska tas i beaktning av
styrelsen. Feedbacken ska inkomma till den som utlyste valet innan det väljande mötet
så att vederbörande har möjlighet att bemöta feedbacken.

Har Valberedningsordförande sökt en post eller sittit med i valberedningen under det berörda valet ska Valberedningsordförande överge ansvaret av att samla in och presentera
feedback till en valberedningsreprsentant. Valberedningsrepresentanten ska inte sökt en
post i det berörda valet eller suttit med i den berörda valberedningen.

Vid beredda val då nomineringar inte är offentliga ska statistik över antalet som genomgått en valprocess föras. Statistiken ska ingå i den möteshandling där valberedningen presenterar sitt förslag till mötet. Statistiken ska presenteras i fasta intervall om 5, där 0-4
är det lägsta intervallet, nästföljande är 5-9, sen 10-14 och så
vidare.

\pagebreak
\subsection*{Appendix A: Kärnposter}
\begin{multicols}{2}

    \subsubsection*{Cafémästeriet}
    \begin{itemize}
        \item Dagsansvarig
        \item Brunchmätare
        \item Inventarieansvarig
    \end{itemize}

    \subsubsection*{Näringslivsutskottet}
    \begin{itemize}
        \item Mentorsansvarig
        \item Medlem i projektgruppen för Teknikfokus
    \end{itemize}

    \subsubsection*{Källarmästeriet}
    \begin{itemize}
        \item Sudo
        \item Root
        \item Vice root
    \end{itemize}

    \subsubsection*{Aktivitetsutskottet}
    \begin{itemize}
        \item Karnevalsansvarig
        \item Sångarstridsförman
        \item Tandemgeneral
    \end{itemize}

    \subsubsection*{Informationsutskottet}
    \begin{itemize}
        \item DWWW-ansvarig
    \end{itemize}

    \subsubsection*{Sexmästeriet}
    \begin{itemize}
        \item Hovmästare
        \item Pubmästare
        \item Vice Pubmästare
        \item Barmästare
        \item Vice Barmästare
        \item Sångförman
        \item Köksmästare
        \item Preferensmästare
        \item Vice Köksmästare
    \end{itemize}

    \subsubsection*{Nollningsutskottet}
    \begin{itemize}
        \item Stabsmedlem
        \item Øverpeppare
        \item Peppare
        \item Phadder
    \end{itemize}

    \subsubsection*{Framtidsutskottet}
    \begin{itemize}
        \item Framtidsledamot
    \end{itemize}

    \subsubsection*{Jubileet}
    \begin{itemize}
        \item Jubileumsansvarig
    \end{itemize}
\end{multicols}

\pagebreak
\subsection*{Appendix B}
\subsubsection*{Poster som bör väljas innan början av verksamhetsåret}
\begin{multicols}{2}

    \subsubsection*{Cafémästeriet}
    \begin{itemize}
        \item Stekare
        \item Funktionär
        \item Brunchmästare
        \item Inventarieansvarig
    \end{itemize}

    \subsubsection*{Källarmästeriet}
    \begin{itemize}
        \item Root
        \item Vice root
        \item Sudo
        \item Trädgårdsmästare
        \item Bilansvarig
    \end{itemize}

    \subsubsection*{Aktivitetsutskottet}
    \begin{itemize}
        \item Idrottsförman
        \item LAN-party ansvarig
    \end{itemize}

    \subsubsection*{Informationsutskottet}
    \begin{itemize}
        \item DWWW-ansvarig
    \end{itemize}

    \subsubsection*{Sexmästeriet}
    \begin{itemize}
        \item Hovmästare
        \item Pubmästare
        \item Vice Pubmästare
        \item Barmästare
        \item Vice Barmästare
        \item Sångförman
        \item Köksmästare
        \item Preferensmästare
        \item Vice Köksmästare
    \end{itemize}

    \subsubsection*{Nollningsutskottet}
    \begin{itemize}
        \item Stabsmedlem
        \item Øverpeppare
    \end{itemize}

    \subsubsection*{Framtidsutskottet}
    \begin{itemize}
        \item Framtidsledamot
    \end{itemize}
\end{multicols}

\subsubsection*{Poster som bör väljas innan läsårets start}

\begin{multicols}{2}
    \subsubsection*{Näringslivsutskottet}
    \begin{itemize}
        \item Medlem i projektgruppen Teknikfokus
    \end{itemize}

    \subsubsection*{Aktivitetsutskottet}
    \begin{itemize}
        \item Karnevalsansvarig
    \end{itemize}

    \subsubsection*{Jubileet}
    \begin{itemize}
        \item Jubileumsansvarig
    \end{itemize}

\end{multicols}





\end{document}