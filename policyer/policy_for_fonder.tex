\documentclass{dsekprotokoll}
\usepackage[T1]{fontenc}
\usepackage[utf8]{inputenc}
\usepackage[swedish]{babel}
\usepackage{multicol}

\setheader{Policy för fonder}{Policydokument}{}

\title{Policy för fonder}
\author{Fred Nordell}

\begin{document}

\section*{Policy för Fonder}
\section{Formalia}
\subsection{Sammanfattning}
Policyn beskriver hur och varför man avsätter pengar till större projekt i fonder.
\subsection{Syfte}
Fonder avser i denna policy konton med medel avsatta till ett särskilt ändamål och ej sparande i aktier. Sektionen har behov av att avsätta pengar till större projekt samt hantera uppkomna kostnader som inte tagits hänsyn till i budgeten. Den här policyn är till för att vägleda sektionens sparande i fonder.
\subsection{Omfattning}
D-sektionen som helhet
\subsection{Ägande}
Sektionsmötet äger policyn
\subsection{Historik}
Policyn är antagen HTM2 2018.
Uppdaterad HTM1 2020.
Uppdaterad enl. Policy för policyer HTM2 2021.
Uppdaterad HTM1 2022.

\section{D-sektionen fonder}
Sektionen har ett antal fonder för att kunna täcka oförutsedda utgifter eller de utgifter som är svåra att budgetera för. 
För samtliga fonder finns det beskrivet huruvida fondens medel disponeras av styrelsebeslut eller beslut från sektionsmötet
samt hur mycket pengar fonden maximalt skall innehålla. Även syftet med fonden skall framgå av fondens beskrivning. 
Slutligen skall även den summa som bör avsättas årligen finnas med i beskrivningen av fonden. Sektionens årliga budget 
bör avsätta medel till de fonder som behöver enligt de riktlinjer som finns i fondens beskrivning.
Fonderna skall sedan fyllas på med resultatet från det föregående verksamhetsåret i samband med bokslutet. 
I händelse av att en fond redan är fylld då pengarna skall överföras bör de budgeterade pengarna som överskrider maxtaket 
för fonden läggas till resultatet. Samtliga uttag ur någon av fonderna skall redovisas för nästkommande sektionsmöte. 
D-sektionens fonder skall fyllas på i hierarkisk ordning i den ordning som de är uppräknade under paragraf §2.1.

\subsection{Sektionsfond}
Inventariefonden syftar till att kunna finansiera större projekt som ej hör till Källarmästeriets årliga verksamhet samt oförutsägbara eller större kostnader, exempelvis ersätta en trasig Soundboks eller inköp av ett större antal bord. Inventariefonden får uppgå till maximalt 40 000kr och dess medel disponeras av styrelsen. Den årliga budgeten bör avsätta 10 000kr till fonden årligen.

Bilfondens medel skall nyttjas till att kunna reparera skador på vår sektionsbil samt tillse att den är säker att köra. I händelse av att en ny sektionsbil behöver införskaffas kan pengarna ur fonden också nyttjas. Fonden får maximalt uppgå till 60 000kr och dess medel disponeras av styrelsen. Den årliga budgeten bör avsätta 10 000kr till fonden årligen.

Projektfonden syftar till att kunna avsätta medel för projekt inom D-sektionen. Dessa projekt skall pågå under en begränsad tid och vara till nytta för sektionen eller dess medlemmar. Fonden får uppgå till maximalt 20 000kr och dess medel disponeras av styrelsen. Den årliga budgeten bör avsätta 5 000kr till fonden årligen.

Medaljfonden syftar till att kunna köpa in de medaljer som sektionen, enligt vårt reglemente, skall delas ut till funktionärer, riddare eller liknande. Medaljfondens medel kan också nyttjas till framtagande av nya medaljer som sektionsmötet beslutat om. Fondens medel får maximalt uppgå till 10 000kr och de disponeras av medaljelelekommitén med godkännande av styrelsen. Den årliga budgeten bör avsätta 2 000kr till fonden årligen.

Motionsfonden syftar till att kunna genomföra de motioner som föreslås av medlemmarna till sektionsmötet och som sedan röstas igenom där. Fonden har inget maxtak på mängden pengar som får finnas i den och dess medel får endast nyttjas av sektionsmötet.

Jubileumsfonden syftar till att kunna finansiera sektionens jubileumsfirande vart femte år. Pengarna skall nyttjas för att säkerställa tillgängligheten för medlemmarna på jubileumsfirandet. Fondens medel disponeras av jubileumskommittén i samråd med styrelsen. Fondens medel får uppgå till maximalt 50 000kr. Den årliga budgeten bör avsätta 10 000kr till fonden årligen och bör inte heller överstiga detta om inte särskilda skäl finnes.

\section{Etiska hänsynstagningar och risker}

D-sektionen bör följa Teknologkårens policy för ekonomi när det kommer till etiska hänsynstagningar och riskaspekter kring sparande.




\end{document}