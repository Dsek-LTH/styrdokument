\documentclass[]{dsekprotokoll}

\usepackage[T1]{fontenc}
\usepackage[utf8]{inputenc}
\usepackage[swedish]{babel}
\usepackage{titlesec}

\setheader{Policy för Policyer}{Policydokument}{}

\title{Policy för Policyer}
\author{David Jobrant, Love Barany, Victor Winkelmann}

\begin{document}

\section{Formalia}
\subsection{Sammanfattning}
Denna policy beskriver hur policyer på D-sektionen fungerar och innehåller riktlinjer för hur policyer ska vara utformade. Policyn existerar även för att tydliggöra vem/vilket utskott som äger en policy.

\subsection{Syfte}
Syftet med denna policy är att tydliggöra vilka riktlinjer som ska följas och varför denna policy är viktig.

\subsection{Omfattning}
Riktlinjerna är gällande för D-sektionens policyer.

\subsection{Ägande}
Styrelsen äger policyn i sin helhet och har rätt att behandla och ändra denna på styrelsemöten.

\subsection{Historik}
Utkast färdigställt av David Jobrant, Informationsansvarig 2021, Victor Winkelmann, Näringslivsansvarig 2021 samt Love Barany, Cafémästare 2021. \\
Antaget VTM Extra 2021.

\section{Syftet med policyer}
Policyerna behandlar D-sektionens åsikter i enstaka frågor och hur D-sektionen ska agera i olika situationer. Policyer kan därmed ge vägledning till D-sektionen centralt, dess styrelse eller ett av sektionens utskott. En policy kan dessutom reglera dokument och andra handlingar.

Vilka som berörs av en policy skall stå i policyn under punkt \textit{1.2 Omfattning}. Aktuella policyer ska årligen samt vid ändringar tillställas berörda funktionärer.

\section{Uppbyggnad av policy}

\subsection{Namn}

En policy ska namnges enligt: \textit{``Policy för X''}.

\subsection{Struktur}

Samtliga av D-sektionens policyer ska inledas på samma sätt som denna med:


\hfill\begin{minipage}{\dimexpr\textwidth-3cm}
    \xdef\tpd{\the\prevdepth}
    \section*{1. Formalia}
    \subsection*{1.1 Sammanfattning}
    Här ska det finnas en kort sammanfattning av policyn för att läsaren snabbt ska få en översikt över vad policyn reglerar. \\

    \subsection*{1.2 Syfte}
    Här ska det finnas en kort beskrivning över varför denna policy är skriven och varför den behövs. \\

    \subsection*{1.3 Omfattning}
    Här ska det stå vad som omfattas av policyn. Exempel är: D-sektionens styrelse, Funktionärer vid D-sektionen och D-sektionens policyer.\\

    \subsection*{1.4 Ägande}
    Här ska information fastställas om vem/vilka som äger policyn och därmed har rätt att uppdatera dess innehåll. \\

    \subsection*{1.5 Historik}
    Här ska det finnas en kort historik över alla ändringar som skett till policyn samt vem som kan kontaktas för mer information om förändringen. Det ska stå vem som gjorde utkastet till det ursprungliga förslaget, enligt vilket beslut det ursprungliga förslaget blev fastställt, vem som gjort eventuella omarbetningar samt enligt vilka beslut de eventuella omarbetningarna har blivit fastställda.
    \\

    Historik samt korta beskrivningar av förändringar som skett bör presenteras i en lista i kronologisk ordning. \\
\end{minipage}

\prevdepth\tpd

\section{Ägande av policy}
Ägande innebär vem/vilka som äger policyn. Detta är viktigt om en policy behöver uppdateras eller ändras. Ägandet av policyn gör det möjligt att ändra den utan att behöva gå igenom ett sektionsmöte och kan vara viktigt om snabba ändringar behöver göras. Ägande av en policy medför även ansvaret att hålla policyn uppdaterad efter beslut.
\\

Ett exempel är policy för marknadsföring. Denna kan behöva uppdateras om styrelsen beslutar om något specifikt. Styrelsens ägande av den ovannämna policyn medför att styrelsen kan besluta om en förändring i policyn.

\section{Uppdatering av policy}
Vid uppdatering av en policy så ska en kort sammanfattning läggas till under historik tillsammans med vem eller vilka som gjort förändringen. Detta är för att kunna följa policyns utveckling.

\subsection{Uppdatering av denna policy}
Vid ändring av punkt \textit{3. Uppbyggnad} i denna policy så ska alla andra policyer uppdateras så att de följer den nya uppbyggnaden.

\section{Förteckning}

D-sektionen har följande policyer:

\begin{itemize}
    \item Alkoholinköp
    \item Dansplattepolicy
    \item Engelska titlar
    \item Informationsspridning
    \item Jämlikhetspolicy
    \item Lokalaccess
    \item Mötestider
    \item Policy för ekonomirutiner
    \item Policy för facebook
    \item Policy för fonder
    \item Policy för hantering av personsuppgifter
    \item Policy för marknadsföring
    \item Policy för röstning
    \item Policy för sektionsbil
    \item Policy för val
    \item Policy för valberedningens arbete
    \item Styrelsedatorer, snejk och mushroom
    \item Städregler för iDét
    \item Tackpolicy
    \item Utmärkelsen DuCtig
\end{itemize}
\end{document}