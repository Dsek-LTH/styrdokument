\documentclass{dsekprotokoll}

\usepackage[T1]{fontenc}
\usepackage[utf8]{inputenc}
\usepackage[swedish]{babel}

\setheader{Policy: Hantering av Personuppgifter}{Policydokument}{}

\title{Policy för Hantering av Personuppgifter}
\author{Anna Qvil, Fred Nordell}

\begin{document}

\section*{Policy för hantering av personuppgifter}
\section{Formalia}
\subsection{Sammanfattning}
Policy för hantering av personuppgifter beskriver hur sektionen skall hantera personuppgifter och vilken bakgrund som finns i olika situationer.
\subsection{Syfte}

Denna policy har tagits fram för att säkerställa att D-sektionen följer GDPR samt att underlätta för funktionärer att hantera personuppgifter på ett korrekt sätt.
\subsection{Omfattning}
Sektionen i sin helhet.
\subsection{Historik}
Policyn är antagen på styrelsemöte S24 2018.
Uppdaterad enl. Policy för Policyer på HTM2 2021. Uppdaterad enl. Policy för styrdokument på VTM-extra 2023. Policyn updaterades VTM-extra 2023 av Rafael Holgersson.

\section{Insamling av personuppgifter}

\subsection{Insamling}
D-sektionens utskott kan inför evenemang eller i andra sektionsrelaterade fall behöva samla in och förvara personuppgifter internt för att evenemanget ska kunna genomföras på ett smidigt sätt. Exempel på tillfällen kan vara:
\begin{itemize}
    \item Sittning eller annan fest
    \item Lunchföreläsning, Casekväll eller liknande
    \item Idrottsevenemang
\end{itemize}

Syftet med att samla in personuppgifter inför ett evenemang, vem som har tillgång till dessa samt under vilken tidsperiod de lagras ska tydliggöras i Sektionens registerförteckning. Samma information skall också tydligt framgå för personen som skall lämna sina uppgifter.

Exempel på personuppgifter som kan sparas tillfälligt:
\begin{itemize}
    \item För- och efternamn
    \item Årskurs
    \item Matpreferens
    \item E-postadress
    \item Telefonnummer
\end{itemize}

Insamlingen av personuppgifter kan ske genom utskick av formulär eller i samband med betalning/anmälningslänk.

\subsection{Samtycke}

I samband med att personuppgifterna hämtas skall godkännande från personen i fråga också hämtas. Samtycke måste lämnas aktivt och tydligt samt skall sparas tills dess att uppgifterna raderas. Samtycke kan ges muntligt om detta dokumenteras på tillfredsställande vis. Om personen inte godkänner att dennes personuppgifter sparas ska Sektionen ej spara dessa. Det är funktionären som samlade in uppgifterna som
ansvarar för att vid samma tillfälle också samla in samtycke. Det ska vara tydligt framgå vad som sparas och till vilket syfte.

\section{Lagring av personuppgifter}
Personuppgifter som tillfälligt samlas in lagras på konton kopplade till Sektionens Google-konton. Uppgifterna lagras tills dess att evenemanget är genomfört och det inte längre finns något syfte att spara dem.

\subsection{Tillgänglighet}

Vid behov kan andra funktionärer ges tillgång
till de insamlade personuppgifterna i ställföreträdande syfte eller som extra bemanning. Den ansvariga funktionären ska kunna
redogör vem som har haft tillgång till systemet, i vilket syfte och under vilken tidsperiod.

\section{Hantering}
De insamlade uppgifterna hanteras av funktionären som är ansvarig över evenemanget och som samlade
in uppgifterna. Ordförande för berört utskott är ansvarig för att funktionären hanterar uppgifterna enligt policyn. Personuppgifter från det tillfälliga registret kan endast begäras ut av personen i fråga under förutsättningen att den kan identifiera sig med godkänd ID-handling. I övrigt kan endast personuppgifter lämnas ut till andra parter på skriftligt beslut från Styrelsen. Styrelsen är skyldig att informera personen ifråga innan personuppgifterna lämnas ut.

\subsection{Spridning}
Funktionären som samlar in personuppgifterna är ansvarig för att uppgifterna inte sprids eller missbrukas på något sätt. Vid dataintrång eller annat missbruk skall personen vars uppgifter utnyttjats underrättas om situationen direkt. Styrelsen är ansvarig för att berörda parter underrättas vid dataintrång/spridning/missbruk.

\subsection{Radering}
När man samlar in godkännande för lagringen av personuppgifter ska också ett datum anges
där personuppgifterna kommer raderas. Detta datumet ska vara efter potentiellt evenemang
eller då de inte längre kommer användas. Exempel kan vara efter ett val, eller i slutet av året
när ens funktionärer har fått sitt tack. Om radering ej är möjlig ska
uppgifterna anonymiseras. Funktionären som samlat in personuppgifterna är ansvarig för
radering/anonymisering.

\section{Ansvar}
D-sektionen och dess företrädare (d.v.s styrelsen) är ytterst ansvarig för att insamling, lagring samt hantering av personuppgifter sker på korrekt sätt enligt lag och sektionens antagna bestämmelser. Respektive utskottsordförande är ansvarig för att medlemmarna inom utskottet
sköter insamling, lagring och hantering på rätt sätt.

\end{document}
