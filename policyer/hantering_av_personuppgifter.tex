\documentclass{dsekkallelse}

\usepackage[T1]{fontenc}
\usepackage[utf8]{inputenc}
\usepackage[swedish]{babel}

\setheader{Policy: Hantering av Personuppgifter}{Policydokument}{}

\title{Policy för Hantering av Personuppgifter}
\author{Anna Qvil, Fred Nordell}

\begin{document}

\section{Policy för Hantering av Personuppgifter}

\subsection{1. Bakgrund och Syfte}
EU:s nya dataskyddsförordning General Data Protection Regulation, GDPR, som träder i kraft den 25:e maj 2018 innebär bland annat hårdare krav på hantering av personuppgifter. Det kommer att ställas krav på nya rutiner och processer för säker hantering av register samt krav på ansvarig ledningsnivå. Nya dataskyddsförordningen kommer att gälla för alla organisationer och branscher som sparar eller på något sätt hanterar personlig och känslig information om sina anställda, kunder eller medlemmar.

\par Det ligger mycket fokus på ett ökat integritetsskydd och att stärka kraven på att företag och organisationer ska informera om vilka personuppgifter de hanterar samt hur och varför detta görs. Det ska också gå att under vissa omständigheter gå att säga nej till att personuppgifterna används. I det ökade skyddet för privatpersoner ingår också rätten att be företag och organisationer att ta bort personlig data på begäran.

Denna policy har tagits fram för att säkerställa att D-sektionen följer den nya lagstiftningen samt att underlätta för funktionärer att hantera personuppgifter på ett korrekt sätt.

\subsection{Historik}
Policyn är antagen på styrelsemöte S24 2018. 

\subsection{2. Insamling av personuppgifter}

\subsubsection{2.1 Insamling}
D-sektionens utskott kan inför evenemang eller i andra sektionsrelaterade fall behöva samla in och förvara personuppgifter internt för att evenemanget ska kunna genomföras på ett smidigt sätt. Exempel på tillfällen kan vara:
\begin{itemize}
\item Sittning eller annan fest 
\item Lunchföreläsning, Casekväll eller liknande
\item Idrottsevenemang 
\end{itemize}

Syftet med att samla in personuppgifter inför ett evenemang, vem som har tillgång till dessa samt under vilken tidsperiod de lagras ska tydliggöras i Sektionens registerförteckning. Samma information skall också tydligt framgå för personen som skall lämna sina uppgifter.

Exempel på personuppgifter som kan sparas tillfälligt: 
\begin{itemize}
\item För- och efternamn
\item Årskurs 
\item Matpreferens 
\item E-postadress 
\item Telefonnummer
\end{itemize}

Insamlingen av personuppgifter kan ske genom utskick av formulär eller i samband med betalning/anmälningslänk. 

\subsubsection{2.2 Samtycke}

I samband med att personuppgifterna hämtas skall godkännande från personen i fråga också hämtas. Samtycke måste lämnas aktivt och tydligt samt skall sparas tills dess att uppgifterna raderas. Samtycke kan ges muntligt om detta dokumenteras på tillfredsställande vis. Om personen inte godkänner att dennes personuppgifter sparas ska Sektionen ej spara dessa. Funktionären som samlar in uppgifterna har som uppgift att också begära in samtycke i samband med att denne samlar in personuppgifter. Det ska vara tydligt framgå vad som sparas och till vilket syfte.

\subsection{3. Lagring av personuppgifter}
Personuppgifter som tillfälligt samlas in lagras på konton kopplade till Sektionens Google-konton. Uppgifterna lagras tills dess att evenemanget är genomfört och det inte längre finns något syfte att spara dem.

\subsubsection{3.1 Tillgänglighet}

Vid behov kan andra förtroendevalda poster ges tillgång till de insamlade personuppgifterna i ställföreträdande syfte eller som extra bemanning. Vid dessa tillfällen skall den ansvariga funktionären dokumentera vem som haft tillgång till systemet, i vilket syfte och under vilken tidsperiod

\subsection{4. Hantering}
De insamlade uppgifterna hanteras av funktionären som är ansvarig över evenemanget och som samlade
in uppgifterna. Ordförande för berört utskott är ansvarig för att funktionären hanterar uppgifterna enligt riktlinjen. Personuppgifter från det tillfälliga registret kan endast begäras ut av personen i fråga under förutsättningen att den kan identifiera sig med godkänd ID-handling. I övrigt kan endast personuppgifter lämnas ut till andra parter på skriftligt beslut från Styrelsen. Styrelsen är skyldig att informera personen ifråga innan personuppgifterna lämnas ut.

\subsubsection{4.1 Spridning}
Funktionären som samlar in personuppgifterna är ansvarig för att uppgifterna inte sprids eller missbrukas på något sätt. Vid dataintrång eller annat missbruk skall personen vars uppgifter utnyttjats underrättas om situationen direkt. Styrelsen är ansvarig för att berörda parter underrättas vid dataintrång/spridning/missbruk.

\subsection{4.2 Radering}
Efter genomfört evenemang skall personuppgifterna som samlats in raderas. Om radering ej är möjlig ska
uppgifterna anonymiseras. Funktionären som samlat in personuppgifterna är ansvarig för
radering/anonymisering.

\subsection{5. Ansvar}
D-sektionen och dess företrädare (d.v.s styrelsen) är ytterst ansvarig för att insamling, lagring samt hantering av personuppgifter sker på korrekt sätt enligt lag och sektionens antagna bestämmelser. Respektive utskottsordförande är ansvarig för att medlemmarna inom utskottet
sköter insamling, lagring och hantering på rätt sätt.

%\vfill
%\signature{För Styrelsen}{Anna Qvil}{Ordförande, Datatekniksektionen}
%\signature{För Styrelsen}{Fred Nordell}{Skattmästare, Datatekniksektionen}

\end{document}