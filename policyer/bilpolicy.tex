\documentclass{dsekprotokoll}

\usepackage[T1]{fontenc}
\usepackage[utf8]{inputenc}
\usepackage[swedish]{babel}

\newcommand{\datum}{2020--09--03}

\setheader{Policy för sektionsbil}{Policydokument}{2020-09-03, Lund}

\author{Albin Sverreson}

\title{Policy för sektionsbil}
\begin{document}

\maketitle
\section{Formalia}
\subsection{Sammanfattning}
Policyn beskriver hur och vem som får använda sektionsbilen som antingen är inhyrd eller ägd av sektionen.
\subsection{Syfte}
Ibland händer det att D-sektionen antingen äger eller hyr en bil under längre tid. Denna policy ämnar att reglera vem som får använda bilen samt hur och i vilket syfte den ska användas.
\subsection{Omfattning}
Sektionen i sin helhet.
\subsection{Ägande}
Styrelsen äger policyn.
\subsection{Historik}
Policyn är antagen på styrelsemöte S18 2020.
Uppdaterad enl. Policy för policyer på HTM2 2021


\section{Ändamål}
Sektionsbilen ska endast användas av sektionens medlemmar inom Sverige i sektionsfrämjande syfte. Detta kan vara t.ex frakt eller inköp av materiel till sektionen och dess evenemang samt kickoffer, tackfester eller dylikt.
\section{Bokning}
Ett bokningssystem bör tillhandahållas av Källarmästaren. Styrelsen är de enda som bör kunna boka bilen och funktionärer ska gå till sina respektive utskottsledare för att boka. Utskottsledaren kan då godkänna eller neka bokningen beroende på om denne anser att ändamålet är del av utskottets verksamhet. Styrelsen bör komma överens om ett maximalt antal timmar man får boka i sträck för att minimera risken att bilen bokas en längre tid än den behövs. I nödfall kan styrelsen besluta att upphäva en bokning.
\section{Nycklar}
Nycklarna till bilen bör förvaras på en säker plats som bara styrelsen och eventuellt vissa kärnposter kommer åt. I de fall bilen bokats av en funktionär lämnas nyckeln ut av någon styrelsemedlem.

\section{Ansvar}
Du som bokat är ansvarig och är den enda som får köra. Du förblir ansvarig tills nyckeln är tillbakalämnad. Innan körning ser du över bilens skick och rapporterar skador till Källarmästaren, annars riskerar du att få ansvaret för skadorna. Övriga kostnader såsom parkering eller böter betalas av föraren. Om en skada uppkommer eller en olycka sker under körningen ska du omedelbart rapportera detta till Källarmästaren, oavsett storlek. Källarmästaren kan alltid kontaktas på kallarm@dsek.se. Kostnaden för sagda olycka betalas av sektionen givet att föraren inte varit oaktsam.
\section{Kostnad och budget}
Styrelsen får i budgeten bestämma hur transportkostnader delas upp. Om varje utskott har en egen budgetpost för transport bör Källarmästaren tillhandahålla någon form av körjournal som utskotten kan fylla i för att kunna beräkna transportkostnaderna för de olika utskotten. I det fall där alla utskotten delar på en under sektionen ordnad budgetpost för transport så bör styrelsen göra en intern uppdelning över hur mycket de olika utskotten bör spendera. Varje utskottsledare bör då beakta denna interna uppdelning och hålla koll på ungefär hur mycket ens bokningar kostar.\newline



\end{document}
%dokumentklass??
%beslut som grundar sig på möte. 