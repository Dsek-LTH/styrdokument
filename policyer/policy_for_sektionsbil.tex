\documentclass{dsekprotokoll}

\usepackage[T1]{fontenc}
\usepackage[utf8]{inputenc}
\usepackage[swedish]{babel}

\newcommand{\datum}{2020--09--03}

\setheader{Policy för sektionsbil}{Policydokument}{Lund -- 3 September 2020}

\author{Albin Sverreson}

\title{Policy för sektionsbil}
\begin{document}

\maketitle
\section{Formalia}
\subsection{Sammanfattning}
Policyn beskriver hur och vem som får använda sektionsbilen som antingen är inhyrd eller ägd av sektionen.
\subsection{Syfte}
Ibland händer det att D-sektionen antingen äger eller hyr en bil under längre tid. Denna policy ämnar att reglera vem som får använda bilen samt hur och i vilket syfte den ska användas.
\subsection{Omfattning}
Sektionen i sin helhet.
\subsection{Historik}
Policyn är antagen på styrelsemöte S18 2020.
Uppdaterad enl. Policy för policyer på HTM2 2021 av Albin Sverresson.
Uppdaterad enl. Policy för styrdokument på VTM-extra 2023.
Uppdaterad enligt beslut på VTM-Extra 2023 av Julia Karlsson.


\section{Ändamål}
Sektionsbilen ska endast användas av sektionens medlemmar inom Sverige i sektionsfrämjande syfte. Detta kan vara t.ex frakt eller inköp av materiel till sektionen och dess evenemang samt kickoffer, tackfester eller dylikt.
\section{Bokning}
Ett bokningssystem bör tillhandahållas av Källarmästaren. I nödfall kan styrelsen besluta att upphäva en bokning. Hur bokning går till och regler att förhålla sig till finns i Riktlinje för sektionsbilens användning. Bilen får inte användas utan Källarmästarens tillåtelse.

\section{Ansvar}
Du som bokat är ansvarig och är den enda som får köra. Undantag för detta gäller då bokningen görs för någon annans räkning, till exempel då en utskottsordförande bokar åt sina funktionärer. I detta fall ska det vara tydligt vem som ska använda bilen i bokningen. Föraren förblir ansvarig tills nyckeln är tillbakalämnad. Innan körning ska bilens skick ses över och skador rapporteras till Källarmästaren eller
Bilansvariga, annars riskerar föraren att få ansvaret för skadorna. Privat utlägg kan göras för parkering under godkända
bokningar, men övriga kostnader så som böter betalas av föraren.

Om en skada uppkommer eller en olycka sker under körningen ska du omedelbart rapportera detta till Källarmästaren eller Bilansvariga, oavsett storlek. Källarmästaren kan alltid kontaktas på kallarm@dsek.se. Kostnaden för sagda olycka betalas av sektionen givet att föraren inte varit oaktsam.
\section{Kostnad och budget}
Styrelsen får i budgeten bestämma hur transportkostnader delas upp. Om varje utskott har en egen budgetpost för transport bör Källarmästaren tillhandahålla någon form av körjournal som utskotten kan fylla i för att kunna beräkna transportkostnaderna för de olika utskotten. I det fall där alla utskotten delar på en under sektionen ordnad budgetpost för transport så bör styrelsen göra en intern uppdelning över hur mycket de olika utskotten bör spendera. Varje utskottsledare bör då beakta denna interna uppdelning och hålla koll på ungefär hur mycket ens bokningar kostar.\newline



\end{document}
%dokumentklass??
%beslut som grundar sig på möte. 