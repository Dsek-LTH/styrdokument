\documentclass{dsekkallelse}

\usepackage[T1]{fontenc}
\usepackage[utf8]{inputenc}
\usepackage[swedish]{babel}
\usepackage{multicol}

\setheader{Policy för Facebook}{Policydokument}{}

\title{Policy för hantering av sektiones facebookssidor}
\author{}

\begin{document}

\section{Policy för hantering av sektiones facebookssidor}

\subsection{Bakgrund}

Sektionen har flera sidor på facebook, varav en är sektionens huvudsidan och övriga är utskottssidor. För att det ska bli enkelt att hantera eventanmälan och för att undvika intressekonflikter behövs en del riktlinjer för hur sektionens olika facebooksidor hanteras.

\subsubsection{Historik}
Policyn är antagen på styrelsemöte S03 2014.


\section{Policy:} 
\begin{attlista}
	\item sektionen har en huvudsida på Facebook, där Ordförande och Propagandamästare är
administratörer.
\item på sektionens sida ska information som berör hela sektionen läggas upp.
\item resterande styrelsemedlemmar har tillåtelse att lägga upp information och liknande genom att de är så kallade Content Creators.
\item vid behov kan andra ges Content Creator rättigheter.
\item  sektionens utskott får lov att ha egna sidor på Facebook som respektive mästare ansvarar
för.
\item event som skapas antingen på huvudsidan eller på någon av utskottens sidor och som
når alla medlemmar, ska länka till en befintlig eventanälan på desk.se. Det är endast anmälan på dsek.se som är bindande.
\item på dsek.se länkas endast sektionens huvudsida på Facebook.
\item sektionens huvudsida ska hänvisa till utskottssidorna.
\item utskottens sidor ska hänvisa till sektionens huvudsida.

\end{attlista}

Eventuella förtydligande:

Med hänvisa menas att det ska stå på utskottens sidor att sektionen har en huvudsida aom man
kan besöka. Huvudsidan ska visa besökare att det finns utskottssidor man kan besöka. Dessa
innehar som har mer riktad information som rör ett specifikt utskott.

Med Huvudsidan menas sektionens sida på Facebook och inte dsek.se. Detsamma gäller utskottens sidor.
	

\end{document}