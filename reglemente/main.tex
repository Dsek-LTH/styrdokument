\documentclass[pdfbookmarks,a4paper,11pt]{article}
\usepackage{dsekcommon}
\usepackage[utf8]{inputenc}
\usepackage{enumerate}
\usepackage{lastpage}
\usepackage{longtable}
\ifpdf
  \usepackage{thumbpdf}
\fi
\usepackage{calc}

\pagestyle{fancy}

\lhead{}
\chead{}
\rhead{}
\lfoot{}
\cfoot{\thepage\ (\nohyperpageref{LastPage})}
\rfoot{}

% Puts a full stop after the section number
\makeatletter
\def\@seccntformat#1{\csname the#1\endcsname.\quad}
\makeatother

\newcommand{\funktionar}[1]{%
  \subsection*{#1}\par
  \addcontentsline{toc}{subsection}{#1}}

\newlength{\itemcollength}
\setlength{\itemcollength}{40mm}
\newenvironment{reglemlista}{%
  \begin{list}{}{%
      \setlength{\labelwidth}{\itemcollength}%
      \setlength{\leftmargin}{\labelwidth + \labelsep}%
      \renewcommand{\makelabel}[1]{%
        \raisebox{0pt}[1ex][0pt]{%
          \makebox[\labelwidth][l]{%
            \parbox[t]{\itemcollength}{%
              \raggedright\hspace{0pt}##1}}}\hfill}%
      }}{%
  \end{list}}

%%%%%%%%%%%%% HEADER %%%%%%%%%%%%%

\title{Reglemente för D-sektionen inom TLTH\\ Organisationsnummer: 845003-2878}
\date{\today}

\setheader{REGLEMENTE}{HTM1 2019}{\today}

%%%%%%%%%%%%% HEADER %%%%%%%%%%%%%
\begin{document}

\maketitle

%%%%%%%%%%%%% SECTION %%%%%%%%%%%%%
\section{Teknologia}

\begin{reglemlista}

	\item[Råsa]
	En lämplig tolkning av råsa är 0xF280A1. Detta kan tolkas som Pantone
	189~U eller 35,3\,\% svart.

	\item[Rosa Pantern]
	Med Rosa Pantern åsyftas den figur som skapades av Friz Freleng och David
	DePatie till Blake Edwards film ''The Pink Panther'' år 1963.

	\item[Råsa Digitalis]
	Med Råsa Digitalis menas en råsa fingerborgsblomma, på latin Digitalis Purpurea.

\end{reglemlista}

%%%%%%%%%%%%% SECTION %%%%%%%%%%%%%
\section{Hedersomnämnande}

\begin{reglemlista}

	\item[Hedersmedlemmar]
	Hedersmedlemmar är:
	\begin{itemize}
		\item Nina Reistad\\
		      \emph{Inlagd i reglementet på VTM1 2000. Inröstad 1993}
		\item Per-Henrik Rasmussen \\ \emph{inröstad 1994}
		\item Rune Kullberg\\
		      \emph{Invald på HTM1 2002 för sitt arbete för studentinflytande, hans handlingskraft i studentfrågor, och hans stora initiativförmåga som ordförande i Utbildningsnämnden för D}
		\item Nora Ekdahl\\
		      \emph{Invald VTM 2010 för sina insatser som studievägledare för C- och D-programmet.}
		\item Mats Cedervall \\
		      \emph{Invald på HTM1 2018 för sitt stora Intresse i sektionens lokaler och för att han som Husprefekt alltid har verkat för studenternas ökade inflytande i husstyrelsen}
	\end{itemize}

	\item[Grundarna av D-sektionen inom TLTH]
	Grundarna av D-sektionen inom TLTH är:

	\begin{itemize}
		\item Lars Svensson
		\item Ulf Bengtsson
		\item Henrik Cronström
		\item Kristin Andersson
		\item Stefan Molund
		\item Jan Eric Larsson
	\end{itemize}

\end{reglemlista}

%%%%%%%%%%%%% SECTION %%%%%%%%%%%%%
\section{Sektionsmöte}

\begin{reglemlista}

	\item[Utlysande]
	Utöver anslag på anslagstavlan skall följande tillställas kallelse:
	\begin{itemize}
		\item Valberedningen
		\item Revisorerna
		\item Kårordföranden
		\item Inspektorn
	\end{itemize}
	
	\item[Förtydligande]
	Efter varje post står en siffra inom parentes, det är antal personer som kan sitta på en post, om det inte står någon siffra så är det erforderligt antal.

	\item[Vårterminsmöte]
	Under vårterminsmötet skall följande poster tillsättas:
	\begin{itemize}

		\item Teknikfokusansvarig (1)
		\item Jubileumsansvarig (1) (väljes året innan jubileet äger rum)
		\item Valberedningens ordförande (1)
		\item Valberedningsrepresentanter från de fyra lägsta årskurserna
		\item Medaljelelekommittémedlem
	\end{itemize}

	\item[Höstterminsmöte ett]
	Under höstterminsmöte ett skall följande poster tillsättas:
	\begin{itemize}
		\item Valberedningsrepresentant från den lägsta årskursen
		\item Øverphøs (1)
	\end{itemize}

	\item[Valmöte]
	Under valmötet skall följande poster tillsättas:
	\begin{itemize}
		\item Ordförande (1)
		\item Vice Ordförande (1)
		\item Informationsansvarig (1)
		\item Skattmästare (1)
		\item Cafémästare (1)
		\item Sexmästare (1)
		\item Studierådsordförande (1)
		\item Näringslivsansvarig (1)
		\item Källarmästare (1)
		\item Aktivitetsansvarig (1)
		\item Vice Informationsansvarig (1)
		\item Vice Skattmästare (1-4)
		\item Vice Cafémästare (1)
		\item Vice Sexmästare (1)
		\item Vice Studierådsordförande (1)
		\item Vice Näringslivsansvarig (1)
		\item Vice Källarmästare (1)
		\item Vice Aktivitetsansvarig (1)
		\item Revisor (2)
		\item Inspektor (1)
		\item Framtidsordförande (1)
		\item Trivselmästare (1)
		\item Valnämndsrepresentant TLTH (1)
		\item Alumniansvarig (1)
		\item Utedischoansvarig (1-2)
		\item Övermarskalk (1)
		\item Talman (1)
	\end{itemize}

\end{reglemlista}

%%%%%%%%%%%%% SECTION %%%%%%%%%%%%%
\section{Valberedningen}

\begin{reglemlista}

	\item[Åligganden]
	Det åligger valberedningen
	\begin{attlista}
		\item intervjua kandidater till styrelseposter
		\item intervjua kandidater till Framtidsutskottet
		\item kontakta samtliga nominerade
	\end{attlista}

\end{reglemlista}

%%%%%%%%%%%%% SECTION %%%%%%%%%%%%%
\section{Styrelsen}

\begin{reglemlista}

	\item[Åligganden]
	Det åligger styrelsen
	\begin{attlista}
		\item inför sektionen ansvara för sektionens verksamhet
		\item förbereda och genomföra sektionsmöten
		\item verkställa och övervaka genomförandet av sektionsmötets beslut
		\item ansvara för sektionens medel
		\item bereda inkomna förslag
		\item handha sektionens korrespondens
		\item utarbeta förslag till budget och resultatdispositioner
		\item utarbeta förslag till verksamhetsplan
		\item senast sex veckor efter verksamhetsårets slut till revisor\-erna överlämna verksamhetsberättelse, protokoll och övriga handlingar revisorerna önskar ta del av
		\item en gång per termin informera medaljelekomittén om de funktionärer som styreslen anser skall mottaga medalj
		\item uträtta val som inte uträttats av sektionsmötet
	\end{attlista}

	\item[Sammanträde]
	Sektionsstyrelsen sammanträder på kallelse av ordförande eller vid dennes frånvaro av vice ordförande. Ständigt adjungerad är, förutom de av stadgan och reglementet fastställda, även E-sektionens ordförande, två styrelseledarmöter från D-chip och sektionens kårkontakter.

\end{reglemlista}

%%%%%%%%%%%%% SECTION %%%%%%%%%%%%%
\section{Utskott}

\begin{reglemlista}

	\item[Åligganden]
	Det åligger utskotten
	\begin{attlista}
		\item söka erforderliga tillstånd
		\item följa av sektionen uppsatta riktlinjer
	\end{attlista}

	\item[Utskottsordföranden]
	Det åligger utskottsordföranden
	\begin{attlista}
		\item ansvara för utskottets verksamhet
		\item senast fyra veckor efter verksamhetsårets slut till styrelsen inlämna verksamhetsberättelse
		\item leda och fördela arbetet inom utskottet
		\item budgetera, redovisa och följa upp utskottets verksamhet
		\item upprätta bokslut och halvårsbokslut
		\item sammankalla utskottet minst två gånger per verksamhetsår
		\item hålla styrelsen informerad om utskottets verksamhet
		\item utse erforderligt antal funktionärer utöver de av sektionen valda
	\end{attlista}

	\item[Titulering]
	Alternativ titulering för utskottsordförande är mästare.

\end{reglemlista}

%%%%%%%%%%%%% SECTION %%%%%%%%%%%%%
\section{Cafémästeriet}

\begin{reglemlista}

	\item[Åligganden]
	Det åligger cafémästeriet
	\begin{attlista}
		\item ansvara för caféts verksamhet
		\item verka för att ett så brett cafésortiment som möjligt finns tillgängligt för sektionens medlemmar
		\item ansvara för daglig städning av iDét
		\item då det är praktiskt lämpligt, assistera D-shopen med försäljning av märken
	\end{attlista}

	\item[Sammansättning]
	Cafémästeriet består av
	\begin{itemize}
		\item Cafémästare
		\item Vice Cafémästare
		\item Dagsansvarig
		\item Sektionslivskvalitetsförhöjare
		\item Stekare
		\item Funktionärer
	\end{itemize}

\end{reglemlista}

%%%%%%%%%%%%% SECTION %%%%%%%%%%%%%
\section{Näringslivsutskottet}

\begin{reglemlista}

	\item[Åligganden]
	Det åligger näringslivsutskottet
	\begin{attlista}
		\item verka för att sektionens goda relationer med näringslivet behålls och förbättras
		\item verka för att information om datateknik- och infocomutbildningarna vid LTH sprids till företag och näringsliv
		\item underlätta kontakten mellan sektionsmedlemmar och verkligheten
		\item vid behov söka sponsorer till sektionen
		\item genom kontakt med gamla sektionsmedlemmar kunna ge en inblick i verksamheten i intressanta företag
		\item bistå Nollningsutskottet i sponsorsökandet inför nollningen
	\end{attlista}

	\item[Sammansättning]
	Näringslivsutskottet består av
	\begin{itemize}
		\item Näringslivsansvarig
		\item Vice Näringlivsansvarig
		\item Mentorsansvarig
		\item Alumnigruppen
		\item Projektgruppen för Teknikfokus
		\item Funktionärer
	\end{itemize}

	\item[\textbf{Alumnigruppen}]

	\item[Åligganden]
	Det åligger alumnigruppen
	\begin{attlista}
		\item  upprätthålla en förteckning över utexaminerade D- och C-teknologer
		\item upprätthålla kontakten med utexaminerade D- och C-teknologer
		\item underhålla examensmonumentet
	\end{attlista}

	\item[Sammansättning]
	Alumnigruppen består av
	\begin{itemize}
		\item Alumniansvarig
		\item Alumnigruppsmedlem (erfoderligt antal)
		\item Alumnimiddagsansvarig
	\end{itemize}

\end{reglemlista}
\textbf{Projektgruppen för Teknikfokus}
\begin{reglemlista}
	\item[Åligganden]
	Det åligger Projektgruppen för Teknikfokus

	\begin{attlista}
		\item på våren anordna arbetsmarknadsmässan Teknikfokus
	\end{attlista}

	\item[Sammansättning]
	Projektgruppen för Teknikfokus består av
	\begin{itemize}
		\item Teknikfokusansvarig
		\item Medlem i Projektgruppen för Teknikfokus (erfoderligt antal)
	\end{itemize}

\end{reglemlista}


%%%%%%%%%%%%% SECTION %%%%%%%%%%%%%
\section{Källarmästeriet}

\begin{reglemlista}

	\item[Åligganden]
	Det åligger källarmästeriet
	\begin{attlista}
		\item underhålla sektionens lokaler, datorer och inventarier
		\item handha bokningar av iDét
		\item hålla minst en Snickerboa per läsperiod
		\item underhålla sektionens spelmaskiner
	\end{attlista}

	%%% Togs bort vtm 2018 "resning reglemente" snickerboa bör dock förklaras
	% \item[Snickerboa]
	%   Under snickerboa ges sektionsmedlemmarna möjlighet att försköna och
	%   underhålla sektionens lokaler och inventarier. Till snickerboa ränner
	%   alla sektionsmedlemmar.

	\item[Sammansättning]
	Källarmästeriet består av
	\begin{itemize}
		\item Källarmästaren
		\item Vice Källarmästare
		\item Ljud- och ljusansvarig
		\item rootmästeriet
		\item Trädgårdsmästare
		\item Sektionslivskvalitetsförhöjare
		\item Funktionärer
	\end{itemize}

	\item[\textbf{Rootmästeriet}]

	\item[Åligganden]
	Det åligger Rootmästeriet
	\begin{attlista}
		\item underhålla och förbättra sektionens datorer, dess tillhörande utrustning, programvara och funktioner
		\item under ledning av Sparkyansvarig underhålla och förbättra Sparky
	\end{attlista}

	\item[Sammansättning]
	Rootmästeriet består av
	\begin{itemize}
		\item root
		\item sudo
		\item Sparkyansvarig
	\end{itemize}

\end{reglemlista}

%%%%%%%%%%%%% SECTION %%%%%%%%%%%%%
\section{Medaljelelekommittén}

Medaljelelekommittén delar ut medaljer till sektionsfunktionärer. Medaljelelekommittén leds av Övermarskalken.

\begin{reglemlista}

	\item[Åligganden]
	Det åligger medaljelelekommittén
	\begin{attlista}
		\item vid högtidliga tillfällen, minst två gånger per år, tillsammans med (på skiftesgasquen avgående) sektionsordförande dela ut medaljer och andra utmärkelser till funktionärer som gjort sig förtjänta av dylikt
		\item underhålla en lista över personer och utmärkelser på sektionens hemsida
		\item minst en gång per termin bjuda in sektionsordförande till möte, så att denna kan informera om styrelsens åsikter om funktionärers prestationer
	\end{attlista}

	\item[Sammansättning]
	Medaljelelekommittén består av
	\begin{itemize}
		\item Övermarskalken
		\item Inspektor
		\item Medaljelelekommittémedlemmar
	\end{itemize}
\end{reglemlista}

%%%%%%%%%%%%% SECTION %%%%%%%%%%%%%
\section{Aktivitetsutskottet}

\begin{reglemlista}

	\item[Åligganden]
	Det åligger Aktivitetsutskottet
	\begin{attlista}
		\item anordna för sektionens medlemmar trevliga evenemang
		\item deltaga i sångarstriden
		\item verka för att sektionen har lag med i av andra studentorganisationer anordnade aktiviteter, såsom Tandemstafetten och DÖMD
		\item informera om större arrangemang på andra orter
	\end{attlista}

	\item[Sammansättning]
	Aktivitetsutskottet består av
	\begin{itemize}
		\item Aktivitetsansvarig
		\item Vice Aktivitetsansvarig
		\item Utedischoansvarig
		\item Idrottsförman
		\item Karnevalsansvarig
		\item LANparty-ansvarig
		\item Sångarstridsförmän
		\item Tandemgeneral
		\item Semesterfirare
		\item D-shopen
		\item Funktionärer
	\end{itemize}


	\item[\textbf{D-shopen}]

	\item[Åligganden]
	Det åligger D-shopen
	\begin{attlista}
		\item handha inköp och försäljning av olika sektionssymboler, exempelvis märken och overaller
		\item förnya sortimentet då så krävs
	\end{attlista}
	\item[Sammansättning]
	D-shopen består av
	\begin{itemize}
		\item Märkvärdig
	\end{itemize}
\end{reglemlista}

%%%%%%%%%%%%% SECTION %%%%%%%%%%%%%
\section{Informationsutskottet}

\begin{reglemlista}

	\item[Åligganden]
	Det åligger Informationsutskottet
	\begin{attlista}
		\item utge sektionens medier
		\item ansvara för sektionens anslagstavlor och websidor
		\item sända kopior av sektionens publikationer till lämpliga instanser, t~ex övriga sektioner
		\item upprätthålla sektionens arkiv
		\item ansvara för sektionens WWW-sidor och att informationen på dessa hålls uppdaterad
	\end{attlista}

	\item[Sammansättning]
	Informationsutskottet består av
	\begin{itemize}
		\item Informationsansvarig
		\item Vice Informationsansvarig
		\item Arkivarie
		\item Artist
		\item DWWW
		\item Fotograf
		\item Kårrepresentant
		\item Funktionärer
	\end{itemize}

	\item[\textbf{DWWW}]

	\item[Åligganden]
	Det åligger DWWW
	\begin{attlista}
		\item underhålla och utveckla sektionens WWW-sidor
	\end{attlista}

	\item[Sammansättning]
	DWWW består av
	\begin{itemize}
		\item DWWW-ansvarig
		\item DWWW-medlem
	\end{itemize}

\end{reglemlista}

%%%%%%%%%%%%% SECTION %%%%%%%%%%%%%
\section{Sexmästeriet}

\begin{reglemlista}

	\item[Åligganden]
	Det åligger sexmästeriet
	\begin{attlista}
		\item anordna de traditionella sektionsfesterna
		\item anordna festverksamhet under nollningen
		\item anordna pubar
	\end{attlista}

	\item[Pubmästeriet]
	Pubmästeriet ansvarar för sektionens pubverksamhet.

	\item[Sammansättning]
	Sexmästeriet består av
	\begin{itemize}
		\item Sexmästare
		\item Vice Sexmästare
		\item DJ
		\item Köksmästare
		\item Vice köksmästare
		\item Pubmästare
		\item Vice Pubmästare
		\item Barmästare
		\item Vice Barmästare
		\item Sektionskock
		\item Sångförman
		\item Preferensmästare
		\item Ölförman
		\item Hovmästare
		\item Vinförman
		\item Funktionärer
	\end{itemize}
\end{reglemlista}

%%%%%%%%%%%%% SECTION %%%%%%%%%%%%%
\section{Skattmästeriet}

\begin{reglemlista}

	\item[Åligganden]
	Det åligger skattmästeriet
	\begin{attlista}
		\item handha sektionens bokföring
		\item handha sektionens transaktioner
		\item skriva budgetförslag
		\item samordna sektionens ekonomiska verksamhet
		\item följa och bevaka sektionens likviditet
	\end{attlista}

	\item[Sammansättning]
	Skattmästeriet består av
	\begin{itemize}
		\item Skattmästare
		\item Vice Skattmästare
		\item Funktionärer
	\end{itemize}
\end{reglemlista}

%%%%%%%%%%%%% SECTION %%%%%%%%%%%%%
\section{Nollningsutskottet}

\begin{reglemlista}

	\item[Åligganden]
	Det åligger Nollningsutskottet
	\begin{attlista}
		\item med hjälp av sektionen planera och genomföra nollningen
		\item tillse att tredje part ej drabbas
		\item ansvara för att aktiviteter under nollningen visar positiv bild av D-sektionen och Teknologkåren
		\item vara goda representanter för D-sektionen och Teknologkåren under nollningen
		\item inom Nollningsutskottet utse ett ställföreträdande Øverphøs
	\end{attlista}

	\item[Sammansättning]
	Nollningsutskottet består av
	\begin{itemize}
		\item Øverphøs
		\item Stabsmedlemmar
		\item \O verpeppare
		\item Peppare
		\item Nollningsfunktionärer
	\end{itemize}

	\item[Staben]
	Staben består av
	\begin{itemize}
		\item Øverphøs
		\item Stabsmedlemmar
	\end{itemize}

\end{reglemlista}

%%%%%%%%%%%%% SECTION %%%%%%%%%%%%%
\section{Studierådet}

\begin{reglemlista}

	\item[Åligganden]
	Det åligger studierådet
	\begin{attlista}
		\item ansvara för sektionens studiebevakning
		\item välja
		\begin{itemize}
			\item Vice Studierådsordförande
			\item Studierådssekreterare
			\item Likabehandlingsombud
			\item Kinaansvarig
			\item Studerandeskyddsombud
			\item Världsmästare
			\item Studentrepresentant, HMS-kommité
			\item Studentrepresentant, Husstyrelse
			\item Studentrepresentant, Institutionsstyrelse
			\item Studentrepresentant, Programledning
			\item Årskursrepresentant, D1
			\item Årskursrepresentant, C1
			\item Årskursrepresentant, D2
			\item Årskursrepresentant, C2
			\item Årskursrepresentant, D3
			\item Årskursrepresentant, C3

			\item Funktionärer
		\end{itemize}
		\item föreslå representanter till TLTH:s externa organ
	\end{attlista}

	\item[Sammanträde]
	Studierådet sammanträder på kallelse av Studierådsordföranden eller vid
	dennes frånvaro av Vice Studierådsordföranden. Rätt att begära utlysning av
	studierådsmöte har
	\begin{enumerate}
		\item minst 10 sektionsmedlemmar
		\item revisor
		\item Inspektor
	\end{enumerate}

	\item[Kallelse]
	Kallelse till studierådsmöte samt föredragningslista skall senast fem
	dagar före sammanträdet anslås på sektionens
	anslagstavla.

\end{reglemlista}

\section{Trivselrådet}

\begin{reglemlista}

	\item[Åligganden]
	Det åligger Trivselrådet
	\begin{attlista}
		\item driva sektionens arbete med frågor som berör likabehandling, internationalisering samt arbetsmiljö
		\item erbjuda en kanal där sektionsmedlemmar kan vända sig till med frågor som berör likabehandling, internationalisering samt arbetsmiljö
		\item rapportera inkomna händelser till styrelsen
		\item stötta sektionsmedlemmar som har utsatts för kränkande behandling
	\end{attlista}
	\item[Sammansättning]
	Trivselrådet består av
	\begin{itemize}
		\item Trivselmästare
		\item Likabehandlingsombud
		\item Studerandeskyddsombud
		\item Världsmästare
		\item Her Tech Future representant
	\end{itemize}

\end{reglemlista}

\section{Framtidsutskottet}

\begin{reglemlista}

	\item[Åligganden]
	Det åligger Framtidsutskottet
	\begin{attlista}
		\item föra tillfredställande protokoll över sitt arbete samt tillgängliggöra dessa för sektionen
		\item bereda förslag för att presentera till styrelsen eller sektionsmötet
		\item  förankra sin arbetsplan hos styrelsen eller sektionsmötet
		\item arbeta strategiskt för att möjligöra långsiktiga mål
	\end{attlista}
	\item[Sammansättning]
	Framtidsutskottet består av
	\begin{itemize}
		\item Framtidsordförande
		\item Framtidsledamöter
	\end{itemize}

\end{reglemlista}

%%%%%%%%%%%%% SECTION %%%%%%%%%%%%%
\section{Funktionärer}

Samtliga funktionärsposter har mandatperioden kalenderår (1/1 - 31/12), om inget annat anges.

%%%%%%%%%%%%% STYRELSEPOSTER %%%%%%%%%%%%%

%%% funktionar %%%
\funktionar{Ordföranden}
Ordföranden leder styrelsens arbete och är sektionens ansikte utåt.

\begin{reglemlista}

	\item[Åligganden]
	Det åligger Ordföranden
	\begin{attlista}
		\item representera och föra sektionens talan
		\item kalla till styrelsemöten och upprätta erforderliga handlingar
		\item leda och fördela arbetet inom styrelsen
		\item kalla till sektionsmöte och upprätta erforderliga handlingar
		\item boka lokal till sektionsmöte
		\item hålla kontakt med TLTH:s ordförande samt övriga sektionsordförande
		\item hålla kontakt med Inspektor, vaktmästaren och husstyrelsen
		\item hålla kontakt med de övriga D-sektionerna i landet
		\item kontinuerligt rapportera vad som händer på sektionen via sektionens informationskanaler
		\item vid begäran skriva intyg till avgående funktionär; avgående ordförandens intyg skrivs av Inspektor tillsammans med tillträdande ordföranden
	\end{attlista}

\end{reglemlista}


%%% funktionar %%%
\funktionar{Informationsansvarig}
Informationsansvarig ansvarar för sektionens informationsspridning samt att upprätthålla
kommunikation mellan styrelsen och sektionen. Informationsansvarig leder även Informationsutskottet.

\begin{reglemlista}

	\item[Åligganden]
	Det åligger Informationsansvarig
	\begin{attlista}
		\item leda Informationsutskottet
		\item vara ansvarig utgivare av sektionens medier
		\item hålla sektionens anslagstavlor och informationsskärmar i god ordning
		\item publicera mötesprotokoll på WWW
		\item upprätthålla god kommunikation mellan styrelsen och sektionens medlemmar

		\item kontinuerligt rapportera vad som händer på sektionen via sektionens informationskanaler
	\end{attlista}

\end{reglemlista}

%%% funktionar %%%
\funktionar{Skattmästare}
Skattmästaren handhar sektionens ekonomi, sköter bokföring och ekonomisk
uppföljning. Skattmästaren leder även skattmästeriet.

\begin{reglemlista}

	\item[Åligganden]
	Det åligger Skattmästaren
	\begin{attlista}
		\item leda skattmästeriet
		\item presentera bokslut och halvårsbokslut
		\item redovisa sektionens ekonomiska status för styrelsen i slutet av varje läsperiod
		\item hålla kontakten med generalsekreteraren på Teknologkåren
	\end{attlista}

\end{reglemlista}

%%% funktionar %%%
\funktionar{Studierådsordförande}
Studierådsordföranden är ansvarig för sektionens studiebevakning.
Studierådsordföranden leder även studierådet.

\begin{reglemlista}

	\item[Åligganden]
	Det åligger Studierådsordförande
	\begin{attlista}
		\item leda studierådet
		\item hålla kontakten med relevanta utbildningsnämnder
		\item aktivt deltaga i SRX, Teknologkårens studierådsordförandekollegium
	\end{attlista}

\end{reglemlista}

%%% funktionar %%%
\funktionar{Cafémästare}
Cafémästaren är ansvarig för sektionens caféverksamhet. Cafémästaren leder även cafémästeriet.

\begin{reglemlista}

	\item[Åligganden]
	Det åligger Cafémästaren
	\begin{attlista}
		\item leda cafémästeriet
	\end{attlista}

\end{reglemlista}

%%% funktionar %%%
\funktionar{Näringslivsansvarig}
Näringslivsansvarig samordnar sektionens kontakter med näringslivet.
Näringslivsansvarig leder även Näringslivsutskottet.

\begin{reglemlista}

	\item[Åligganden]
	Det åligger Näringslivsansvarig
	\begin{attlista}
		\item leda Näringslivsutskottet
		\item verka för att information om datateknik- och infocomutbildningarna vid LTH sprids till företag och näringsliv
		\item aktivt deltaga i Teknologkårens näringslivskollegium
	\end{attlista}

\end{reglemlista}

%%% funktionar %%%
\funktionar{Källarmästare}
Källarmästaren är ansvarig för sektionens lokaler och inventarier.
Källarmästaren leder även källarmästeriet.

\begin{reglemlista}

	\item[Åligganden]
	Det åligger Källarmästaren
	\begin{attlista}
		\item leda källarmästeriet
		\item administrera dörraccess och nyckelåtkomst till iDéts faciliteter

	\end{attlista}

\end{reglemlista}

%%% funktionar %%%
\funktionar{Aktivitetsansvarig}
Aktivitetsansvarig är ansvarig för sektionens studiesociala verksamheter,
undantaget fest- och pubverksamhet. Aktivitetsansvarig leder även Aktivitetsutskottet.

\begin{reglemlista}

	\item[Åligganden]
	Det åligger Aktivitetsansvarig
	\begin{attlista}
		\item leda Aktivitetsutskottet
		\item verka för att sektionen har lag med i tandemstafetten
		\item hålla kontakt med aktivitetssamordnaren på Teknologkåren samt övriga sektioners Aktivitetsansvarig
		\item anordna för sektionens medlemmar trevliga evenemang
		\item informera om större arrangemang på andra orter
	\end{attlista}

\end{reglemlista}

%%% funktionar %%%
\funktionar{Sexmästare}
Sexmästaren är ansvarig för sektionens festverksamhet. Sexmästaren leder även Sexmästeriet.

\begin{reglemlista}

	\item[Åligganden]
	Det åligger Sexmästaren
	\begin{attlista}
		\item leda sexmästeriet
		\item inför varje fest i E-huset utse en festansvarig i enlighet med husstyrelsens regler, samt underrätta vaktmästaren om denne
	\end{attlista}

\end{reglemlista}

%%% funktionar %%%
\funktionar{Øverphøs}
Øverphøs är ansvarig för sektionens nollning.

\begin{reglemlista}

	\item[Åligganden]
	Det åligger Øverphøs
	\begin{attlista}
		\item kontinuerligt informera styrelsen om verksamheten
		\item före nollningens början lägga fram nollningens budgetförslag till styrelsen
		\item samarbeta med Nollegeneralen, de andra Øverphøsen och LTH
	\end{attlista}

\end{reglemlista}

%%%%%%%%%%%%% VICEPOSTER I STYRELSEN %%%%%%%%%%%%%

%%% funktionar %%%
\funktionar{Vice Ordförande}
Vice Ordförande skall vara Ordförande behjälplig, föra protokoll under styrelsemöten och sektionsmöten.

\begin{reglemlista}

	\item[Åligganden]
	Det åligger Vice Ordförande
	\begin{attlista}
		\item i Ordförandes frånvaro, leda Styrelsen.
		\item vara Ordförande behjälplig, när så denne behöver det.
		\item ansvara för att Policy-sidan, Stadgarna och Reglementet 			hålls uppdaterade.
		\item vara kontaktperson för fria föreningar och eventuella projekt
		\item anslå kallelser till styrelse- och sektionsmöten
		\item föra protokoll vid styrelsemöten och sektionsmöten samt anslå dessa
		\item arkivera uppförda protokoll och övriga handlingar i därför avsedd pärm
		\item bistå ordförande i förberedande av sektionsmöten
	\end{attlista}

	\item[Titulering]
	Alternativ titulering är Sekreterare

\end{reglemlista}

%%% funktionar %%%
\funktionar{Vice Informationsansvarig}
Vice Informationsansvarig hjälper Informationsansvarig i dennes arbete.

\begin{reglemlista}

	\item[Åligganden]
	Det åligger Vice Informationsansvarig
	\begin{attlista}
		\item hjälpa Informationsansvarig i dennes arbete
		\item överta Informationsansvarigs funktion i dennes frånvaro
	\end{attlista}

\end{reglemlista}

%%% funktionar %%%
\funktionar{Vice Skattmästare}
Vice Skattmästaren hjälper Skattmästaren i dennes arbete.

\begin{reglemlista}

	\item[Åligganden]
	Det åligger Vice Skattmästaren
	\begin{attlista}
		\item Hjälpa Skattmästaren i dennes arbete
		\item sköta, av skattmästaren tilldelat mästeris bokföring och ekonomi
		\item hjälpa sitt tilldelade mästeri i dess arbete
	\end{attlista}

\end{reglemlista}

%%% funktionar %%%
\funktionar{Vice Cafémästare}
Vice Cafémästaren hjälper Cafémästaren i dennes arbete.

\begin{reglemlista}

	\item[Åligganden]
	Det åligger Vice Cafémästaren
	\begin{attlista}
		\item hjälpa Cafémästaren i dennes arbete
		\item överta Cafémästarens funktion i dennes frånvaro
	\end{attlista}

\end{reglemlista}

%%% funktionar %%%
\funktionar{Vice Näringslivsansvarig}
Vice Näringslivsansvarig hjälper Näringslivsansvarig i dennes arbete.

\begin{reglemlista}

	\item[Åligganden]
	Det åligger Vice Näringslivsansvarig
	\begin{attlista}
		\item hjälpa Näringslivsansvarig i dennes arbete
		\item i Näringslivsansvariges frånvaro leda Näringslivsutskottet
	\end{attlista}

\end{reglemlista}

%%% funktionar %%%
\funktionar{Vice Källarmästare}
Vice Källarmästaren hjälper Källarmästaren i dennes arbete.

\begin{reglemlista}

	\item[Åligganden]
	Det åligger Vice Källarmästaren
	\begin{attlista}
		\item hjälpa Källarmästaren i dennes arbete
		\item överta Källarmästarens funktion i dennes frånvaro
	\end{attlista}

\end{reglemlista}

%%% funktionar %%%
\funktionar{Vice Aktivitetsansvarig}
Vice Aktivitetsansvarig hjälper Aktivitetsansvarig i dennes arbete.

\begin{reglemlista}

	\item[Åligganden]
	Det åligger Vice Aktivitetsansvarig
	\begin{attlista}
		\item hjälpa Aktivitetsansvarig i dennes arbete
		\item överta Aktivitetsansvarigs funktion i dennes frånvaro
	\end{attlista}

\end{reglemlista}

%%% funktionar %%%
\funktionar{Vice Sexmästare}
Vice Sexmästaren hjälper Sexmästaren i dennes arbete.

\begin{reglemlista}

	\item[Åligganden]
	Det åligger Vice Sexmästaren
	\begin{attlista}
		\item hjälpa Sexmästaren i dennes arbete
		\item överta Sexmästarens funktion i dennes frånvaro
	\end{attlista}

\end{reglemlista}

%%% funktionar %%%
\funktionar{Vice Studierådsordförande}
Vice Studierådsordförande hjälper Studierådsordföranden i dennes arbete.

\begin{reglemlista}

	\item[Åligganden]
	Det åligger Vice Studierådsordföranden
	\begin{attlista}
		\item hjälpa Studierådsorföranden i dennes arbete
		\item i Studierådsordförandens frånvaro leda studierådet
	\end{attlista}

	\item[Val]
	Vice Studierådsordförande väljs av Studierådet.
\end{reglemlista}

%%%%%%%%%%%%% ÖVRIGA FUNKTIONÄRER %%%%%%%%%%%%%

%%% funktionar %%%
\funktionar{Alumniansvarig}
\begin{reglemlista}

	\item[Åligganden]
	Det åligger alumniansvarig

	\begin{attlista}
		\item med hjälp av alumnigruppen se till att sektionens alumniverksamhet fortskrider
		\item leda alumnigruppen
	\end{attlista}

\end{reglemlista}

%%% funktionar %%%
\funktionar{Alumnigruppsmedlem}

\begin{reglemlista}

	\item[Åligganden]
	Det åligger alumnigruppsmedlem
	\begin{attlista}
		\item bistå alumniansvarig i dess arbete kring alumniverksamheten
	\end{attlista}

\end{reglemlista}

%%% funktionar %%%
\funktionar{Alumnimiddagsansvarig}

\begin{reglemlista}

	\item[Åligganden]
	Det åligger alumnimiddagsansvarig
	\begin{attlista}
		\item ansvara för arrangerandet av sektionens alumnimiddag
	\end{attlista}
	\item[Mandatperiod]
	Karnevalsansvariges mandatperiod är 1 juli - 30 juni

\end{reglemlista}

%%% funktionar %%%
\funktionar{Arkivarie}
Arkivarien ska handha sektionens arkiv.

\begin{reglemlista}

	\item[Åligganden]
	Det åligger Arkivarien
	\begin{attlista}
		\item föra och underhålla sektionens arkiv
		\item föra och underhålla sektionens dataarkiv
	\end{attlista}

\end{reglemlista}

%%% funktionar %%%
\funktionar{Artist}
Artisten utför de artistiska arbetena på sektionen.

\begin{reglemlista}

	\item[Åligganden]
	Det åligger Artisten
	\begin{attlista}
		\item leda och utföra de artistiska arbetena på sektionen
	\end{attlista}

\end{reglemlista}

%%% funktionar %%%
\funktionar{Ljud- och ljusansvarig}
Ljud- och ljusansvarig är ansvarig för sektionens ljud- och
ljusutrustning.

\begin{reglemlista}

	\item[Åligganden]
	Det åligger Ljud- och ljusansvarig
	\begin{attlista}
		\item sköta och om serva iDéts ljud- och ljusutrustning
		\item då utrustningen skall användas åtse att kompetent personal finns på plats för att sköta densamma
	\end{attlista}

\end{reglemlista}

%%%funktionar%%%

\funktionar{Sektionslivskvalitetsförhöjare}

\begin{reglemlista}

	\item[Åligganden]
	Det åligger Sektionslivskvalitetsförhöjaren
	\begin{attlista}
		\item  se till att kaffe och te finns tillgängligt för sektionens funktionärer

		\item underhålla och rengöra sektionens kaffekokare, samt rengöra sektionen vattenkokare

		\item sköta microvågsungarna

		\item sköta påfyllning av bordershop

		\item sköta rengörning av borden
	\end{attlista}

\end{reglemlista}

%%% funktionar %%%
\funktionar{DJ}
DJ ansvarar för att passande musik spelas då sexmästaren så önskar.

\begin{reglemlista}

	\item[Åligganden]
	Det åligger DJ:n
	\begin{attlista}
		\item se till att passande musik spelas vid lämpliga tillfällen enligt sexmästerens önskemål
		\item underhålla sektionens musiksamling och spellistor
	\end{attlista}

\end{reglemlista}

%%% funktionar %%%
\funktionar{DWWW-ansvarig}
DWWW-ansvarig leder DWWW.

\begin{reglemlista}

	\item[Åligganden]
	Det åligger DWWW-ansvarig
	\begin{attlista}
		\item leda DWWW
		\item ansvara för sektionens WWW-sidor
	\end{attlista}

\end{reglemlista}

%%% funktionar %%%
\funktionar{DWWW-medlem}
DWWW-medlem hjälper DWWW-ansvarig med arbetet inom DWWW.

\begin{reglemlista}

	\item[Åligganden]
	Det åligger DWWW-medlem
	\begin{attlista}
		\item hjälpa DWWW-ansvarig med att underhålla och utveckla sektionens WWW-sidor
	\end{attlista}

\end{reglemlista}

%%% funktionar %%%
\funktionar{Her Tech Future Representant}
Her Tech Future Representant är ansvarig för Her Tech Future-arrangemanget på
sektionen.

\begin{reglemlista}

	\item[Åligganden]
	Det åligger Her Tech Future Representant
	\begin{attlista}
		\item framföra en positiv bild av LTH
		\item ansvara för att sektionen är representerad på ett bra sätt under Her Tech Future
	\end{attlista}

	\item[Mandatperiod]
	Her Tech Future Representant mandatperiod är 1~juli -- 30~juni.

\end{reglemlista}

%%% funktionar %%%
\funktionar{Fotograf}
Fotografen ska dokumentera händelser på sektionen för eftervärlden.

\begin{reglemlista}

	\item[Åligganden]
	Det åligger Fotografen
	\begin{attlista}
		\item fotografera alla för sektionen och mänskligheten viktiga händelser på sektionen
		\item snarast möjligt göra fotografierna tillgängliga för sektionens medlemmar
	\end{attlista}

\end{reglemlista}

%%% funktionar %%%
\funktionar{Hovmästare}
Hovmästaren är den person som har det övergripande ansvaret för sexmästeriets servering och dukning.

\begin{reglemlista}

	\item[Åligganden]
	Det åligger Hovmästaren
	\begin{attlista}
		\item vara ansvarig för Sexmästeriets servering, dukning, bordsplacering och schemaläggning
		\item tillgodose sektionens behov av sittningsdryck till sittningar
		\item hjälpa Sexmästaren
	\end{attlista}

\end{reglemlista}

%%% funktionar %%%
\funktionar{Vinförman}
Vinförmannen är den person som har det övergripande ansvaret för sexmästeriets vinsortiment.

\begin{reglemlista}

	\item[Åligganden]
	Det åligger Vinförmannen
	\begin{attlista}
		\item i samråd med pubmästaren och barmästaren se till att sexmästeriet har ett tillräckligt vinsortiment till pubar och andra tillställningar exklusive sittningar
		\item hjälpa hovmästaren att tillgodose sektionen med sittningsdryck till sittningar

	\end{attlista}

\end{reglemlista}

%%% funktionar %%%
\funktionar{Idrottsförman}
Idrottsförmannen ska tillse sektionsmedlemmarnas fysiska hälsa.

\begin{reglemlista}

	\item[Åligganden]
	Det åligger Idrottsförmannen
	\begin{attlista}
		\item tillgodose medlemmarnas behov av motion
		\item gå på kollegiemöte för kårens idrottsförmän när så 		sammankallas
	\end{attlista}

\end{reglemlista}

%%% funktionar %%%
\funktionar{Jubileumsansvarig}
Jubileumsansvarig arrangerar, under Jubileumsgeneralens ledning, D-sektionensjubileum.

\begin{reglemlista}
	\item[Åligganden]
	Det åligger Jubileumsansvarig
	\begin{attlista}
		\item bistå Jubileumsgeneralens arbete under jubileumet
	\end{attlista}
	
	\item[Mandatperiod]
	Jubileumsansvarigs mandatperiod är 1 juli - 30 juni
\end{reglemlista}

%%% funktionar %%%
\funktionar{Jubileumsgeneral}
Jubileumsgeneral ansvarar för arrangerandet av D-sektionens jubileum.

\begin{reglemlista}
	\item[Åligganden]
	Det åligger Jubileumsgeneralen
	\begin{attlista}
		\item ha det övergripande ansvaret för arrangerandet av D-sektionens jubileum
		\item informera sektionens styrelse om jubileumets fortskridande
		\item före jubileumets start presentera budgetförslag för sektionens styrelse
	\end{attlista}
	\item[Mandatperiod]
	Jubileumsgeneralens mandatperiod är 1 juli - 30 juni
\end{reglemlista}

%%% funktionar %%%
\funktionar{Trivselmästare}
Trivselmästaren är ansvarig för likabehandling- och internationaliseringsarbetet på sektionen. Trivselmästaren ska driva sektionens arbetsmiljöfrågor. Trivselmästaren leder trivselrådet.

\begin{reglemlista}

	\item[Åligganden]
	Det åligger Trivselmästaren

	\begin{attlista}
		\item se till att trivselrådet har en närvaro på Studierådets möten
		\item tillsammans med Likabehandlingsombuden driva sektionens likabehandlingsarbete
		\item ansvara för sektionens internationalisering
		\item tillsammans med Studerandeskyddsombud, bevaka Sektionens intressen i arbetsmiljöfrågor
		\item vara ett studerandeskyddsombud för D-sektionen inom tlth
		\item informera styrelsen om arbetet som sker inom trivselrådet
	\end{attlista}

\end{reglemlista}

%%% funktionar %%%
\funktionar{Likabehandlingsombud}

\begin{reglemlista}

	\item[Åligganden]
	Det åligger Likabehandlingsombudet

	\begin{attlista}
		\item bevaka sektionens verksamhet från ett likabehandlingsperspektiv
		\item uppmärksamma styrelsen på situationer och miljöer som skulle kunna upplevas som kränkande av medlemmar ur sektionen eller bryta mot Teknologkårens likabehandlingspolicy
		\item driva likabehandlingsfrågor inom utbildningen i Trivselrådet
		\item vara studerandeskyddsombud för D-sektionen inom tlth
	\end{attlista}

	\item[Val]
	Likabehandlingsombudet väljs av Studierådet

\end{reglemlista}

%%% funktionar %%%
\funktionar{Karnevalsansvarig}
Karnevalsansvarig har hand om sektionens deltagande i karnevalen och väljs var fjärde år.

\begin{reglemlista}

	\item[Åligganden]
	Det åligger Karnevalsansvarig
	\begin{attlista}
		\item se till att sektionen deltager i karnevalens arrangemang såsom tåg- och tältnöjen
		\item presentera en budget över sin verksamhet för styrelsen senast
		tre månader före karnevalen äger rum
		\item hålla kontakten med Karnevalskommittén
	\end{attlista}

	\item[Mandatperiod]
	Karnevalsansvariges mandatperiod är 1 juli - 30 juni

	\item[Val]
	Karnevalsansvarig väljs var fjärde år

\end{reglemlista}

%%% funktionar %%%
\funktionar{Kinaansvarig}
Den Kinaansvarige har det specifika ansvaret för studiebevakningen av Kinainriktningen, samt fungerar som studierådets högra hand på plats i Kina

\begin{reglemlista}

	\item[Åligganden]
	Det åligger Kinaansvarig

	\begin{attlista}
		\item inneha en av representantposterna i Kinarinriktningens programledning samt att närvara vid dess möten
		\item informera studierådet samt berörda studenter om det rådande läget vad gäller planering och organisation av inriktningen
		\item säkerställa att de kurser som ges för och av LTH inom ramarna för Kinainriktningen håller hög kvalité
		\item i övrigt bistå de av sektionens studenter som studerar i Kina även utanför ramarna för Kinainriktningen så länge detta inte inkräktar på något av ovan nämnda åligganden
	\end{attlista}

	\item[Val]
	Kinaansvarig väljs av Studierådet

\end{reglemlista}

%%% funktionar %%%
\funktionar{Kårrepresentant}
Kårrepresentanten är ansvarig för att sektionen hålls underrättad om de beslut som kåren
fattar som berör sektionen samt skapar en diskussion kring dessa.

\begin{reglemlista}

	\item[Åligganden]
	Det åligger Kårrepresentanten
	\begin{attlista}
		\item kontinuerligt informera sektionsstyrelsen om vad som händer på kåren
		\item under sektionsmöten i avsaknad av sektionsrepresentant från kåren informera om
		vad som händer på kåren
		\item kontinuerligt kommunicera med kårens sektionskontakter
		\item verka för att sektionen har en god representation i fullmäktige och kårens övriga organ
	\end{attlista}

\end{reglemlista}

%%% funktionar %%%
\funktionar{Köksmästare}
Köksmästaren har det övergripande ansvaret för Sexmästeriets matlagning under deras evenemang.

\begin{reglemlista}

	\item[Åligganden]
	Det åligger Köksmästaren
	\begin{attlista}
		\item vara ansvarig för Sexmästeriets matlagning under deras evenemang
	\end{attlista}
\end{reglemlista}

%%% funktionar %%%
\funktionar{Sektionskock}
\begin{reglemlista}
	\item[Åligganden]
	Det åligger Sektionskocken
	\begin{attlista}
		\item laga mat utanför sittningar och pubbar
	\end{attlista}
\end{reglemlista}

%%% funktionar %%%
\funktionar{Preferensmästare}
\begin{reglemlista}
	\item[Åligganden]
	Det åligger Preferensmästaren
	\begin{attlista}
		\item vara ansvarig för Sexmästeriets matlagning gällande specialkost under deras evenemang
	\end{attlista}
\end{reglemlista}

%%% funktionar %%
\funktionar{Vice köksmästare}
Vice Köksmästaren hjälper Köksmästaren i dennes arbete.
\begin{reglemlista}

	\item[Åligganden]
	Det åligger Vice Köksmästaren
	\begin{attlista}
		\item hjälpa Köksmästaren i dennes arbete
		\item överta Köksmästarens funktion vid dennes frånvaro
	\end{attlista}

\end{reglemlista}

%%% funktionar %%%
\funktionar{LANparty-ansvarig}
LANparty-ansvarig ansvarar för att anordna LAN-partyn.

\begin{reglemlista}

	\item[Åligganden]
	Det åligger LANparty-ansvarig
	\begin{attlista}
		\item anordna LAN-partyn
	\end{attlista}

\end{reglemlista}


%%% funktionar %%%
\funktionar{Medaljelelekommittémedlem}
Medaljelelekommittémedlemen ingår i medaljelelekommittén och sköter om utdelning av medaljer och riddarskap.

\begin{reglemlista}

	\item[Åligganden]
	Det åligger Medaljelelekommittémedlemmen
	\begin{attlista}
		\item närvara på Medaljelelekommitténs möten
		\item lusteligen dela ut medaljer och andra utmärkelser till funktionärer som gjort sig förtjänta därav
	\end{attlista}
\end{reglemlista}

%%% funktionar %%%
\funktionar{Medlem i Projektgruppen för Teknikfokus}

\begin{reglemlista}
	\item[Åligganden]
	Det åligger Medlem i Projektgruppen för Teknikfokus
	\begin{attlista}
		\item bistå Teknikfokusansvarig i dennes arbete
	\end{attlista}

	\item[Mandatperiod]
	Mandatperioden för Medlem i Projektgruppen för Teknikfokus är 1 juli - 30 juni.
\end{reglemlista}

%%% funktionar %%%
\funktionar{Mentorsansvarig}
Mentorsanvarig är ytterst ansvarig för all verksamhet som rör Mentorsprogrammet.

\begin{reglemlista}

	\item[Åligganden]
	Det åligger Mentorsansvarig
	\begin{attlista}
		\item driva och utveckla D-sektionens Mentorsprogram
	\end{attlista}

\end{reglemlista}

%%% funktionar %%%
\funktionar{Märkvärdig}
Märkvärdig sköter försäljning av sektionssymbolerna.

\begin{reglemlista}

	\item[Åligganden]
	Det åligger Märkvärdig
	\begin{attlista}
		\item sköta försäljningen av sektionssymboler
	\end{attlista}

\end{reglemlista}

%%% funktionar %%
\funktionar{Nollningsfunktionär}
Nollningsfunktionärer verkar under nollningen för att välkomna de nya studenterna till D-sektionen.

\begin{reglemlista}

	\item[Åligganden]
	Det åligger nollningsfuntionärer
	\begin{attlista}
		\item vara Nollningsutskottet behjälplig under nollningen
		\item välkommna de nya studenterna till D-sektionen
	\end{attlista}

	\item[Mandatperiod]
	Nollningsfunktionärens mandatperiod är 1~april -- 30~september

\end{reglemlista}

%%% funktionar %%%
\funktionar{Pubmästare}
Pubmästaren ansvarar för pubverksamheten.

\begin{reglemlista}

	\item[Åligganden]
	Det åligger Pubmästaren
	\begin{attlista}
		\item anordna pubar under nollningen och vid tillfällen då så anses lämpligt
		\item tillsammans med barmästaren hålla ett högkvalitativt öl- och drinksortiment samt se till att flera alkoholfria alternativ alltid finns
		\item vara sexmästaren och skattmästeriet behjälplig gällande pubens ekonomi och bokföring
	\end{attlista}

\end{reglemlista}

%%% funktionar %%%
\funktionar{Vice Pubmästare}
Vice Pubmästaren hjälper Pubmästaren i dennes arbete.

\begin{reglemlista}

	\item[Åligganden]
	Det åligger Vice Pubmästaren
	\begin{attlista}
		\item hjälpa Pubmästaren i dennes arbete
		\item överta Pubmästarens funktion vid dennes frånvaro
	\end{attlista}

\end{reglemlista}

%%% funktionar %%%
\funktionar{Barmästare}
Barmästaren ansvarar för barverksamheten under sittningar.

\begin{reglemlista}

	\item[Åligganden]
	Det åligger Barmästaren
	\begin{attlista}
		\item anordna barverksamhet i samband med sittningar och eftersläpp arrangerade av sexmästeriet
		\item tillsammans med pubmästaren hålla ett högkvalitativt öl- och drinksortiment samt se till att flera alkoholfria alternativ alltid finns
	\end{attlista}

\end{reglemlista}

%%% funktionar %%%
\funktionar{Vice Barmästare}
Vice Barmästaren hjälper Barmästaren i dennes arbete.

\begin{reglemlista}

	\item[Åligganden]
	Det åligger Vice Barmästaren
	\begin{attlista}
		\item hjälpa Barmästaren i dennes arbete
		\item överta Barmästarens roll vid dennes frånvaro
	\end{attlista}

\end{reglemlista}


%%% funktionar %%%
\funktionar{Revisor}
Revisorerna skall granska sektionens verksamhet och bokslut.

\begin{reglemlista}

	\item[Krav]
	Det krävs
	\begin{attlista}
		\item revisorerna är myndiga svenska medborgare
		\item revisorerna har den insikt i ekonomiska förhållanden som uppdraget kräver
		\item en av revisorerna har god insyn i sektionens verksamhet
		\item revisorerna inte är försatta i konkurs eller har näringsförbud
		\item revisorerna inte har förvaltare
		\item revisorerna inte är styrelseledamöter
		\item revisorerna bör inte vara gift, sambo eller nära släkt med någon i styrelsen
	\end{attlista}

	\item[Åligganden]
	Det åligger revisorerna
	\begin{attlista}
		\item granska sektionens böcker och räkenskaper
		\item taga del av sektionsmötenas och styrelsens protokoll
		\item verkställa fortlöpande inventering eller kontrollera ställd inventering av sektionens kassa och övriga tillgångar
		\item tillse huruvida sektionens organisation av och kontroll över bokföring är tillfredsställande
		\item senast fjorton dagar innan vårterminsmötet, året
		efter mandatperioden, till styrelsen inlämna revisionsberättelse
	\end{attlista}

	\item[Rättigheter]
	Revisorerna äger rätt
	\begin{attlista}
		\item närhelst han eller hon önskar taga del av sektionens samtliga räkenskaper, protokoll och andra handlingar
		\item beredas tillträde till alla sektionens lokaler
		\item erhålla upplysningar rörande sektionens verksamhet och förvaltning
		\item närvara vid sektionens samtliga möten som ständigt adjungerade
	\end{attlista}

	\item[Skyldigheter]
	Revisorerna är skyldiga
	\begin{attlista}
		\item alltid handla för sektionens bästa
	\end{attlista}

\end{reglemlista}

%%% funktionar %%%
\funktionar{root}
root leder Rootmästeriet.

\begin{reglemlista}

	\item[Åligganden]
	Det åligger root
	\begin{attlista}
		\item underhålla och förbättra sektionens datorer och därtill kopplad utrustning
	\end{attlista}

\end{reglemlista}

%%% funktionar %%%
\funktionar{Studerandeskyddsombud}
Studerandeskyddsombudet bevakar sektionens intressen i olika organ för
arbetsmiljöfrågor.

\begin{reglemlista}

	\item[Åligganden]
	Det åligger Studerandeskyddsombudet
	\begin{attlista}
		\item verka för god studiemiljö
		\item närvara på HMS-kommitténs möten
		\item verka som Brandskyddssamordnare för D-sektionen
	\end{attlista}

	\item[Val]
	Studerandeskyddsombudet väljs av Studierådet.

\end{reglemlista}

%%% funktionar %%%
\funktionar{Sparkyansvarig}
Sparkyansvarig ansvarar för Sparky.

\begin{reglemlista}

	\item[Åligganden]
	Det åligger Sparkyansvarig
	\begin{attlista}
		\item underhålla och uppdatera spelmaskinen Sparky
	\end{attlista}

\end{reglemlista}

%%funktionar %%
\funktionar{Stekare}
Stekaren tillagar varor som används i maträtterna som serveras i Café iDét.
\begin{reglemlista}
	\item[Åligganden]
	Det åligger Stekaren
	\begin{attlista}
		\item med lämpligt intervall
		tillaga mat åt Cafémästeriets dagliga verksamhet, så att det alltid finns tillgängligt
	\end{attlista}

\end{reglemlista}

%%funktionar %%
\funktionar{Dagsansvarig}

\begin{reglemlista}
	\item[Åligganden]
	Det åligger Dagsansvarig
	\begin{attlista}
		\item ansvara för sektionens café under dennes tilldelade arbetspass.
	\end{attlista}

\end{reglemlista}

%%% funktionar %%%
\funktionar{Stabsmedlem}
Stabsmedlem arrangerar, under Øverphøs ledning, nollning.

\begin{reglemlista}

	\item[Åligganden]
	Det åligger Stabsmedlem
	\begin{attlista}
		\item bistå Øverphøs i dennes arbete
	\end{attlista}

\end{reglemlista}

%%% funktionar %%%
\funktionar{Studentrepresentant, HMS-kommité}
Studentrepresentanten för fram sektionens och studierådets åsikter gentemot den HMS-kommité
som studenten sitter i. Studentrepresentanten ska förmedla sektionens åsikter
inför, under och efter dessa möten. Åsikterna som förmedlas ska vara förankrade
i studierådet. Studentrepresentanten ansvarar för att informera studierådet om det
som tagits upp på mötet för HMS-kommitén.

\begin{reglemlista}

	\item[Åligganden]
	Det åligger Studentrepresentant, HMS-kommité
	\begin{attlista}
		\item  medverka på HMS-kommitémöten för den HMS-kommité som studentrepresentanten
		blivit vald till av TLTHs fullmäktige
		\item förmedla studierådets åsikter på HMS-kommitémöten
		\item informera studierådet om kommande HMS-kommitémöten
		\item informera studierådet om vad som sagts på tidigare HMS-kommitémöten

	\end{attlista}

\end{reglemlista}

%%% funktionar %%%
\funktionar{Studentrepresentant, Husstyrelse}
Studentrepresentanten för fram sektionens och studierådets åsikter gentemot den husstyrelse som studenten sitter i. Studentrepresentanten ska förmedla sektionens åsikter inför,
under och efter dessa möten. Åsikterna som förmedlas ska vara förankrade i studierådet. Studentrepresentanten ansvarar för att informera studierådet om det som tagits upp på mötet för husstyrelsen.

\begin{reglemlista}

	\item[Åligganden]
	Det åligger Studentrepresentant, Husstyrelse
	\begin{attlista}
		\item medverka på husstyrelsemöten för den husstyrelse som studentrepresentanten blivit vald till av TLTHs fullmäktige
		\item förmedla studierådets åsikter på husstyrelsemöten
		\item informera studierådet om kommande Husstyrelsemöten
		\item informera studierådet om vad som sagts på tidigare Husstyrelsemöten

	\end{attlista}
\end{reglemlista}

%%% funktionar %%%
\funktionar{Studentrepresentant, Institutionsstyrelse}
Studentrepresentanten för fram sektionens och studierådets åsikter gentemot institutionsstyrelsen som studenten sitter i. Studentrepresentanten ska förmedla sektionens åsikter inför, under och efter dessa möten. Åsikterna som förmedlas ska vara förankrade i studierådet. Studentrepresentanten ansvarar för att informera studierådet om det som tagits upp på institutionsstyrelsemötena.

\begin{reglemlista}

	\item[Åligganden]
	Det åligger Studentrepresentant, Institutionsstyrelse
	\begin{attlista}
		\item medverka på institutionsstyrelsemötena för den institution som studentrepresentanten blivit vald till av TLTHs fullmäktige
		\item förmedla studierådets åsikter på institutionsstyrelsemötena
		\item informera studierådet om kommande Institutionsstyrelsemöten
		\item informera studierådet om vad som sagts på tidigare Institutionsstyrelsemöten
	\end{attlista}

\end{reglemlista}

%%% funktionar %%%
\funktionar{Studentrepresentant, Programledning}
Studentrepresentanten för fram sektionens och studierådets åsikter gentemot den programledning som studenten sitter i. Studentrepresentanten ska förmedla sektionens åsikter inför, under och efter dessa möten. Åsikterna som förmedlas ska vara förankrade i studierådet. Studentrepresentanten ansvarar för att informera studierådet om det som tagits upp på mötet för programledningen

\begin{reglemlista}

	\item[Åligganden]
	Det åligger Studentrepresentant, Programledning
	\begin{attlista}
		\item medverka på programledningsmötena för den programledning som studentrepresentanten blivit vald till av TLTHs fullmäktige
		\item förmedla studierådets åsikter på programledningsmötena
		\item informera studierådet om kommande Programledningsmöten
		\item informera studierådet om vad som sagts på tidigare Programledningsmöten

	\end{attlista}
\end{reglemlista}


%%% funktionar %%%
\funktionar{Studierådssekreterare}
Studierådssekreterare för protokoll under studierådetsmöten.

\begin{reglemlista}

	\item[Åligganden]
	Det åligger Studierådssekreteraren
	\begin{attlista}
		\item föra protokoll under studierådsmöten
		\item se till att dessa förs upp på D-sektionens hemsida
	\end{attlista}

	\item[Val]
	Studierådssekreteraren väljs av Studierådet

\end{reglemlista}

%%% funktionar %%%
\funktionar{sudo}
sudo hjälper root i dennes arbete.

\begin{reglemlista}

	\item[Åligganden]
	Det åligger sudo
	\begin{attlista}
		\item hjälpa root i dennes arbete
		\item överta roots funktioner vid dennes frånvaro
	\end{attlista}

\end{reglemlista}

%%% funktionar %%%
\funktionar{Sångarstridsförman}
Sångarstridsförmannen är ansvarig för sektionens deltagande i Sångarstriden.

\begin{reglemlista}

	\item[Åligganden]
	Det åligger Sångarstridsförmannen
	\begin{attlista}
		\item se till att sektionen ställer upp i Sångarstriden
	\end{attlista}

\end{reglemlista}

%%% funktionar %%%
\funktionar{Sångförman}
Sångförmannen leder sången under sektionens fester.

\begin{reglemlista}

	\item[Åligganden]
	Det åligger Sångförmannen
	\begin{attlista}
		\item leda sektionens tillställningar
		\item  hjälpa hovmästaren sätta tidsschema på sektionens tillställningar, särskilt med avseende på gyckel och andra framträdanden, samt upprätthålla kontakt med dessa parter inför och under tillställningen
		\item lära nollan vett och etikett på sittningar
		\item göra sångblad då så erfodras
		\item lära nollan nollevisan
	\end{attlista}

\end{reglemlista}

%%% funktionar %%%
\funktionar{Talman}
Talmannen är mötesordförande på sektionsmötena.

\begin{reglemlista}

	\item[Åligganden]
	Det åligger Talmannen
	\begin{attlista}
		\item sitta som mötesordförande under sektionsmöten
		\item hjälpa styrelsen med förberedelserna inför sektionsmöten
	\end{attlista}

\end{reglemlista}

%%% funktionar %%%
\funktionar{Tandemgeneral}
Tandemgeneralen ska representera D-sektionen på Tandemstafetten, delta
vid dess planering och se till att D-sektionens medlemmar får chansen att delta.

\begin{reglemlista}

	\item[Åligganden]
	Det åligger Tandemgeneralen
	\begin{attlista}
		\item ge D-sektionens medlemmar en chans att delta i Tandemstafetten
	\end{attlista}

\end{reglemlista}

%%% funktionar %%%
\funktionar{Teknikfokusansvarig}
Teknikfokusansvarig har tillsammans med Näringslivsansvarig hand om
D-sektionens årliga arbetsmarknadsmässa Teknikfokus.

\begin{reglemlista}

	\item[Åligganden]
	Det åligger Teknikfokusansvarig
	\begin{attlista}
		\item tillsammans med Näringslivsansvarig ansvara för arrangerandet av Teknikfokus
		\item hjälpa Näringslivsansvarig att i god tid presentera en budget för Teknikfokus
	\end{attlista}
	\item[Mandatperiod]
	Teknikfokusansvariges mandatperiod är 1 juli - 30 juni
\end{reglemlista}

%%% funktionar %%%
\funktionar{Trädgårdsmästare}
Trädgårdsmästaren har hand om sektionens växter.

\begin{reglemlista}

	\item[Åligganden]
	Det åligger Trädgårdsmästaren
	\begin{attlista}
		\item sköta sektionens växter
	\end{attlista}

\end{reglemlista}

%%% funktionar %%%
\funktionar{Utedischoansvarig}
Utedischoansvarig ansvarar för D-sektionens arrangemang i det årliga utedischot.

\begin{reglemlista}

	\item[Åligganden]
	Det åligger Utedischoansvarig
	\begin{attlista}
		\item arrangera utedischo i samband med nollningen
		\item senast på det sista ordinarie styrelsemötet innan sommaruppehållet presentera en budget för styrelsen
	\end{attlista}
\end{reglemlista}

%%% funktionar %%%
\funktionar{Valberedningens ordförande}
Valberedningens ordförande leder valberedningens arbete.

\begin{reglemlista}

	\item[Åligganden]
	Det åligger Valberedningens ordförande
	\begin{attlista}
		\item leda valberedningens arbete
		\item inte diskutera intervjuerna och detaljer om kandidaterna förutom med den berörda kandidaten
	\end{attlista}

	\item[Mandatperiod]
	Valberedningsordförandens mandatperiod är 1~juli -- 30~juni.

\end{reglemlista}

%%% funktionar %%%
\funktionar{Valberedningsrepresentant}
Valberedningsrepresentanten verkar för att bredda sektionens
rekryteringsbas.

\begin{reglemlista}

	\item[Åligganden]
	Det åligger Valberedningsrepresentanten
	\begin{attlista}
		\item bredda sektionens rekryteringsbas
		\item inte diskutera intervjuerna och detaljer om kandidaterna förutom med den berörda kandidaten
	\end{attlista}

	\item[Mandatperiod]
	Valberedningsrepresentantens mandatperiod är 1~juli -- 30~juni.

\end{reglemlista}

%%% funktionar %%%
\funktionar{Valnämndsrepresentant TLTH}
Valnämndsrepresentanten TLTH är sektionens representant i TLTHs valnämnd.

\begin{reglemlista}

	\item[Åligganden]
	Det åligger Valnämndsrepresentanten TLTH
	\begin{attlista}
		\item representera sektionen i TLTHs valnämnd
	\end{attlista}

\end{reglemlista}

%%% funktionar %%%
\funktionar{Världsmästare}
Världsmästaren verkar för sektionens internationalisering.

\begin{reglemlista}

	\item[Åligganden]
	Det åligger Världsmästaren
	\begin{attlista}
		\item representera sektionen i TLTHs internationella utskott
		\item verka för sektionens internationalisering
		\item ge internationella studenter den information de behöver
		\item öka medvetenhet om internationella studenter på sektionen
	\end{attlista}

	\item[Val]
	Världsmästaren väljs av Studierådet.
\end{reglemlista}

%%% funktionar %%%
\funktionar{Årskursrepresentant för de tre lägre årskurserna}
Årskursrepresentanten är den som utvärderar våra kurser och framför sina
kursarers åsikter.

\begin{reglemlista}

	\item[Åligganden]
	Det åligger Årskursrepresentanten
	\begin{attlista}
		\item agera representant för sina kursare inför kursansvarig och framföra deras åsikter som kursombud
		\item delta på årskursens CEQ-möten
		\item skriva slutkommentarer för kursen
	\end{attlista}
	\item[Mandatperiod]
	Årskursrepresentants mandatperiod är 1~juli -- 30~juni.

	\item[Val]
	Årskursrepresentant väljs av Studierådet.
\end{reglemlista}


%%% funktionar %%%
\funktionar{Ölförman}
Ölförmannen är i samråd med pub- och barmästare ansvarig för sexmästeriets inköp av bryggerivaror.

\begin{reglemlista}

	\item[Åligganden]
	Det åligger Ölförmannen
	\begin{attlista}
		\item i samråd med pubmästaren och barmästaren se till så att sexmästeriet har nödvändiga drycker exklusive sittningsdryck i god tid före varje tillställning
	\end{attlista}

\end{reglemlista}

%%% funktionar %%%
\funktionar{Övermarskalk}
Övermarskalken bär sektionens fana samt sköter inbjudningar vid större sektionsfester.

\begin{reglemlista}

	\item[Åligganden]
	Det åligger Övermarskalken
	\begin{attlista}
		\item vid högtider såsom Regatta och Kårbal bära sektionens fana
		\item handha medaljer och av sektionen fastställda utmärkelser samt utdelandet av dessa
		\item övervaka Ordenssällskapet D-sektionen inom TLTH och tillse att det verkar enligt gällande reglemente och stadgar
		\item handha inbjudningar och anmodningar vid sektionens festligheter
		\item ansvara för inhandling och förvaltning av ordensband
		\item ha hand om sektionsflaggan
	\end{attlista}

\end{reglemlista}

%%% funktionar %%%
\funktionar{Semesterfirare}

\begin{reglemlista}

	\item[Åligganden]
	Det åligger Semesterfiraren
	\begin{attlista}
		\item se till att sektionens medlemmar får åka på utomlandssemester till andra universitet i Sverige
	\end{attlista}

\end{reglemlista}

%%% funktionar %%%
\funktionar{\O verpeppare}
\O verpepparna agerar Nollningsutskottets högra hand genom att vara med i planering
och genomförande av nollningsrelaterade evenemang. \O verpepparna ansvarar för att leda Pepparna.
\begin{reglemlista}
	\item[Åligganden]
	Det åligger \O verpeppare
	\begin{attlista}
		\item hjälpa Nollningsutskottet med att planera och genomföra nollningsrelaterade evenemang
		\item leda Papparna
	\end{attlista}

\end{reglemlista}

%%% funktionar %%%
\funktionar{Peppare}
Pepparna agerar Nollningsutskottets högra hand genom att vara med i planering
och genomförande av nollningsrelaterade evenemang.
\begin{reglemlista}
	\item[Åligganden]
	Det åligger Pepparna
	\begin{attlista}
		\item hjälpa Nollningsutskottet och \O verpepparna med att planera och genomföra nollningsrelaterade evenemang
	\end{attlista}

\end{reglemlista}

%%% funktionar %%%
\funktionar{Framtidsordförande}
Framtidsordförande ansvarar för att leda Framtidsutskottet.
\begin{reglemlista}
	\item[Åligganden]
	Det åligger Framtidsordförande
	\begin{attlista}
		\item kalla till möten för Framtidsutskottet och upprätta erforderliga handlingar
		\item leda och fördela arbetet inom utskottet
		\item samverka med styrelsen för att gynna sektionens strategiska arbete
	\end{attlista}

\end{reglemlista}

\funktionar{Framtidsledamot}
Framtidsledamoten sitter i Framtidsutskottet
\begin{reglemlista}
	\item[Åligganden]
	Det åligger Framtidsledamöter
	\begin{attlista}
		\item arbeta med sektionens strategiska arbete
		\item bereda förslag för att presentera till styrelsen eller sektionsmötet
		\item bistå Framtidsordförannde i dess arbete
	\end{attlista}

\end{reglemlista}

%%%%%%%%%%%%% SECTION %%%%%%%%%%%%%
\section{Medaljer}

I detta avsnitt listas sektionens medaljer samt de kriterier som ligger bakom förvärvandet av dessa.

Medaljer delas företrädesvis ut två gånger om året. Under det årliga ordenskapitlet i samband med Shiphtesgasquen i början av det nya kalenderåret för de som gick av vid årskiftet, alternativt sommarphesten för de som går av under sommaren. Vid speciella tillfällen, såsom t.ex. Jubileum, kan ett extra ordenskapitel hållas.

\subsection{Funktionärsmedaljen}
Funktionärsmedaljen erhålles av funktionärer efter avslutad mandatperiod. Denna medalj bäres med stolthet. Funktionärsmedaljen utdelas till en och samma person endast en gång.

\subsection{Stor D-medalj}
Stor D-medalj utdelas av Medaljelelekommittén på årligt ordenskapitel, till person som Medaljelelekommittén har funnit dels under en ansenlig mängd år innehaft avancerade förtroendeuppdrag inom D-sektionen, dels verkat på det mest föredömliga sätt samt åstadkommit exceptionella fördelar för D-sektionen. Stor D-medalj utdelas till en och samma person endast en gång.

\subsection{D-medalj}
D-medalj utdelas av Medaljelelekommittén på årligt ordenskapitel, till person som Medaljelelekommittén har funnit dels under flera år innehaft förtroendeuppdrag inom D-sektionen, dels på olika sätt verkat mycket föredömligt för D-sektionen. I folkmun kallat för ''riddare''. D-medalj utdelas till en och samma person endast en gång.

\subsection{Inspektor Emeritussignum}
Inspektor Emeritussignum utdelas av Medaljelelekommittén till D-sektionens avgående Inspektor vid installation av ny Inspektor för D-sektionen. Installationen sker under årligt ordenskapitel, eller vid annat av Medaljelelekommittén och D-sektionens sektionsmöte valt tillfälle. Inspektor Emeritussignum utdelas till en och samma person endast en gång.

\subsection{Hedersledamotsignum}
Hedersledamotsignum utdelas av Medaljelelekommittén till av D-sektionens sektionsmöte vald hedersledamot av D-sektionen vid installation av denne. Installationen sker under årligt ordenskapitel, eller vid annat av Medaljelelekommittén och D-sektionens sektionsmöte valt tillfälle. Hedersledamotsignum utdelas till en och samma person endast en gång.

\subsection{Medaljelelekommittémedaljen}
Medaljelelekommittémedaljen bäres av de som är medlemmar i Medaljelelekommittéen.

\subsection{DuCtig-medaljen}
DuCtig-medaljen erhålles som en del av studierådets utmärkelse DuCtig (Data und infocoms tentaprestationer i grundblocket). Medaljen finns i tre valörer; brons, silver och guld. Medaljen delas ut enligt studierådets kriterier.

\subsection{Insignier}
Insignier är små plaketter som fästes på erhållen funktionärsmedalj.

\begin{reglemlista}
	\item[Styrelseinsigniner]
	Styrelseinsigniner tilldelas de personer som innehaft funktionärspost inom sektionsstyrelsen. En styrelseinsigna delas ut för varje fullgjord mandatperiod på posten.
	\item[Gammal och äcklig]
	Utdelas till de som uppfyller kraven på Gammal och äcklig: nämligen att ha varit vald funktionär i tre hela år, samt innehaft ett mandat i sektionsstyrelsen eller att ha varit vald funktionär i fyra hela år
	\item[TLTH-Fullmäktige]
	Tilldelas den som valts in som ordinarie ledamot i Teknologkårens fullmäktige
\end{reglemlista}

%%%%%%%%%%%%% SECTION %%%%%%%%%%%%%
\section{Tolkningar}

I denna paragraf får endast tolkningar beträffande reglementet införas.

\begin{reglemlista}

	\item[Sektionens likviditet]
	Sektionens grundlikviditet bör vara 500 000 kr. Tillfälliga avvikelser är godtagbara i samband med större arrangemang såsom t.ex. nollning.

\end{reglemlista}

\end{document}


