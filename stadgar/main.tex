\documentclass[stadgar]{dsekprotokoll}
\usepackage{enumerate}
\ifpdf
  \usepackage{thumbpdf}
\fi

\usepackage[utf8]{inputenc}

% Puts a full stop after the section number
\makeatletter
\def\@seccntformat#1{\csname the#1\endcsname.\quad}
\makeatother

\setheader{STADGAR}{VTM 2019}{\today}

\begin{document}

\title{Stadgar för D-sektionen inom TLTH}
\date{\today}

\maketitle

\section{Sektionen}

\begin{stadgeavsnitt}

\paragraf{1.1}{Namn}

Datatekniksektionen inom Teknologkåren vid Lunds Tekniska Högskola.

\paragraf{1.2}{Ändamål}

Föreningens ändamål och syfte är att verka för sammanhållningen mellan sina
medlemmar, att främja deras studier och utbildning, att tillvarata deras
gemensamma intressen, samt vad därmed äger sammanhang. Sektionen drivs utan
vinstintresse.

\paragraf{1.3}{Symbol}

Sektionens officiella symbol ser ut enligt nedanstående bild.

\tthdump{\Dsymbol[10mm]}
%%tth:\special{html:<img src="/ikoner/D-symbol.gif" alt="[D-sektionens D]">}

\paragraf{1.4}{Färg}

Sektionens färg är Råsa.

\paragraf{1.5}{Maskot}

Sektionens maskot är Rosa Pantern.

\paragraf{1.6}{Blomma}

Sektionens blomma är Råsa Digitalis.

\paragraf{1.7}{Hymn}

Sektionshymnen är ''Rosa på bal'' av E. Taube.

\paragraf{1.8}{Överklagande}

Beslut taget av någon av sektionens myndigheter kan hos kårens fullmäktige
överklagas av minst en tiondel, eller etthundra av sektionens medlemmar
inom tre veckor från den dag beslutet tillkännagetts.

\paragraf{1.9}{Undanröjande}

Beslut taget av någon av sektionens myndigheter kan av kårens fullmäktige
undanröjas endast om det uppenbart strider mot ändamålsparagrafen \S1.2
eller mot ändamålsparagrafen i kårens stadgar.

\paragraf{1.10}{Stadfästelse}

Stadfästelse innebär att en högre instans bekräftar ett beslut taget av en instans underordnad den högre instansen. Ett beslut som behöver stadfästas gäller från den tidpunkt beslutet togs, men kan rivas upp, ändras eller godkännas vid tiden för stadfästelse.

\end{stadgeavsnitt}

\section{Ordinarie medlemmar}

\begin{stadgeavsnitt}

\paragraf{2.1}{Definition}

 Ordinarie medlem är var person som är medlem i Teknologkåren vid LTH samt antingen läser vid
ett av följande program:
\begin{itemize}
\item Civilingenjörsutbildningen i Datateknik vid LTH
\item Civilingenjörsutbildningen i Informations- och kommunikationsteknik vid LTH
\item Student av kårstyrelsen förd till sektionen.
\end{itemize}

\paragraf{2.2}{Skyldigheter}

Medlem är skyldig:
\begin{attlista}
\item vara ordinarie medlem vid TLTH
\item iaktta kårens och sektionens stadgar och reglemente.
\end{attlista}

\paragraf{2.3}{Rättigheter}

Medlem som fullgjort sina skyldigheter enligt \S2.2 äger rätt att:
\begin{attlista}
\item med en röst deltaga i de val av funktionärer och representanter som
  företräder sektionen
\item närvara med yttranderätt, yrkanderätt och rösträtt vid sektionsmöte
\item taga del av protokoll och övriga handlingar som berör sektionen, samt
\item kandidera vid allmänna val inom sektion och kår.
\end{attlista}

\end{stadgeavsnitt}

\section{Hedersmedlemmar}

\begin{stadgeavsnitt}

\paragraf{3.1}{Definition}

Person som gjort betydande insatser för sektionens verksamhet, existens
eller fortlevnad.

\paragraf{3.2}{Val}

Hedersmedlem kan utses av sektionsmöte med minst nio tiondelar av samtliga
avgivna röster.

\paragraf{3.3}{Skyldigheter}

Hedersmedlem är skyldig:
\begin{attlista}
\item iaktta kårens och sektionens stadgar och reglemente.
\end{attlista}

\paragraf{3.4}{Rättigheter}

Hedersmedlem äger rätt:
\begin{attlista}
\item närvara med yttranderätt och yrkanderätt vid sektionsmöte,
\item taga del av protokoll och övriga handlingar som berör sektionen, samt
\item kandidera vid funktionärsval inom sektionen undantaget poster som innebär styrelsepost eller chef för utskott.
\end{attlista}

\end{stadgeavsnitt}

\section{Virtuella medlemmar}

\begin{stadgeavsnitt}

\paragraf{4.1}{Definition}

Person som av praktiska och ideologiska skäl önskar tillhöra sektionen samt
med själ och hjärta är villig att prisa densamma.

\paragraf{4.2}{Val}

Virtuell medlem kan utses av sektionsmöte med kvalificerad majoritet.

\paragraf{4.3}{Skyldigheter}

Virtuell medlem är skyldig:
\begin{attlista}
\item iaktta kårens och sektionens stadgar och reglemente.
\end{attlista}

\paragraf{4.4}{Rättigheter}

Virtuell medlem äger rätt:
\begin{attlista}
\item närvara med yttranderätt och yrkanderätt vid sektionsmöte,
\item taga del av protokoll och övriga handlingar som berör sektionen, samt
\item kandidera vid funktionärsval inom sektionen undantaget poster som
  innebär styrelsepost eller chef för utskott.
\end{attlista}

\end{stadgeavsnitt}

\section{Alumnimedlemmar}

\begin{stadgeavsnitt}

\paragraf{5.1}{Definition}
 Person som tidigare varit ordinarie medlem på sektionen samt är stödmedlem på Tekonologkåren vid TLTH.

\paragraf{5.2}{Skyldigheter}

Alumnimedlem är skyldig:
\begin{attlista}
\item iaktta kårens och sektionens stadgar och reglemente.
\item att vara stödmedlem vid TLTH.
\end{attlista}

\paragraf{5.3}{Rättigheter}

Alumnimedlem äger rätt:
\begin{attlista}
\item närvara med yttranderätt och yrkanderätt vid sektionsmöte,
\item taga del av protokoll och övriga handlingar som berör sektionen, samt
\item kandidera vid funktionärsval inom sektionen undantaget poster som
  innebär styrelsepost eller chef för utskott.
\end{attlista}

\paragraf{5.4}{Utträde}
 Alumnimedlemmar kan utträda ur sektionen genom egen förfrågan varpå sektionsmöte må besluta så. Sektionsmötet må även besluta att entlediga Alumnimedlemmar om denne inte uppfyller de skyldigheter som åligger posten.

\end{stadgeavsnitt}

\section{Organisation}

\begin{stadgeavsnitt}

\paragraf[1]{6.1}{Högsta beslutande organ}

Sektionsmötet är sektionens högsta beslutande organ.

\paragraf{6.2}{Organisation}

Sektionens verksamhet utövas på det sätt denna stadga föreskriver genom:
\begin{enumerate}[a)]
\item Styrelsen,
\item Studierådet,
\item Utskott,
\item Valberedningen,
\item Revisorer,
\item Inspektor, samt
\item Fristående föreningar.
\end{enumerate}

\end{stadgeavsnitt}

\section{Verksamhetsår}

\begin{stadgeavsnitt}

\paragraf{7.1}{Verksamhetsår}

Sektionens verksamhets- och räkenskapsår omfattar tiden 1~januari till
31~december.

\end{stadgeavsnitt}

\section{Sektionsmöte}

\begin{stadgeavsnitt}

\paragraf{8.1}{Befogenhet}

Sektionsmötet är sektionens högsta beslutande organ.

\paragraf{8.2}{Rösträtt}

Rösträtt tillkommer sektionens ordinarie medlemmar.

\paragraf{8.3}{Beslutsförhet}

Sektionsmötet äger rätt att fatta beslut då antalet närvarande
röstberättigade medlemmar är minst tjugotre.

\paragraf{8.4}{Adjungeringar}

Ständigt adjungerande med yttranderätt och yrkanderätt vid sektionsmöten
är:
\begin{enumerate}[a)]
\item inspektorn,
\item revisorerna, samt
\item kårstyrelsens ledamöter.
\end{enumerate}

\paragraf{8.5}{Förhandlingsledare}

Sektionsmötets förhandlingar leds av talmannen eller av mötet tillförordnad
mötesordförande.

\paragraf{8.6}{Utlysande}

Styrelsen kallar till sektionsmöte.

Kallelse till ordinarie sektionsmöte skall anslås på sektionens
anslagstavla senast femton läsdagar före mötet.

Kallelse till extra sektionsmöte skall anslås på sektionens anslagstavla
senast fem läsdagar före mötet.

\paragraf{8.7}{Handlingar}

Föredragningslista samt erforderliga handlingar skall anslås på
anslagstavlan senast fem läsdagar före mötet.

\paragraf{8.8}{Upptagande av ärende}

Varje medlem äger rätt att upptaga ärende till behandling
på ordinarie sektionsmöte, undantaget valmöte. Sådan motion skall skriftligen ha inkommit till styrelsen, digitalt eller i brevlådan utanför RåsenbaD, minst tio läsdagar innan sammanträdet

\paragraf{8.8.2}{Interpellation}
Varje medlem som önskar erhålla förklaring av funktionär angående
dennes verksamhet äger rätt att ställa interpellation till denne. Om
så begärs av den till vilken interpellation är riktad, skall den avges
i skriftlig form. Sådan interpellation skall skriftligen ha inkommit
till styrelsen samt ha tillställts den som interpellationen är riktad
till senast tre läsdagar före sammanträdet. Svar på
interpellationen skall avges på sammanträdet. 

\paragraf{8.8.3}{Misstroendevotum}

Varje medlem äger rätt att ställa misstroendevotum mot
sektionsfunktionär. Sådan misstroendevotum ska skriftligen ha inkommit
till styrelsen senast tio läsdagar före
sammanträdet. Om sektionsmötet bifaller yrkandet om misstroende
entledigas funtionären från sitt uppdrag.

\paragraf{8.9}{Sen handling}
En sen handling definieras som en handling, vilken har inkommit efter utsatt sista datum enligt
\S8.8, eller annan handling, exempelvis bilaga, som inte anslagits i tid enligt \S8.7. Om sådan
handling har inkommit innan sista datum för anslag enligt \S8.7 kan styrelsen välja att anslå och
ta med den i föredragningslistan under prefixet Sen handling. Mötet beslutar sedan med enkel
majoritet huruvida punkten skall behandlas. I den händelse att handlingen inte kan anslås i
tid enligt \S8.7 kan styrelsen istället välja att låta mötet avgöra huruvida den skall införas på föredragningslistan
när denna fastställs. Alla sena handlingar som ej lyfts enligt ovan bordläggs
till nästföljande sektionsmöte, då de behandlas som korrekt inkomna handlingar.

\paragraf{8.10}{Vårterminsmöte}

Vid vårterminsmötet skall följande ärenden förekomma:
\begin{enumerate}[a)]
\item verksamhetsberättelse och bokslut från föregående års styrelse,
\item revisionsberättelse för det förflutna verksamhetsåret,
\item fråga om ansvarsfrihet för föregående års styrelse, 
\item Verksamhetsrapport från styrelsen, samt
\item val enligt reglemente.
\end{enumerate}

\paragraf{8.11}{Höstterminsmöte ett}

Det första höstterminsmötet skall hållas i första läsperioden efter sommaren. Vid detta möte skall följande ärenden förekomma:
\begin{enumerate}[a)]
\item halvårsbokslut, 
\item Verksamhetsrapport från styrelsen, samt
\item val enligt reglemente.
\end{enumerate}

\paragraf{8.12}{Valmöte}

Valmötet skall hållas innan höstterminsmöte två. Vid detta möte skall följande ärenden tas upp:
\begin{enumerate}[a)]
\item val enligt regelmente. 
\end{enumerate}

\paragraf{8.13}{Höstterminsmöte två}

Vid andra höstterminsmötet skall följande ärenden tas upp:
\begin{enumerate}[a)]
\item information om sektionens ekonomiska status,
\item budget för nästkommande verksamhetsår, 
\item Verksamhetsrapport från styrelsen, samt
\item verksamhetsplan för nästkommande verksamhetsår
\end{enumerate}

\paragraf{8.14}{Extra sektionsmöte}

Extra sektionsmöte skall hållas
\begin{fetlista}{då}
\item sektionens styrelse beslutar så,
\item revisorerna hos styrelsen därom skriftligt anhåller, med
  uppgivande av vilket ärende som önskas behandlas, eller
\item minst tjugofem medlemmar hos styrelsen därom skriftligen
  anhåller, med uppgivande av vilket ärende som önskas behandlas.
\end{fetlista}

Dylikt möte skall hållas inom tio läsdagar efter det att
anhållan inkommit till styrelsen.


\end{stadgeavsnitt}

\section{Valberedning}

\begin{stadgeavsnitt}

\paragraf{9.1}{Uppgift}

Valberedningen har som uppgift att förbereda val som skall förrättas av
sektionsmötet.

\paragraf{9.2}{Sammansättning}

Sektionens valberedning består av ordförande samt minst en representant
från var och en av de fyra senaste årskurserna.

\paragraf{9.3}{Åligganden}

Det åligger valberedningen att senast fem läsdagar innan
sektionsmötet, skriftligen anslå sitt förslag på sektionens anslagstavla.

\end{stadgeavsnitt}

\section{Val}

\begin{stadgeavsnitt}

\paragraf{10.1}{Valbarhet}

Valbarhet till funktionärspost regleras av var sektionsmedlems
medlemstyp. Valbarhet till inspektor regleras av \S15.2. Valbarhet till revisor är öppet för
allmänheten.

\paragraf{10.2}{Val}

Sektionsmötets val av funktionärer förbereds av valberedningen. Utöver
valberedningens förslag må fri kandidatnominering ske intill tiden då talmannen fastslagit valets kandidater. 

Talmannen fastslår kandidaterna enligt följande: 
\begin{enumerate}
\item Talmannen frågar valberedningen om deras kandidater.
\item Var kandidat som de nominerat frågas om de godtar nomineringen.
\item Därefter frågas mötet om fler kandidater finns, alla som önskar kandidera eller nominera
ges då möjlighet att göra så.
\item Efter att kandidaturerna godkänns, antingen direkt eller via ett ombud (om man exempelvis
är på utlandsstudier och inte kan närvara), så fastslår talmannen vem som är kandidater
till valet. I och med det så stängs möjligheten att kandidera.
\end{enumerate}

\paragraf{10.3}{Fyllnadsval}

Styrelsemedlemmar, revisorer, inspektor, valberedning, vice skattmästare och Framtidsordförande kan endast väljas av sektionsmöte.
Representanter i universitetets organ kan endast nomineras av sektionsmöte
eller studierådet. Fyllnadsval av övriga funktionärer kan förrättas av styrelsen.



\paragraf{10.4}{Entledigande}

Funktionär kan bli entledigad av den instans som har valt denne. Sektionsmötet äger rätt
att med kvalificerad majoritet entlediga alla funktionärer.

\end{stadgeavsnitt}

\section{Procedurregler}

\begin{stadgeavsnitt}

\paragraf{11.1}{Röstning}

Röstning via fullmakt är ej tillåten.

\paragraf{11.2}{Jävighet}

Ingen må deltaga i beslut om ansvarsfrihet för åtgärd för vilken han/hon är
ansvarig, eller i beslut av vars utgång han/hon kan äga ett ekonomiskt
intresse.

\paragraf{11.3}{Acklamation}

Beslut fattas genom acklamation, såvida votering ej begärts.

\paragraf{11.4}{Votering}

Votering skall vara sluten vid personval. I alla övriga fall skall öppen
votering hållas. Om så begärs skall vid öppen votering föras röstprotokoll.
Sluten votering sker efter namnupprop. Om röstetalet är lika äger
mötesordförande utslagsröst vid öppen votering, vid sluten votering avgör
lotten.

\paragraf{11.5}{Adjungerade}

Adjungerade må deltaga i sammanträde med yttranderätt och yrkanderätt.

\paragraf{11.6}{Kvalificerad majoritet}

Med kvalificerad majoritet menas, om ej annat stadgats, minst två
tredjedelar av de avgivna rösterna.

\paragraf{11.7}{Övrigt}

Om ej annat stadgats i stadgan eller reglementet gäller den förhandlingsord- ning som är reglerad i Teknologkårens stadgar, med lämplig anpassning för sektionens organ.

\end{stadgeavsnitt}

\section{Protokoll}

\begin{stadgeavsnitt}

\paragraf{12.1}{Sektionsmöte}

Vid sektionsmöte skall föras protokoll, som upptar anteckningar om ärendets
art, samtliga ställda och ej återtagna yrkanden, beslut, särskilda
yttranden och reservationer samt förteckning över de närvarande.

\paragraf{12.2}{Övriga myndigheter}

Vid övriga beslutande myndigheters sammanträden förs protokoll, som upptar
tagna beslut och förteckning över de närvarande.

\paragraf{12.3}{Justering}

Sektionsmötesprotokoll justeras inom tio läsdagar av mötesordföranden samt
två, på mötet valda justeringsmän.

Övriga myndigheters protokoll justeras snarast av mötesordförande eller av mötet vald justeringsperson.

\paragraf{12.4}{Offentliggörande}

Justerade protokoll skall anslås på sektionens anslagstavla.

\paragraf{12.5}{Arkivering}

Samtliga möteshandlingar skall arkiveras.

\end{stadgeavsnitt}

\section{Styrelsen}

\begin{stadgeavsnitt}

\paragraf{13.1}{Befogenhet}

Sektionsstyrelsen är sektionens högsta verkställande organ.

\paragraf{13.2}{Sammansättning}

Sektionsstyrelsen utgörs av Ordförande, Vice Ordförande, Skattmästare, Informationsansvarig, Studierådsordförande, Cafémästare, Näringslivsansvarig, Källarmästare, Aktivitetsansvarig, Sexmästare samt Överphös.

\paragraf{13.3}{Beslut}

Sektionsstyrelsen är beslutsför om fler än hälften av sektionsstyrelsens
ledamöter är närvarande.

\paragraf{13.4}{Sammanträde}

Sektionsstyrelsen sammanträder på kallelse av ordföranden eller vid dennes
frånvaro av vice ordföranden. Rätt att begära utlysning av styrelsemöte
har
\begin{enumerate}[a)]
\item styrelseledamot,
\item revisor, samt
\item inspektor.
\end{enumerate}

\paragraf{13.5}{Kallelse}

Kallelse till styrelsemöte samt föredragningslista skall senast tre läsdagar före sammanträdet tillställas styrelsens ledamöter,
revisorerna, inspektorn samt anslås på sektionens anslagstavla.

\paragraf{13.6}{Solidaritet}

Styrelseledamot som utan reservation deltagit i beslut som fattats i
sektionsstyrelsen är solidariskt ansvarig för detta. Styrelseledamot som ej
varit närvarande vid beslut är solidariskt ansvarig om han/hon inte
reserverat sig i protokollet senast vid nästa sammanträde då han/hon varit
närvarande.

\end{stadgeavsnitt}

\section{Utskott}

\begin{stadgeavsnitt}

\paragraf{14.1}{Definition}

Sektionens utskott är:
\begin{enumerate}[a)]
\item Informationsutskottet,
\item Skattmästeriet,
\item Studierådet,
\item Cafémästeriet,
\item Näringslivsutskottet,
\item Källarmästeriet,
\item Aktivitetsutskottet,
\item Sexmästeriet,
\item Nollningsutskottet,
\item Medaljelelekommittén
\item Trivselrådet, samt
\item Framtidsutskottet
\end{enumerate}

\paragraf{14.2}{Åligganden}

Det åligger sektionens utskott att följa sektionens stadgar och
reglemente.

\paragraf{14.3}{Informationsutskottet}

Informaitonsutskottet har till uppgift att tillgodose sektionsmedlemmarnas behov av information och fotografering, underhåll och handhavande av sektionens arkiv samt vad därmed äger sammanhang.

\paragraf{14.4}{Skattmästeriet}

Skattmästeriet har till uppgift att handha sektionens ekonomi och bokföring
samt vad därmed äger sammanhang.

\paragraf{14.5}{Studierådet}

Studierådet vid Datatekniksektionen är sektionens högsta beslutande organ
för studiebevakning samt vad därmed äger sammanhang. Undantaget är frågor
som berörs av sektionsmötet.

\paragraf{14.6}{Cafémästeriet}

Cafémästeriet har till uppgift att till studentvänliga priser tillgodose
sektionsmedlemmarnas behov av cafévaror samt vad därmed äger sammanhang.
\paragraf{14.7}{Näringslivsutskottet}

Näringslivsutskottet har till uppgift att tillgodose sektionsmedlemmarnas
intressen vad gäller kontakter med näringslivet samt vad därmed äger
sammanhang.

\paragraf{14.8}{Källarmästeriet}

Källarmästeriet har till uppgift att tillgodose sektionsmedlemmarnas intressen i frågor om underhåll av sektionens lokaler, sektionens datorer samt vad därmed äger sammanhang.

\paragraf{14.9}{Aktivitetsutskottet}

Aktivitetsutskottet har till uppgift att tillgodose sektionsmedlemmarnas behov
av nöjen och förströelser samt vad därmed äger sammanhang.

\paragraf{14.10}{Nollningsutskottet}

Nollningsutskottet har till uppgift att genomföra nollning samt vad därmed äger sammanhang.

\paragraf{14.11}{ Sexmästeriet}

Sexmästeriet har till uppgift att tillgodose sektionsmedlemmarnas intressen
vad gäller festaktivitet samt vad därmed äger sammanhang.

\paragraf{14.12}{ Medaljelelekommittén}

Medaljelelekommittén har till uppgift att dela ut medaljer till
sektionsfunktionärer.

\paragraf{14.13}{ Trivselrådet}

Trivselrådet har till uppgift att organisera och se över sektionens likabehandlings-, internationaliserings- samt arbetsmiljöarbete.
\paragraf{14.14}{Framtidsutskottet}
Framtidsutskottet har som uppgift att stödja sektionen i dess strategiska arbete, bereda förslag för styrelsen och sektionsmötet samt jobba med långsiktiga projekt.

\paragraf{14.15}{ Alternativ titulering}

Alternativ titulering för utskott är mästeri.

\end{stadgeavsnitt}

\section{Inspektor}

\begin{stadgeavsnitt}

\paragraf{15.1}{Uppgift}

Inspektorn skall ägna uppmärksamhet åt och stödja sektionens verksamhet
samt vad därmed äger sammanhang.

\paragraf{15.2}{Val}
%%inte säker, kolla motionen igen sen o veta vilket framvaskat förslag det gäller
Inspektorn väljs, med kvalificerad majoritet, av sektionsmöte för en tid på två år. Inspektorn ska väljas bland de anställda vid Lunds Tekniska Högskola som uppvisar vilja att främja Datatek- niksektionens verksamhet samt dess medlemmars studietid.

\paragraf{15.3}{Åligganden}

Det åligger inspektorn att hålla sig informerad om sektionens verksamhet.

\paragraf{15.4}{Rättigheter}

Inspektorn äger rätt:
\begin{attlista}
\item närvara vid samtliga sektionens myndigheters sammanträden med
  yttranderätt och yrkanderätt,
\item erhålla samtliga sektionens handlingar rörande sektionens
  myndigheters sammanträden,
\item taga del av sektionens protokoll och övriga handlingar, samt i övrigt
  bli informerad om sektionens verksamhet, samt
\item erhålla av sektionen utgivna publikationer.
\end{attlista}

\end{stadgeavsnitt}

\section{Funktionärer}

\begin{stadgeavsnitt}

\paragraf{16.1}{Definition}

Sektionens funktionärer är:
\begin{fetlista}{av}
\item sektionsmötet utsedda till förtroendeposter, samt
\item sektionsstyrelsen utsedda till förtroendeposter
\end{fetlista}

\paragraf{16.2}{Skyldigheter}

Sektionens funktionärer är skyldiga att följa gällande stadgar och
reglemente, samt att verka för sektionens bästa.

\paragraf{16.3}{Mandatperiod}

Mandatperioden för sektionens funktionärer är kalenderår då ej annat
stadgats i reglementet.

\end{stadgeavsnitt}

\section{Ekonomi}

\begin{stadgeavsnitt}

\paragraf{17.1}{Skötsel}

Sektionens ekonomi sköts av sektionens skattmästare. Varje utskott är dock
ansvarig för sin verksamhet.

\paragraf{17.2}{Redovisning}

Sektionsstyrelsen ansvarar för att sektionens redovisning sköts i enlighet
med myndigheters krav och god redovisningssed.

\paragraf{17.3}{Firma}

Sektionens firma tecknas av styrelsen som helhet samt skattmästaren och ordföranden
var för sig.

\end{stadgeavsnitt}

\section{Stadgan}

\begin{stadgeavsnitt}

\paragraf{18.1}{Tolkning}

Vid tolkning av stadgan gäller inspektorns tolkning intill dess
sektionsmötet beslutat i saken. Vid avsaknad av inspektor gäller
sektionsordförandens tolkning intill dess sektionsmötet beslutat i saken.
Tolkningar skall protokollföras och redovisas på nästkommande
sektionsmöte.

\paragraf{18.2}{Ändring}

För ändring av stadgan krävs likalydande beslut med kvalificerad majoritet
på två sektionsmöten med minst en månad och högst tretton månader emellan.
\\\\
Alternativ till stadgeändringar kan endast inlämnas i förväg som ändringsyrkande, eller som svar av styrelsen. Sådana ändringsyrkande ska vara talman tillhanda minst en läsdag före sektionsmötet. Smärre ändringar av förslag vad gäller grammatik och dylikt kan göras av mötet

\paragraf{18.3}{Fastställande}

Ändring av paragraf träder ej i kraft förrän de stadfästs på vederbörligt
sätt i enlighet med kårens stadgar.

\paragraf{18.4}{Giltighet}

Skulle någon av dessa stadgar strida mot kårens stadgar, gäller i sådana
fall kårens stadgar.

\end{stadgeavsnitt}

\section{Reglemente}

\begin{stadgeavsnitt}

\paragraf{19.1}{Definition}

Reglementet är ett tillägg till stadgarna, i vilket
tillämpningsföreskrifter och övriga föreskrifter återfinns.

\paragraf{19.2}{Tolkning}

Vid tolkning av reglementet gäller inspektorns tolkning intill dess
sektionsmötet beslutat i saken. Vid avsaknad av inspektor gäller
sektionsordförandens tolkning intill dess sektionsmötet beslutat i saken.
Tolkningar skall protokollföras och redovisas på nästkommande
sektionsmöte.

\paragraf{19.3}{Ändring}

För ändring av reglementet krävs likalydande beslut på två sektionsmöten
med minst en månad och högst tretton månader emellan, eller beslut med
kvalificerad majoritet på ett sektionsmöte.

\end{stadgeavsnitt}

\section{Fristående föreningar}

\begin{stadgeavsnitt}

\paragraf{20.1}{Definition}

För att betraktas som en fristående förening skall föreningen vara godkänd
av sektionsmötet och ha av sektionsmötet stadfästa stadgar.

\paragraf{20.2}{Bildande}

Fristående föreningar kan bildas bland sektionens medlemmar.

\paragraf{20.3}{Stadgar}

Stadgar skall innehålla namn och bestämmelser om syfte, medlemskap,
ekonomi, föreningsmöte och föreningsstyrelse.

\paragraf{20.4}{Syfte och verksamhet}

En fristående förenings syfte och verksamhet får inte gå i konflikt med
sektionens dito.

\paragraf{20.5}{Rättigheter}

En fristående förening inom sektionen må av styrelsen beviljas rätt att
utnyttja vissa av sektionslokalerna för sin verksamhet.

\paragraf{20.6}{Skyldigheter}

En fristående förening är skyldig att, följa sektionen och kårens styrdokument. En fristående förening är skyldig att, erbjuda Datatekniksektionens alla medlemmar möjlighet att vara en del av föreningen och verka för föreningens mål och syften.

\paragraf{20.7}{Förlust- och underskottsgaranti}

Sektionen kan lämna förlust- och underskottsgaranti upp till högst hälften
av förlusten/underskottet till föreningen.

\paragraf{20.8}{Revisorer}

Om sektionen givit någon form av ekonomiska garantier till föreningen skall
en av föreningens revisorer utses av sektionen.

\paragraf{20.9}{Sammansättning}

Den fristående föreningens medlemmar skall till övervägande del utgöras av
sektionsmedlemmar.

\paragraf{20.10}{Fristående föreningar under Datatekniksektionen}
D-chip \\

\end{stadgeavsnitt}

\section{Ordenssällskapet}

\begin{stadgeavsnitt}

\paragraf[2]{21.1}{Ordenssällskapet Datatekniksektionen vid Lunds Tekniska
  Högskola, ODLTH}

ODLTH är organisatoriskt underställt övermarskalken. ODLTH lyder under, av
sektionen, för ODLTH fastslagna stadgar.

\end{stadgeavsnitt}

\section{Upplösning}

\begin{stadgeavsnitt}

\paragraf{22.1}{Upplösning}

Datatekniksektionen kan ej upplösas.

\paragraf{22.2}{Tillgångar}

Vid upplösning av sektionen tillfaller tillgångarna Teknologkåren vid Lunds
Tekniska Högskola, vilken har att förvalta dessa under minst fem år,
varefter kåren äger använda tillgångarna till förmån för studenterna vid
LTH. I fall av en ny studentsammanslutning bildas inom dessa fem år, må
kåren överväga om tillgångarna skall överlåtas till den nya organisationen.

\end{stadgeavsnitt}

\end{document}
