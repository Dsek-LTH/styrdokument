\documentclass{dsekkallelse}

\usepackage[T1]{fontenc}
\usepackage[utf8]{inputenc}
\usepackage[swedish]{babel}
\usepackage{multicol}

\setheader{Policy för utmärkelsen DuCtig}{Policydokument}{}

\title{Policy för utmärkelsen DuCtig}
\author{Carolina Sartorius}

\begin{document}

\section{Policy för utmärkelsen DuCtig}

\textbf{Beskrivning}\\
DuCtig är en utmärkelse för att uppmärksamma D- och C-studenter som är DuCtiga, det vill
säga studenter som föregår med gott exempel genom att klara sina kurser.

\textbf{Historik}\\
Policyn är antagen på SRD 16 vt 2015. 

\textbf{Mål}\\
Målet med DuCtig är att få fler studenter att prioritera sin studier redan från början och därmed
öka andelen studenter som klarar minst 60 poäng per läsår (från programmet).

\textbf{Berättigade}\\
DuCtig delas ut retroaktivt på läsårsbasis. Under det läsåret ska studenten ha uppfyllt följande
kriterier:

\begin{itemize}
	\item Studenten ska ha varit registrerad på Datatekniksprogrammet eller Informations- och
Kommunikationstekniksprogrammet.
\item Studenten ska ha minst 60 poäng i avklarade kurser från det läsåret.
\item Studenten ska ha varit förstagångsregistrerad på de kurserna.
\item Kurserna ska vara från programmet.
\item Kurserna ska inte varit tillgodoräknade (undantag ges vid utbytesstudier).
\item Läsåret ska ha varit något av studentens tre första år på programmet.
\item Studenten skall inte tidigare mottagit utmärkelsen för detta läsår.
\end{itemize}


\textbf{Utdelning}\\
DuCtig delas ut av studierådet till de studenter som uppfyllde kriterierna föregående läsår. Studenter kan själva ansöka om att få utmärkelsen för tidigare läsår (genom att uppvisa Ladokutdrag), dock ska detta ske inom fem år från slutet av det berörda läsåret. DuCtig kan delas ut
maximalt tre gånger till en student.

\textbf{Utmärkelsen}\\
Utmärkelsen består av följande tre saker:
\begin{itemize}
	\item Tygmärke. Delas endast ut första gången en student får DuCtig.
	\item Diplom. Delas ut varje gång studenten får DuCtig.
	\item Medalj. Medaljen finns i tre olika valörer; brons, silver och guld. Brons delas ut första
gången, silver andra och guld sista. Studenten byter in sin medalj för att få nästa. 
\end{itemize}




\end{document}