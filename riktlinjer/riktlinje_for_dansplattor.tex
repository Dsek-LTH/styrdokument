\documentclass{dsekprotokoll}

\usepackage[T1]{fontenc}
\usepackage[utf8]{inputenc}
\usepackage[swedish]{babel}
\usepackage{multicol}

\setheader{Riktlinje för dansplattor}{Riktlinje}{}

\title{Riktlinje för dansplattor}
\author{Gillis Oldfeldt, Viktor Claesson}

\begin{document}
\maketitle
\section{Formalia}
\subsection{Sammanfattning}
Riktlinjen beskriver hur sektionens dansplattor får användas och uthyras.
\subsection{Syfte}
Våra dansplattor, inköpta i enlighet med beslut på HTM-2 2014, är robusta styrdon till en dator
eller konsol. De används primärt för att spela spel såsom Stepmania och Dance Dance Revolution. På sektionen har de sedan de köptes in enbart använts för regelbunden träning samt vid
pubar i iDét. Vid ett antal tillfällen har andra sektioner eller TLTH inkommit med förfrågningar
om plattorna går att låna/hyra ut. Dessa förfrågningar föranledde denna riktlinje.

\subsection{Omfattning}
Sektionen i sin helhet.
\subsection{Historik}
Riktlinjen är antagen på styrelsemöte S21 2017.
Uppdaterad enl. Policy för Policyer på HTM2 2021. Dokumentent omformades från att vara en policy till att bli en riktlinje på VTM-extra 2023.

\section{Bruk}
\begin{itemize}
    \item Dansplattorna får ej nyttjas utan tillsyn av en funktionär inom programmästeriet vid D-sektionen, alternativt av annan person som på förhand har godkänts som ansvarig av programmästeriet.
    \item Vid nyttjande av dansplattor får ej ytterskor bäras.
    \item Vid nyttjande av dansplattor får ingen dryck intas i deras absoluta närhet.
\end{itemize}


\section{Uthyrning}
\begin{itemize}
    \item Vid uthyrning måste ansökan inkomma till Programmästaren i god tid före önskat uthyrningsdatum. Minst två veckors framförhållning rekommenderas.
    \item Uthyrning må ej ske utan tillsyn av ansvarig funktionär enligt ovan. Denna person måste beredas tillträde till den lokal där plattorna befinner sig under uthyrningen.
    \item Efter uthyrning skall dansplattorna och tillhörande attiraljer återlämnas till D-sektionen i samma skick som när de hämtades.
    \item Kostnad för uthyrning är 500kr per gång. Då ingår 2st dansplattor med tillhörande attiraljer, samt funktionär. Kablaget tillåter inkoppling till dator, samt en uppsjö konsoller. Konsol, dator, spel eller musik ingår ej. Om medföljande funktionär har möjlighet kan denne bistå med detta. I annat fall får den som hyr ordna med sådant.
    \item det fall att en dansplatta eller medföljande attiralj går sönder under uthyrning utkräves en ersättning om 500kr plus kostnad för reparation. Om reparation ej är möjlig utkräves istället ersättning vilken täcker inköp av liknande produkt.
\end{itemize}


\subsection{Övrigt}
\begin{itemize}
    \item Dansplattornas attiraljer består av 2st yoga-plattor, 2st anti-glid-mattor, samt kablage.
    \item Försäljning av dansplattor av denna typ har upphört, likaså support och service. Därför faller det på sektionen att laga plattorna när de går sönder. Plattorna är enkla styrdon med fysiska kontaktytor, samt en styrkrets. De har tidigare lagats av ETF, vilka med stor sannolikhet kan laga dem igen mot låg kostnad.
\end{itemize}

\end{document}