\documentclass{dsekprotokoll}

\usepackage[T1]{fontenc}
\usepackage[utf8]{inputenc}
\usepackage[swedish]{babel}
\usepackage{graphicx}
\usepackage{wrapfig}

\newcommand{\datum}{2020--09--03}

\setheader{Riktlinje för Sektionsbilens användning}{Riktlinje}{Lund -- 3 maj 2023}

\author{Julia Karlsson}

\title{Riktlinje för Sektionsbilens användning}
\begin{document}

\maketitle
\section{Formalia}
\subsection{Sammanfattning}
Riktlinjen beskriver hur bilen som antingen är inhyrd eller ägd av sektionen bokas, och regler vid utlåning samt uthyrning. Sektionens bil är för tillfället en Volvo V50 med registreringsnummer WPW 221, namngiven \"yalle.

\section{Bokning}
Bokningssystemet sköts via hemsidan (https://www.dsek.se/booking) och tillhandahålls av Källarmästaren. I fall där bokning inte kan ske via hemsidan så kan det även göras i direkt samverkan med Källarmästaren via exempelvis personligt meddelande eller mejl. 

Om bilen används utan Källarmästarens tillåtelse så kan det vara underlag för entledigande då det bryter mot Policy för Sektionsbil. 

Endast funktionärer med, i Sverige, giltigt körkort får använda bilen i syfte av sektionsverksamhet. Längden på en bokning ska vara rimlig för det syfte bilen ska användas till, vilket ska specifieras i bokningen. Källarmästaren beslutar om en bokning är godtagbar enligt dessa krav.

\section{Nycklar}
Nycklarna till bilen förvaras på en säker plats och lämnas ut på passande sätt. För tillfället är de i kassaskåpet i Shäraton och koden till detta ges ut till berörda funktionärer vid bokning.

\section{Parkering}

Då \"yalle inte är bokad ska den vara på parkeringen vid Navet. Parkering utanför E-huset samt Navet betalas av sektionen, men längre parkering på E-husets parkering är inte tillåten eftersom bilen tidigare fått sin katalysator stulen där. 

En funktionär kan göra privat utlägg för parkering utanför LTH:s område. 

\begin{wrapfigure}[3]{r}{0.22\textwidth}
    \raggedright
    \vspace{-0.6cm}
    \includegraphics[width=0.2\textwidth]{nedsatt_skyddsläge.png}
\end{wrapfigure}

Vid parkering måste bilen sättas i \emph{Nedsatt skyddsläge} för att inte börja larma vid "extrema temperaturer". Detta aktiveras genom att trycka på knappen i taket, längst till vänster. När Nedsatt skyddsläge är aktiverat så lyser knappen orange. Knappen ser ungefär ut som bilden bredvid.

\section{Bensin} 
Om det är under en kvarts tank kvar i bilen \emph{måste} Källarmästaren eller Bilansvariga meddelas om detta. Om du har möjlighet, tanka bilen, spara kvittot och gör ett privat utlägg.

\section{Att köra bilen} 

\"yalle har en körjournal som måste fyllas i en gång för varje bokning. Den finns fysiskt i bilen, samt elektroniskt via Källarmästeriets Linktree (linktr.ee/kallarm) om det inte finns plats på de fysiska papprena. Låt Källarmästaren eller Bilansvariga veta om den fysiska körjournalen börjar bli fylld så att den kan ersättas. 

Du ska endast anteckna mätarställningen då \"yalle ska lämna Navets parkering för första gången, samt när \"yalle återvänt dit för sista gången under din bokning.

Viktigt när man kör: kör \emph{\textbf{försiktigt}} över gupp, då den nya katalysatorn är lite stor. Det är speciellt ett farthinder på östra sidan om E-huset som åtföljs av en grop som man ska vara försiktig vid. 

När du är klar med bilen, parkera på Navets parkering igen, fyll i körjournalen och aktivera nedsatt skyddsläge. Nycklarna ska låsas in i samma kassaskåp igen, och med det så är din bokning klar.


\section{Frågor?}
Vid frågor eller oklarheter i riktlinjen eller med bilen i övrigt, hör av dig till Källarmästaren via mejl (kallarm@dsek.se), eller till Bilansvariga.

\end{document}
%dokumentklass??
%beslut som grundar sig på möte. 



% Sektionsbilen info

Nyckeln finns i kassaskåpet i Shäraton, koden är än så länge samma som tidigare men vi har tänkt byta den inom kort. Man bokar bilen via hemsidan under Sektionen-boka nåt (https://www.dsek.se/booking) Om det är under en kvarts tank kvar i bilen MÅSTE Källarmästaren meddelas om detta. Om du kan, snälla tanka bilen (tack ängel) och gör ett privat utlägg, så får du tillbaka pengarna senare. Hör av dig om du inte vet hur du ska göra!

Sektionsbilen är på parkeringen vid navet, så det är bara att följa mittvägen precis förbi kårhuset, och sedan gå vänster vid det där trädet som är mitt i vägen. Regnumret står på bilnyckelns lapp, WPW221. 

Viktigt innan ni kör iväg: kolla mätarställningen och skriv ner körjournalen. Det finns också en digital körjournal om du hellre fyller i den: 
https://docs.google.com/forms/d/e/1FAIpQLSdY5cN7OlorvnM9I-Nh-5Dsqy9CHrkMDdKQlebt9cEyO-aNRg/viewform

Viktig sak när man kör: kör LÅNGSAMT över guppar, det är en del under bilen som kan gå sönder annars (inte farligt för körandet) 

Och viktigt när man parkerar, se till att bilen är i ”nedsatt skyddsläge” genom att trycka på en knapp till vänster i taket  (det finns en lapp i bilen som har en ritning av knappen) annars går billarmet för att luften är för kall

När ni är klara med allt, parkera den på samma parkering, fyll i körjournalen igen med slutmätarställningen och lås in nyckeln i kassaskåpet.

