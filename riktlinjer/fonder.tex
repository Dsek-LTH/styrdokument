\documentclass{dsekkallelse}

\usepackage[T1]{fontenc}
\usepackage[utf8]{inputenc}
\usepackage[swedish]{babel}
\usepackage{multicol}

\setheader{Policy fonder}{Policydokument}{}

\title{Policy för tackverksamhet}
\author{Fred Nordell}

\begin{document}

\section{Policy Fonder}

\subsection{1. Bakgrund och syfte}
Sektionen har ibland behov av att avsätta pengar till större projekt genom att sätta in dem i fonder.
Denna policy är till för att vägleda hur det går till att spara och nyttja pengar i en fond.

\subsubsection{1.1 Historik}
Policyn är antagen HTM2 2018 och uppdaterad HTM1 2020. 

\subsection{2. D-sektionen fonder}
D-sektionen fonderar medel via sektionsmötet eller vid bokslutsdispositionen, öronmärkta för
att användas till vissa specifika saker. Nedan följer en lista på D-sektionen fonder och vilket organ som disponerar respektive fond tillsammans med en fondavsättningsstrategi. Om inköp inte passar
någon fond ska detta inköp belasta verksamhetsårets resultat.

\subsubsection{2.1 Sektionsfond}
Syftet med sektionsfonden är att sektionen ska kunna göra långsiktiga investeringar i inventarier och lokaler. Fonden disponeras av styrelsen. Uttag ur fonden ska presenteras
till sektionsmöte.

Fonden består av de medel som avsattes av beslut på HTM-1 2018; Ytterligare medel tillskjutna av
sektionsmöte eller annan genom frivilliga bidrag.

Fonden disponeras av Sektionsstyrelsen. Uttag ur fonden ska presenteras till sektionsmöte.

\subsection{3 Fondavsättningstrategi}
D-sektionen fondavsättningsstrategi bör användas som underlag för verksamhetens resultatsdispositioner, beroende på hur D-sektionen ekonomiska resultat ser ut.


\paragraph{Scenario 1}  När D-sektionen genererar mer överskott än fonduttagen gjorda för verksamhetsåret ska D-sektionen i första hand disponera överskottet till de fonderna där medel använts.
Disponeringen ska motsvara summan av de använda medlen. Resterande överskott ska disponeras av
sektionsmötet.


\paragraph{Scenario 2} När D-sektionen genererar överskott som är lägre eller motsvarande fonduttagen
gjorda för verksamhetsåret ska sektionsmötet disponera överskottet till de fonder som anses lämpliga.

\subsection{4 Etiska hänsynstagningar och risker}

D-sektionen bör följa Teknologkårens policy för ekonomi när det kommer till etiska hänsynstagningar och riskaspekter kring sparande.




\end{document}