\documentclass{dsekprotokoll}

\usepackage[T1]{fontenc}
\usepackage[utf8]{inputenc}
\usepackage[swedish]{babel}
\usepackage{graphicx}
\usepackage{wrapfig}

\newcommand{\datum}{2020--09--03}

\setheader{Riktlinje för anmodanden och inbjudningar}{Riktlinje}{Lund -- 1 oktober 2023}

\author{Axel Svensson}

\title{Riktlinje för anmodanden och inbjudningar}
\begin{document}

\maketitle
\section{Formalia}
\subsection{Sammanfattning}
Riktlinje för anmodanden och inbjudningar beskriver vilka och i vilken prioritetsordning personer bör anmodas och
bjudas in till vissa tillställningar.

\subsection{Syfte}
Riktlinjens syfte är att ge ett lätt och konsekvent sätt för anordnare av sittningar och baler att veta vem som bör anmodas och bjudas in.

\subsection{Omfattning}
D-sektionen i sin helhet med undantag för Teknikfokus.

\subsection{Historik}
Antagen på S22 2023.

\section{Terminologi}

\subsection{Anmodan}
En anmodan till en tillställning är ett erbjudande att köpa
biljett till den. För det mesta utan vidare formaliteter, men
mer exklusiva gäster bör tillställas en formell anmodan.

\subsection{Inbjudan}
En inbjudan är det samma som en anmodan fast att anordnaren står för biljettpriset; en erbjudan att delta. Eftersom
inbjudan oftast är mer exklusiv än anmodan är de flesta inbjudningar formella.
I enlighet med svensk alkohollag betalar inte arrangören priset för eventuell förköpbar alkohol (t.ex. vinpaket, punschbiljetter).




\section{Nollningen}

\subsection{Icke-gasque-sittningar under nollningen} \label{icke-gasque}

\subsubsection{Anmodanden}
Till en icke-gasque-sittning under nollningen ska följande
anmodas i den turordning som följer i mån av plats:

\begin{itemize}
    \item utsedda nyktra och ansvariga phaddrar,
    \item øverphøs och stabsmedlemmar,
    \item øverpeppare och peppare,
    \item nollor,
    \item grupphaddrar,
    \item övriga phaddrar, samt
    \item styrelseledamöter.
\end{itemize}

\subsection{Lär-känna-gasque (LKG)}

\subsubsection{Inbjudningar}
Till LKG ska följande erhålla en inbjudan:
\begin{itemize}
    \item inspektorn och inspektrix.
\end{itemize}

\subsubsection{Anmodanden}
Till LKG gäller samma prioritetsordning för anmodan som
i \ref{icke-gasque}


\subsection{Nollegasque}

\subsubsection{Inbjudningar}
Till nollegasque ska följande erhålla en inbjudan:
\begin{itemize}
    \item inspektorn och inspektrix, samt
    \item hedersmedlemmar
\end{itemize}

\subsubsection{Anmodanden}
Till nollegasque ska följande anmodas i den turordning som
följer i mån av plats:
  \begin{itemize}
    \item utsedda ansvariga phaddrar,
    \item øverphøs och stabsmedlemmar,
    \item øverpeppare och peppare,
    \item nollor,
    \item grupphaddrar,
    \item styrelseledamöter,
    \item övriga phaddrar,
    \item nollningsfunktionärer,
    \item heltidare vid Teknologkåren från D-sektionen,
    \item sektionens kårkontakter,
    \item kårens nollningsutskott,
    \item övriga aktiva medlemmar, samt
    \item passiva medlemmar.
  \end{itemize}



\section{Andra sittningar}

\subsection{Skiphtesgasque}

\subsubsection{Inbjudningar}
Till skiphtesgasquen ska följande erhålla en inbjudan:

\begin{itemize}
    \item tillträdande styrelseledamöter,
    \item avgående styrelseledamöter,
    \item stabsmedlemmar,
    \item tillträdande övermarskalk,
    \item avgående övermarskalk,
    \item inspektorn och inspektrix, samt
    \item hedersmedlemmar.
\end{itemize}

\subsubsection{Anmodanden}

Till skiphtesgasquen ska följande anmodas i den turordning som följer i mån av plats:
  \begin{itemize}
    \item tillträdande funktionärer,
    \item avgående funktionärer,
    \item före detta styrelseledamöter,
    \item övriga aktiva medlemmar, samt
    \item passiva medlemmar.
  \end{itemize}


\section{Övrigt}

\subsection{Mottagare av utmärkelser}
Person som ska utdelas utmärkelser (t.ex. medalj eller gyllene pekpinnen) ska inbjudas alternativt anmodas.

\subsection{Ordenskapitel}
Inspektorn och inspektrix ska inbjudas och övermarskalken
ska inbjudas alternativt anmodas om sittningen är kopplad
till ett ordenskapitel.

\subsection{Övriga}
Andra personer kan inbjudas/anmodas beroende på tillställning.

\subsection{Respektive (plus ett)}
Sexmästaren (eller eventuell annan arrangör av evenemanget) avgör huruvida personer anmodas med respektive.




\end{document}
%dokumentklass??
%beslut som grundar sig på möte. 



% Sektionsbilen info

Nyckeln finns i kassaskåpet i Shäraton, koden är än så länge samma som tidigare men vi har tänkt byta den inom kort. Man bokar bilen via hemsidan under Sektionen-boka nåt (https://www.dsek.se/booking) Om det är under en kvarts tank kvar i bilen MÅSTE Källarmästaren meddelas om detta. Om du kan, snälla tanka bilen (tack ängel) och gör ett privat utlägg, så får du tillbaka pengarna senare. Hör av dig om du inte vet hur du ska göra!

Sektionsbilen är på parkeringen vid navet, så det är bara att följa mittvägen precis förbi kårhuset, och sedan gå vänster vid det där trädet som är mitt i vägen. Regnumret står på bilnyckelns lapp, WPW221. 

Viktigt innan ni kör iväg: kolla mätarställningen och skriv ner körjournalen. Det finns också en digital körjournal om du hellre fyller i den: 
https://docs.google.com/forms/d/e/1FAIpQLSdY5cN7OlorvnM9I-Nh-5Dsqy9CHrkMDdKQlebt9cEyO-aNRg/viewform

Viktig sak när man kör: kör LÅNGSAMT över guppar, det är en del under bilen som kan gå sönder annars (inte farligt för körandet) 

Och viktigt när man parkerar, se till att bilen är i ”nedsatt skyddsläge” genom att trycka på en knapp till vänster i taket  (det finns en lapp i bilen som har en ritning av knappen) annars går billarmet för att luften är för kall

När ni är klara med allt, parkera den på samma parkering, fyll i körjournalen igen med slutmätarställningen och lås in nyckeln i kassaskåpet.

