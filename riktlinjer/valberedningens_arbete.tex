\documentclass{dsekprotokoll}

\usepackage[T1]{fontenc}
\usepackage[utf8]{inputenc}
\usepackage[swedish]{babel}
\usepackage{multicol}

\setheader{Policy för valberedningens arbete}{Policydokument}{17 september 2019}

\title{Policy för valberedningens arbete}
\author{}

\begin{document}
\maketitle

\section{Formalia}
\subsection{Sammanfattning}
Policyn beskriver hur valberedningen skall arbeta med kandidater, sekretess, information samt
hur de skall agera på sektions-/styrelsemöten.

\subsection{Syfte}
Syftet med denna policy är att förtydliga stadgan och reglementet samt att vägleda valberedningen i sitt arbete.

\subsection{Omfattning}
Valberedningen omfattas.

\subsection{Ägande}
Sektionsmötet äger policyn.
\subsection{Historik}
Utkast färdigställt av: – \\
Ursprungligen antagen enligt beslut: S08 2014\\
Omarbetning färdigställd av: Framtidsutskottet genom Anna Qvil\\
Fastställd omarbetning enligt beslut: VTM 2019, HTM1 2020. \\
Uppdaterad enl. Policy för policyer på HTM2 2021 av Kaspian Jakobsson


\section{Valberedningens allmänna arbete}

Inför första valprocessen ska valberedningen strukturera upp allmänt hur både intervjuprocessen och den
interna valprocessen går till.

Yttrande i valberedningens namn skall baseras på konsensus inom hela valberedningen om
så är möjligt. Om konsensus ej kan nås inom valberedningen skall ett majoritetsbeslut avgöra
uttalandet samt att det skall presenteras att valberedningen är oendad.
Om någon representant i valberedningen på något vis anser sig själv eller någon annan representant i valberedningen som partisk rörande en kandidat eller en post så ska denne vara
öppen med detta inför resterande valberedning.


Efter varje valprocess ska valberedningen utvärdera hela processen och använda den utvärderingen som grund för nästa valprocess.

\section{Sekretess och etik}
Valberedningens intervjuer sker i förtroende och är sekretessbelagda. Valberedningens medlemmar får ej delge denna information utanför valberedningen. Valberedningen har rätt att till
det väljande mötet delge relevant information som framkommit under intervjuprocessen för
de som kandiderar.

\section{Nominering}
Nominering respektive kandidatur sker, och presenteras, genom sektionens informationskanaler. Valberedningen får under nomineringstiden ej uppge icke publik information om nomierade. Om en medlem i valberedningen själv ämnar kandidera till en valberedd post skall
denna meddela detta innan nomineringsperioden och stå utanför; beslut, intervjuer och diskussion rörande posten och dess kandidater.

\section{Kandidatur}
Valberedningen fastställer formerna för intervjun. Intervjuer mellan kandidater till samma post
skall hållas likartat. Valberedningens intervju skall avslutas med information om valförfarandet, som:
\begin{itemize}
	\item Tid och plats för det väljande mötet
	\item När och hur de kan förvänta sig besked angående valberedningens förslag
	\item Möjligheten att motkandidera och hur detta går till
	\item Det typiska valförfarandet
\end{itemize}

Valberedningsordförande skall även se till att det erbjuds feedback till kandidaterna samt gå
ut med resultatet till kandidater innan resultatet offentliggörs.

\section{Valberedingens förslag}
Innan sista dagen för inlämning av handlingar till respektive valmöte åligger det valberedningens ordförande att sammankalla valberedningen till möte för att besluta och lämna in förslag
på kandidater till poster som ska väljas.

\section{Feedback}
Feedback med innehållet Valberedningen ska förbereda feedback för samtliga kandidater inför det att valberedningens nominering
släpps. Kandidater ska kunna begära ut denna feedback upp till 14 dagar efter det att valberedningens nominering har gått ut.

Feedbacken ska innehålla information om varför man blev eller inte blev valberedningens nominering samt feedback på intervjun. Feedbacken ska vara individuell och därför inte vara kopplad eller
jämföras med andra kandidater.

\section{Sektionsmöte}
Valberedningen bör inleda sektionsmötets diskussion med motivering av valberedningens förslag. Är valberedningen oenig i detta förslaget skall mötet informeras om detta. Förslagen skall
motiveras med en kort text som speglar valberedningens samlade intryck från intervjun. Motiveringarna skall vara informativa och kärnfulla.

\section{Styrelsemöte}
På styrelsemöte skall tillvägagången redovisas samt att då det gäller en grupp som väljs skall
en motivering ges kring dess sammansättning.

\section{Processgång}
Valberedningsordförande har ansvar för att aktivt arbeta för en öppnare processgång.
\end{document}