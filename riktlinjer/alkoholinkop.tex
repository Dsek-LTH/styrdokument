\documentclass{dsekkallelse}

\usepackage[T1]{fontenc}
\usepackage[utf8]{inputenc}
\usepackage[swedish]{babel}
\usepackage{multicol}

\setheader{Policy för D-sektionens alkoholinköp}{Policydokument}{Antagen S31 2014-12-09}

\title{Policy för D-sektionens alkoholinköp}
\author{xx}

\begin{document}

\section{Policy för D-sektionens alkoholinköp}

D-sektionen är en förening. Vid alkoholinköp gäller samma lagar som reglerar
företags alkoholinköp. Denna policy är till för att sammanställa vad som gäller för D-sek när
det kommer till inköp av alkohol och vad lagen säger eftersom dessa är strängare än för privatpersoner. Med alkoholinköp avser både det som är till försäljning och det som inte är det.
Företag får endast köpa alkohol av godkända svenska återförsäljare. Alkoholinköp från annat
land än Sverige får endast ske av behöriga företag, vilket D-sek inte är, och om alkoholskatt
betalas. Enligt alkohollagen (4 kap, §4) och smugglingslagen (§3), riskerar enskilda medlemmar att dömas för smugglingsbrott bara genom att följa med om en förening skulle importera
alkohol. Sektionen riskerar även att dömas för bokföringsbrott och/eller skattebrott genom att
köpa alkohol för föreningens pengar i annat land än Sverige. Sektionen skulle även förlora sitt
serveringstillstånd, vilket skulle vara väldigt tråkigt, därför gäller följande:
\begin{attlista}
	\item styrelsen inte kommer godkänna utläggsräkningar på inköp av alkohol på några andra ställen än godkända svenska återförsäljare.
\item försäljning av alkohol på evenemang i sektionens namn endast får ske då ett serveringstillstånd finns, detta gäller även om alkoholen inte skulle vara inköpt av sektionen.

\end{attlista}



Lund, dag som ovan\newline

Louise Hauzenberger\\
Skattmästare


\end{document}